%
% newton.tex
%
% (c) 2020 Prof Dr Andreas Müller, Hochschule Rapperswil
%
\section{Newton-Verfahren
\label{buch:section:newtion}}
\rhead{Newton-Verfahren}
\index{Newton-Verfahren}%
Die bisher vorgestellten Verfahren zur Bestimmung einer Nullstelle
$x^*$ der Funktion $f$, $f(x^*)=0$ sind linear konvergent und damit
eher langsam.
\index{Nullstelle}%
Dafür war nicht mehr als Stetigkeit nötig, um Konvergenz des Verfahrens
sicherzustellen.

%
% Analytischer Ansatz
%
\subsection{Analytischer Ansatz für ein quadratisch konvergentes Verfahren
\label{buch:subsection:newton:analytisch}}
In Abschnitt~\ref{buch:subsection:linearekonvergenz} wurde dargestellt,
dass nach Möglichkeit quadratische Konvergenz angestrebt werden sollte.
\index{quadratische!Konvergenz}%
\index{Konvergenz!quadratisch}%
Quadratische Konvergenz könnte in einer Fixpunktiteration $x_{n+1}=g(x_n)$
erreicht werden, wenn die Ableitung $g'(x^*)=0$ ist.
\index{Fixpunkt}%

Je grösser $f(x_n)$ ist, desto weiter dürfte $x_n$ von der Nullstelle
entfernt sein.
Wir versuchen daher, die Approximation $x_{n+}$ proportional zum Wert von
$f(x_n)$ zu korrigieren mit Hilfe der Funktion
\[
x_{n+1} = g(x_n) = x_n - a(x_n)\cdot f(x_n).
\]
Die Funktion $a(x_n)$ muss noch bestimmt werden.
Sie soll so gewählt werden, dass die Konvergenz quadratisch wird, was
mit $g'(x^*)=0$ erreicht wird.

Die Ableitung von $g$ ist
\begin{align*}
g'(x)
&=
1-a'(x)f(x)-a(x)f'(x).
\intertext{An der Stelle $x^*$ gilt}
g'(x^*)
&=
1-a'(x^*)\underbrace{f(x^*)}_{\displaystyle=0} - a(x^*)f'(x^*) = 0
\\
\Leftrightarrow\qquad
1
&=
a(x^*) f'(x^*)
\qquad\Rightarrow\qquad
a(x)=\frac{1}{f'(x)}.
\end{align*}
Damit finden wir das im folgenden Satz beschriebene Verfahren mit
quadratischer Konvergenz.

\begin{satz}[Newton-Verfahren]
\index{Newton-Verfahren}%
\label{buch:satz:newton-verfahren}
Hat die differenzierbare Funktion $f$ eine Nullstelle $x^*$ und gilt
$f'(x^*)\ne 0$, dann konvergiert die Iterationsfolge
\begin{equation}
x_{n+1} = x_n - \frac{f(x_n)}{f'(x_n)} 
\label{buch:equation:newtoniteration}
\end{equation}
für Startwerte $x_0$ genügend nahe bei $x^*$ quadratisch gegen die
Nullstelle.
\end{satz}
\index{Startwert}%

Das Sekantenverfahren hat darunter gelitten, dass die Berechnung
der der nächsten Approximation mit Hilfe der Formel 
\eqref{buch:eqn:sekanten-iteration} von umso stärker Auslöschung
geplagt ist, je näher man bereits an der Lösung ist.
\index{Sekantenverfahren}%
Die Iterationsformel
\eqref{buch:equation:newtoniteration}
für die Newton-Iteration hat dieses Problem nicht.
In \eqref{buch:equation:newtoniteration} wird der aktuelle Wert $x_n$
um einen Korrekturbetrag korrigiert, der proportional zu $f(x_n)$
verkleinert wird.
Je genauer der Wert $x_n$ schon ist, desto kleiner wird auch die
Korrektur.

%
% Geometrische Interpretation
%
\subsection{Geometrische Interpretation des Newton-Verfahrens}
\begin{figure}
\centering
\includegraphics{chapters/20-gleichungen/figures/newton.pdf}
\caption{Graphische Interpretation des Newton-Verfahrens.
In jedem Iterationsschritt wird die bisherige Approximation $A$
mit Hilfe einer Tangente vom Funktionswert $B$ zum Punkt $C$
korrigiert, $\overline{AC}=f(x_0)/f'(x_0)$.
\label{buch:figure:newton}}
\end{figure}
Die Iterationsformel~\eqref{buch:equation:newtoniteration}
lässt sich sehr schön graphisch interpretieren.
In Abbildung~\ref{buch:figure:newton} wird die Nullstelle der
Funktion $f(x) = e^x-\frac12$ mit dem Newton-Verfahren bestimmt.
Im Iterationsschritt wird die Approximation $x_n$ korrigiert
nach der Formel
\[
x_{n+1}
=
x_n - \frac{f(x_n)}{f'(x_n)}
=
x_n - \frac{e^{x_n}-\frac12}{e^{x_n}}
=
x_n - \biggl(1 -\frac{e^{-x_n}}{2}\biggr)
\]
Die Korrektur für $n=0$ ist in Abbildung~\ref{buch:figure:newton}
als Grundseite des rechtwinkligen Dreiecks $ABC$ erkennbar.
Die Hypothenuse hat die Steigung $f'(x_0)$, daher ist
\index{Hypothenuse}%
\index{Dreieck}%
\[
\overline{AC}\cdot f'(x_0) = f(x_0)
\qquad\Rightarrow\qquad
\overline{AC} = \frac{f(x_0)}{f'(x_0)}.
\]
Die vom Newton-Verfahren berechnete Korrektur ist also die optimale
Korrektur, die sich aus Funktionswert und erster Ableitung
an der Stelle $x_n$ berechnen lässt.
\index{Ableitung}%

%
% Wurzeln
%
\subsection{Wurzeln}
Als Beispiel für die Anwendung des Newton-Verfahrens berechnen wir die
$k$-te Wurzel einer positiven reellen
Zahl $a$, wir lösen also die Gleichung
$x^k = a$.
\index{Wurzel}%
Dies ist gleichbedeutend damit, eine Nullstelle der Funktion
$f(x)=x^k-a$ zu bestimmen.
Die Ableitung von $f$ ist
$f'(x)=kx^{k-1}$, woraus wir die Iterationsformel des
Newton-Verfahrens ablesen können:
\[
x_{n+1} = x_n - \frac{f(x_n)}{f'(x_n)}=x_n - \frac{x_n^k-a}{kx_n^{k-1}}
=
\frac{1}{k}\biggl((k-1)x_n+\frac{a}{x_n^{k-1}}\biggr).
\]
Im Falle $n=2$ finden wir das bereits in 
Abschnitt~\ref{buch:subsection:linearekonvergenz}
untersuchte, quadratisch konvergente Verfahren zur Bestimmung
der Quadratwurzel wieder.
\index{Quadratwurzel}%
In Tabelle~\ref{buch:table:wurzel5newton} ist das Verfahren für
$a=10$, $k=5$ und $x_0=a$ gezeigt.
Quadratische Konvergenz stellt sich allerdings erst bei $x_{10}$ ein,
der Startwert $x_0$ ist zu weit von der Lösung entfernt.

\begin{table}
\centering
\renewcommand\arraystretch{1.15}
\begin{tabular}{|>{$}r<{$}|>{$}r<{$}|}
\hline
n& x_n\\
\hline
 0 & 10.00000000000000\\
 1 &  8.00020000000000\\
 2 &  6.40064823242493\\
 3 &  5.12171019598693\\
 4 &  4.10027465454472\\
 5 &  3.28729556684556\\
 6 &  2.64696320430731\\
 7 &  2.15831219143923\\
 8 &  \underline{1}.81881622015378\\
 9 &  \underline{1.}63781027109793\\
10 &  \underline{1.58}820394873794\\
11 &  \underline{1.5849}0696686523\\
12 &  \underline{1.584893192}70054\\
13 &  \underline{1.58489319246111}\\
\hline
\infty&\sqrt[5]{10}=1.58489319246111\\
\hline
\end{tabular}
\caption{Berechnung von $\sqrt[5]{10}$ mit dem Newton-Verfahren.
Der Startwert $x_0=10$ ist sehr weit von der Lösung entfernt, so dass es
einige Iterationen braucht, bis die Konvergenz quadratisch wird.
\label{buch:table:wurzel5newton}}
\end{table}

%
% Newton-Verfahren in R^n
%
\subsection{Newton-Verfahren in $\mathbb R^n$
\label{buch:section:newtoninRn}}
\index{Newton-Verfahren!in $\mathbb R^n$}%
In der bisher beschriebenen Form erlaubt das Newton-Verfahren, 
Nullstellen von rellwertigen Funktion zu finden.
Es eignet sich nicht, Vektorgleichungen zu lösen.
\index{Vektorgleichung}%

Sei daher im folgenden $f\colon \mathbb R^k \to \mathbb R^k$ eine
Vektorfunktion mit einer Nullstelle $x^*$, $f(x^*)=0$, die numerisch gefunden 
werden soll.
Wie in Abschnitt~\ref{buch:subsection:newton:analytisch} soll eine
Approximation $x_n$ für die Nullstelle $x^*$ proportional zur Grösse
von $f(x_n)$ korrigiert werden.
Eine skalare Funktion $a(x)$ wird aber im allgemeinen zu wenig
allgemein für eine performante Lösung des Problem sein, daher
wird für $A(x)$ eine matrixwertige Funktion mit Werten in
$\operatorname{GL}_k(\mathbb R)$ gewählt.
\index{Funktion!matrixwertig}%
\index{GLkR@$\operatorname{GL}_k(\mathbb R)$}%
Wir setzen also an
\[
x_{n+1} = g(x_n) = x_n - A(x_n) f(x_n)
\]
und versuchen wie früher $A(x)$ so zu wählen, dass die Ableitung von $g$
an der Stelle $x^*$ verschwindet.

Die Ableitung von $g(x)=x-A(x)f(x)$ ist
eine lineare Abbildung, die auf dem Vektor $h\in\mathbb R^k$ den Wert
\begin{align*}
Dg(x) \cdot h
&=
h - (DA(x)\cdot h) f(x) - A(x) Df(x)\cdot h
\end{align*}
hat.
An der Stelle $x=x^*$ verschwindet der mittlere Term wegen $f(x^*)=0$, so
dass als Gleichung für $A(x)$
\[
0=h-A(x) Df(x) \cdot h
\qquad\Rightarrow\qquad
h = A(x) Df(x) \cdot h\qquad \forall h\in\mathbb R^n
\]
übrigbleibt.
Dies ist nur möglich, wenn $A(x)$ die inverse Matrix von $Df(x)$ ist,
was den folgenden Satz motiviert.

\begin{satz}[Newton-Verfahren für Vektorgleichungen]
\label{buch:satz:newtonVektorgleichungen}
\index{Newton-Verfahren!für Vektorgleichungen}
Hat die Funktion $f\colon\mathbb R^k\to\mathbb R^k$ eine Nullstelle
$x^*\in\mathbb R^n$, dann ist die Folge 
\[
x_{n+1} = x_n - Df(x_n)^{-1}\cdot f(x_n)
\]
für Startwerte $x_0$ nahe genug an $x^*$ quadratisch konvergent mit
Grenzwert $x^*$.
\end{satz}

\begin{proof}[Beweis]
Es ist klar, das $x^*$ ein Fixpunkt der Abbildung
\[
g(x)=x-Df(x)^{-1}\cdot f(x)
\]
ist.
Wir müssen nur noch zeigen, dass der Fehler der Iteration quadratisch
abnimmt.
Dazu entwickeln wir $f$ um den Punkt $x^*$ in eine Taylor-Reihe
\begin{align*}
f(x^* + \delta)
&=
f(x^*) + Df(x^*)\cdot \delta + O(|\delta|^2),\qquad\delta\in\mathbb R^k
\\
&=
Df(x^*)\cdot \delta + O(|\delta|^2)
\end{align*}
wegen $f(x^*)=0$.
Die Iteration ist
\begin{align}
x_{n+1}
&=
x^* +\delta_{n+1}
=
g(x^*+\delta_n)
=
x^*+\delta_n  - Df(x_n)^{-1}\cdot f(x_n)
\notag
\\
&=
x^* + \delta_n -Df(x_n)^{-1}\cdot f(x^* + \delta_n)
\notag
\\
&=
x^* + \delta_n -Df(x_n)^{-1}\cdot
(Df(x^*)\cdot \delta_n + O(|\delta_n|^2).
\label{buch:equation:newtonn:iteration}
\end{align}
Um \eqref{buch:equation:newtonn:iteration}
zu berechnen, muss man auch $Df(x_n)$ in eine Taylor-Reihe entwickeln, sie ist
\index{Taylor-Reihe}%
\[
Df(x_n)
=
Df(x^*+\delta_n)
=
Df(x^*) + D^2f(x^*)\cdot\delta_n.
\]
Setzt man dies in~\eqref{buch:equation:newtonn:iteration} ein, erhält man
\begin{align*}
x_{n+1}
=
x^*+\delta_{n+1}
&=
x^* + \delta_n -
(Df(x^*) + D^2f(x^*)\cdot\delta_n)^{-1}
(Df(x^*)\cdot \delta_n + O(|\delta_n|^2)
\\
&=
x^* + \delta_n -
(Df(x^*)^{-1} + O(|\delta_n|))
\cdot
(Df(x^*)\cdot \delta_n + O(|\delta_n|^2)
\\
&=
x^* + \delta_n
- \underbrace{Df(x^*)^{-1}Df(x^*)}_{\displaystyle = E}\mathstrut\cdot\delta_n
+
O(|\delta_n|^2)
\\
&=x^* + O(|\delta_n|^2).
\end{align*}
Der Fehler $\delta_{n+1}=O(|\delta_n|^2)$ nimmt somit quadratisch ab
und damit ist gezeigt, dass die Iterationsfolge quadratrisch
konvergiert.
\index{Iterationsfolge}%
\end{proof}

\begin{beispiel}
Es sollen die Polarkoordinaten des Punktes $(x,y)$
\index{Polarkoordinaten}%
als Lösung der Gleichung
\[
\begin{pmatrix}
r\cos\varphi\\r\sin\varphi
\end{pmatrix}
=
\begin{pmatrix}x\\y\end{pmatrix}
\qquad\Rightarrow\qquad
f(r,\varphi) =\begin{pmatrix}r\cos\varphi -x \\ r\sin\varphi -y \end{pmatrix}
=0
\]
bestimmt werden.
Dies ist ein Beispiel für die Dimension $k=2$.

Die Ableitungsmatrix von $f$ ist
\index{Ableitungsmatrix}%
\[
Df(r,\varphi)
=
\begin{pmatrix}
\cos\varphi&-r \sin\varphi\\
\sin\varphi&\phantom{-} r \cos\varphi
\end{pmatrix}
\qquad\Rightarrow\qquad
Df(r,\varphi)^{-1}
=
\begin{pmatrix}
\cos\varphi&\sin\varphi\\
-\frac1r\sin\varphi&\frac1r\cos\varphi
\end{pmatrix}.
\]
Die Iterationsformel wird jetzt
\begin{align}
\begin{pmatrix}
r_{n+1}\\
\varphi_{n+1}
\end{pmatrix}
&=
\begin{pmatrix}r_n\\\varphi_n\end{pmatrix}
-
\begin{pmatrix}
\cos\varphi_n&\sin\varphi_n\\
-\frac1{r_n}\sin\varphi_n&\frac1{r_n}\cos\varphi_n
\end{pmatrix}
\begin{pmatrix}
r_n\cos\varphi_n-x\\
r_n\sin\varphi_n-y
\end{pmatrix}
\notag
\\
&=
\begin{pmatrix}
x\cos\varphi_n+y\sin\varphi_n\\
\varphi_n -\frac{x}{r_n}\sin\varphi_n+\frac{y}{r_n}\cos\varphi_n
\end{pmatrix}
\label{buch:equation:polar}
\end{align}
\begin{table}
\centering
\begin{tabular}{|>{$}r<{$}|>{$}r<{$}>{$}r<{$}|}
\hline
n &                r_n               &                    \varphi_n    \\
\hline
0 &              3.1415926535897931  &              0.3678794411714423 \\
1 &              0.8067763763745672  &              0.5559550343664228 \\
2 &   \underline{0.9}030213557712843 &   \underline{1.0}884392177149671 \\
3 &   \underline{0.99}60918007116625 &   \underline{0.99}06298199545209 \\
4 &   \underline{0.9999}561001841604 &   \underline{1.0000}366265580796 \\
5 &   \underline{0.999999999}3292477 &   \underline{0.99999999}83920385 \\
6 &   \underline{1.0000000000000000} &   \underline{1.0000000000000000} \\
\hline
\infty& 1.0000000000000000 &   1.0000000000000000 \\
\hline
\end{tabular}
\caption{Quadratische Konvergenz des Iterationsverfahrens
\eqref{buch:equation:polar}
zur Bestimmung der Polarkoordinaten 
\label{buch:figure:newtonpolar}}
\end{table}%
Die Resultate der Iteration~\eqref{buch:equation:polar} für 
den Punkt $(x,y)=(\cos 1,\sin 1)$ ist in 
Tabelle~\ref{buch:figure:newtonpolar} gegeben.
Die quadratische Konvergenz ist wieder deutlich erkennbar.
\end{beispiel}

%
% Der Fahll f'(x)=0
%
\subsection{Der Fall $f'(x^*)=0$
\label{buch:subsection:newton0}}
Im Fall $f'(x^*)=0$ versagt die Iterationsformel des Newton-Verfahrens.
Es ist damit zu rechnen, dass das Verfahren sehr langsam oder gar nicht
konvergiert.
Wir untersuchen dies mit Hilfe einer Entwicklung der Funktion $f$
um den Punkt $x^*$:
\[
f(x^*+\delta)
=
x^*
+
\frac12f''(x^*)\delta^2 + \frac16f'''(x^*)\delta^3+ O(\delta^4).
\]
Für den Fehler $\delta_n$ der Approximation $x_n=x^*+\delta_n$ folgt die
Iteration
\begin{align*}
x^*+\delta_{n+1}
=
x_{n+1}
&=
x_n - \frac{f(x_n)}{f'(x_n)}
=
x^*+\delta_n
-
\frac{
\frac12f''(x^*)\delta_n^2+O(\delta_n^4)
}{
f''(x^*)\delta_n + \frac12f'''(x^*)\delta_n^2) + O(\delta^3)
}
\\
&=
x^* + \delta_n
-
\frac12
\delta_n
\frac{1+O(\delta_n)}{1+O(\delta_n)}
=
x^* + \delta_n
-
\frac12
\delta_n
(1+O(\delta_n))
\\
&=
x^* +\frac12\delta_n + O(\delta_n^2)
\\
\Rightarrow\qquad
\delta_{n+1} &= \frac12\delta_n + O(\delta_n^2)
\end{align*}
Der Fehler halbiert sich in jeder Interation.
Die Folge $(x_n)_{n\in\mathbb N}$ konvergiert also immer noch,
aber die Konvergenz ist nur noch linear.

%
% Vergleich mit dem Sekantenverfahren
%
\subsection{Vergleich mit dem Sekantenverfahren
\label{buch:subsection:newtonsekanten}}
\index{Sekantenverfahren}%
Die Ähnlichkeit des Newton-Verfahrens mit dem Sekantenverfahren ist 
unübersehbar.
Um dies deutlich zu machen, berechnen wir den Grenzfall $x_{n-1}\to x_n$
mit Hilfe der Form~\eqref{buch:sekante:stabil}.
%\begin{align*}
%x_{n+1}
%&=
%\frac{x_{n-1}f(x_n)-x_nf(x_{n-1})}{f(x_{n})-f(x_{n-1})}
%=
%\frac{x_{n-1}f(x_n){\color{darkred}\mathstrut-x_{n-1}f(x_{n-1})+x_{n-1}f(x_{n-1})}-x_nf(x_{n-1})}{f(x_{n})-f(x_{n-1})}
%\\
%&=
%x_{n-1} - f(x_{n-1})\frac{x_n-x_{n-1}}{f(x_n)-f(x_{n-1})}.
%\intertext{
Beim Grenzübergang $x_{n-1}\to x_n$ geht der Quotient auf der rechten
Seite in den Kehrwert der Ableitung $f'(x_n)$ über.
\index{Grenzübergang}%
Der Grenzfall des Sekantenverfahrens ist daher
%}
\begin{align*}
x_{n+1}
&=
x_n -\frac{f(x_n)}{f'(x_n)},
\end{align*}
also das Newton-Verfahren.
\index{Newton-Verfahren}%

Der Vorteil des Newton-Verfahrens gegenüber dem Sekantenverfahren ist
jedoch, dass die Ableitung nicht nur mit Hilfe eines Differenzenquotienten
approximiert wird, sondern exakt zur Verfügung steht.
Damit ist das Newton-Verfahren nicht anfällig auf die Auslöschung, die
die Zuverlässigkeit des Sekantenverfahrens beeinträchtigt.

%
% Nullstellen on Polynomen
%
\subsection{Nullstellen von Polynomen
\label{buch:subsection:polynomnullstellen}}
\index{Polynom}%
\index{Polynom!Nullstellen}%
Das Newton-Verfahren verlang, dass die Ableitung $f'(x_n)$ genau
berechnet werden kann.
In einigen Fällen kann dies ein Hindernis für die Anwendung des
Verfahrens sein.
Polynome sind jedoch einfach genug, dass die Ableitung immer berechnet
werden kann.
\index{Ableitung}%
Somit ist das Newton-Verfahren besonders gut geeignet, Nullstellen von
Polynomen zu finden.
In diesem Abschnitt sei daher
\begin{equation}
f(X) = a_nX^n + a_{n-1}X^{n-1} + \dots + a_2X^2 + a_1X + a_0
\label{buch:equation:nullstellenpolynom}
\end{equation}
ein Polynom mit reellen Koeffizienten, $a_k\in\mathbb R$.
Wir gehen davon aus, dass $f$ eine reelle Nullstelle $x^*$ hat
und dass $x_0$ eine ausreichend genaue Schätzung für die Nullstelle ist.

\subsubsection{Berechnung von Funktionswerten}
Die übliche Darstellung~\eqref{buch:equation:nullstellenpolynom}
ist nicht die effizienteste Form zur Berechnung des Polynomwertes.
Die Berechnung der Potenzen $x^k$ für $1\le k\le n$ benötigt bereits
$n-1$ Multiplikationen, dazu kommen $n-1$ Multiplikationen mit
Koeffizienten und $n$ Additionen.
Zudem besteht die Gefahr von Verschmierung.

Durch Ausklammern möglichst vieler Faktoren $x$ findet man die
Formel
\index{Ausklammern}%
\begin{align}
f(x)
&=
((\dots((a_nx+a_{n-1})x+a_{n-2})x+\dots)x+a_1)x+a_0,
\label{buch:equation:polynomwert}
\end{align}
welche den Funktionswert in genau $n$ Multiplikationen und $n$ Additionen
zu berechnen gestattet.

Wir bezeichnen die Teilprodukte in \eqref{buch:equation:polynomwert} mit
\index{Teilprodukte}%
\[
(\dots((a_nx+a_{n-1})x+a_{n-2})x+\dots)x+a_k
=
p_{n-k},
\]
d.~h.
\begin{equation}
\begin{aligned}
p_0 &= a_n
\\
p_1 &= a_nx+a_{n-1} = p_0x + a_{n-1}
\\
p_2 &= (a_nx+a_{n-1})x+a_{n-2} = p_1x+a_{n-2}
\\
\vdots\;&\qquad\vdots
\\
f(x)
=
p_n
&=
p_{n-1}x+a_0.
\end{aligned}
\label{buch:equation:reste}
\end{equation}
Diese Berechnung lässt sich in der folgenden, {\em Horner-Schema}
genannten Tabelle
\index{Horner-Schema}%
zusammenfassen.
\begin{center}
\includegraphics{chapters/20-gleichungen/figures/horner1.pdf}
\end{center}

\begin{beispiel}
Wir berechnen den Wert des Polyoms
\[
f(X) = X^6 - X^5 + X^4 - X^3 + X^2 - X + 1
\]
an der Stelle $X=2$ mit Hilfe des Horner-Schemas
\begin{center}
\begin{tabular}{>{$}r<{$}>{$}r<{$}>{$}r<{$}>{$}r<{$}>{$}r<{$}>{$}r<{$}>{$}r<{$}}
 1& -1& 1& -1&  1& -1&  1\\
  &  2& 2&  6& 10& 22& 42\\
\hline
 1&  1& 3&  5& 11& 21& 43
\end{tabular}
\end{center}
Der Wert in der rechten unteren Ecke stimmt überein mit
\begin{align*}
f(2)
&=
2^6-2^5+2^4-2^3+2^2-2+1
\\
&=
64-32+16-8+4-2+1
\\
&=
32+8+2+1
=
43.
\qedhere
\end{align*}
\end{beispiel}

\subsubsection{Deflation}
Die Bedeutung der Werte $p_0,\dots,p_{n-1}$ lässt sich verstehen, wenn
man den Polynomdivisionsalgorithmus für $f(X) / (X-x)$ ausschreibt.
\index{Polynomdivision}%
Die Rekursionsformeln~\eqref{buch:equation:reste} zeigen, dass die
Teilreste der Division die Koeffizienten $p_k$ haben:
\index{Division}%
\index{Teilreste}%
\begin{equation}
\setcounter{MaxMatrixCols}{20}
\setlength\arraycolsep{1pt}
\renewcommand\arraystretch{1.15}
\begin{matrix}
(a_nX^n&+&a_{n-1}X^{n-1}&+&a_{n-2}X^{n-2}&+&a_{n-3}X^{n-3}&+&\dots)&:&(X&-&x)&=&a_nX^{n-1}+p_1X^{n-2}+p_2X^{n-3}+\dots\\
 a_nX^n&-&a_{n}xX^{n-1} & &              & &              & &      & &  & &  & &                \\
\cline{1-3}
       & &p_1X^{n-1}    &+&a_{n-2}X^{n-2}& &              & &      & &  & &  & &                \\
       & &p_1X^{n-1}    &-&p_1xX^{n-2}   & &              & &      & &  & &  & &                \\
\cline{3-5}
       & &              & &p_2X^{n-2}    &+&a_{n-3}X^{n-3}& &      & &  & &  & &\\
       & &              & &p_2X^{n-2}    &-&a_{n-4}xX^{n-3}&&      & &  & &  &\\
\cline{5-7}
       & &              & &              & &p_3X^{n-3}     &&\dots & &  & &  &\\
       & &              & &              & &\dots          &&\dots & &  & &  &\\
\end{matrix}
\end{equation}
Die Koeffizienten $p_k$ sind daher auch die Koeffizienten des Quotienten
\[
q(X)
=
p_0X^{n-1}+p_1X^{n-2}+p_2X^{n-3}+\dots p_{n-2}X+p_{n-1}.
\]
Es gilt daher
\[
f(X)
=
(X-x) \cdot (p_0X^{n-1}+p_1X^{n-2}+p_2X^{n-3}+\dots p_{n-2}X+p_{n-1})
+
f(x).
\]
Ist $x$ ein Nullstelle, dann ist das Polynom ist $f(X)$ durch $X-x$ teilbar
und $q(X)$ ist der andere Faktor, also $f(X)=(X-x)\cdot q(X)$.
Das Polynom $q(X)$ hat Grad $n-1$, die Suche nach weiteren Nullstellen
wird also vereinfacht. 
Man nennt den Prozess, eine Nullstelle aus dem Polynom $f(X)$
herauszudividieren, {\em Deflation}.
\index{Nullstelle}%
\index{Deflation}%

\begin{beispiel}
Das Polynom
\[
f(x)=x^4-25x^2+144
\]
hat eine Nullstelle $x=3$ und $x=4$ als Nulllstellen.
Man finde zwei weitere Nullstellen.

Die Polynomdivision mit dem Horner-Schema für $x=3$
\begin{center}
\begin{tabular}{>{$}r<{$}>{$}r<{$}>{$}r<{$}>{$}r<{$}>{$}r<{$}}
   1&   0& -25&   0& 144\\
    &   3&   9& -48&-144\\
\hline
   1&   3& -16& -48&   0
\end{tabular}
\end{center}
ergibt 
\[
q_1(x) = f(x)/(x-3)
=
x^3+3x^2-16x-48
.
\]
Weiter bekommt man
\[
a_2(x) = f(x)/((x-3)(x-4)) = (x^2+7x+12) = (x+3)(x+4)
\]
aus
\begin{center}
\begin{tabular}{>{$}r<{$}>{$}r<{$}>{$}r<{$}>{$}r<{$}}
   1&   3& -16& -48\\
    &   4&  28&  48\\
\hline
   1&   7&  12&   0
\end{tabular}
\end{center}
Insbesondere schliesst man, dass $x=-3$ und $x=-4$ die verbleibenden
Nullstellen von $f(x)=(x-3)(x-4)(x+4)(x+3)$ sind.
\end{beispiel}


\subsubsection{Berechnung der Ableitung}
\index{Ableitung eines Polynoms}%
Für das Newton-Verfahren wird ausser dem Funktionswert auch die Ableitung
benötigt.
Der Funktionswert $r=f(x)$ wird mit dem Horner-Schema sofort gefunden, ebenso
\index{Horner-Schema}%
der Quotient $q(x)$,
Es gilt also
\[
f(X) = q(X)(X-x) + r,
\]
was wir dazu verwenden können, die Ableitung von $f$ mit Hilfe der
Produktregel zu berechnen:
\index{Produktregel}%
\[
f'(X) = q'(X) (X-x) + q(X).
\]
An der Stelle $X=x$ ist daher
$ f'(x) = q(x) $.
Da die Koeffizienten von $q(X)$ bereits mit dem Horner-Schema
berechnet worden sind, kann $f'(x)$ durch Iteration des Horner-Schemas
berechnet werden.

\begin{beispiel}
Man berechne den Funktionswert und die Ableitung des Polynoms
\[
f(x) = 2x^3 + x + 9
\]
an der Stelle $x=4$.
Zweimalige Anwendung des Horner-Schemas ergibt
\begin{center}
\begin{tabular}{>{$}r<{$}>{$}r<{$}>{$}r<{$}>{$}r<{$}}
   2&   0&   1&   9\\
    &   8&  32& 132\\
\hline
   2&   8&  33& 141\\
    &   8&  64&    \\
\hline
   2&  16&  97&
\end{tabular}
\end{center}
Man liest $f(4)=141$ und $f'(4)=97$ ab.
\end{beispiel}

Ist $x$ eine doppelte Nullstelle des Polynoms $f(x)$, dann ist $f'(x)=0$.
\index{Nullstelle!doppelte}%
\index{doppelte Nullstelle}%
Das Horner-Schema kann daher auch dazu verwendet werden, doppelte
Nullstellen zu erkennen und damit die Faktorisierung zu vereinfachen,
wie das folgende Beispiel zeigt.
\index{Faktorisierung}%

\begin{beispiel}
Wir betrachten das Polynom
\[
f(x) = x^4-13x^3 +41x^2 - 47x+18.
\]
Es hat die Nullstelle $x=1$, das Horner-Schema liefert für den Quotienten
\begin{center}
\begin{tabular}{>{$}r<{$}>{$}r<{$}>{$}r<{$}>{$}r<{$}>{$}r<{$}}
 1&-13& 41&-47& 18\\
  &  1&-12& 29&-18\\
\hline
 1&-12& 29&-18&  0\\
  &  1&-11& 18&   \\
\hline
 1&-11& 18&  0&   \\
  &  1&-10&   &   \\
\hline
 1&-10&  8&   &   
\end{tabular}
\end{center}
Daraus kann man ablesen, dass $x=1$ eine doppelte aber nicht
eine dreifache Nullstelle ist und dass sich das Polynom schreiben lässt als
\[
f(x)=(x-1)^2\cdot (x-11x+18) = (x-1)^2 (x-2)(x-9).
\]
Damit ist das Polynom $f(x)$ vollständig faktorisiert.
\end{beispiel}

\subsubsection{Newton-Verfahren für Nullstellen von Polynomen}
Da mit dem Horner-Schema sowohl Funktionswerte wie auch Ableitungen
effizient berechnet werden können, kann es dazu verwendet werden,
das Newton-Verfahren für Polynomnullstellen zu implementieren.

Das Horner-Schemas liefert zu jeder Nullstelle $x$ auch immer
gleich den Quotienten $q(X)=f(X)/(X-x)$, welches für die Suche nach
weiteren Nullstellen verwendet werden kann (Deflation).
\index{Deflation}%

\begin{beispiel}
\begin{table}
\centering
\renewcommand\arraystretch{1.15}
\begin{tabular}{|>{$}r<{$}|>{$}r<{$}|>{$}r<{$}|>{$}r<{$}|>{$}l<{$}|}
\hline
n& x_n & f(x_n) & f'(x_n) & q_n(X) \\
\hline
0&-10.000000& -182.00000&129.00000& X^2-X+19\\
1&- \underline{8}.589148&  -38.99232& 75.71571& X^2+0.4108524323 X+5.47112751\\
2& -\underline{8.0}74164&   -4.31028& 59.24143& X^2+0.9258356094 X+1.524651051\\
3& -\underline{8.00}1407&   -0.08021& 57.04221& X^2+0.9985933304 X+1.009848595\\
4& -\underline{8.000000}&   -0.00005& 57.00003& X^2+0.9999990463 X+1.000006676\\
5& -\underline{8.000000}&    0.00000& 57.00000& X^2+X+1\\
\hline
\end{tabular}
\caption{Newton-Verfahren für das Polynom $f(X)=X^3+9X^2+9X+8$
mit Hilfe des Horner-Schemas.
Als Nebeneffekt bestimmt das Horner-Schema in jeder Iteration auch
den Quotienten $q_n(x)=f(x)/(x-x_n)$.
\label{buch:table:hornernewton}}
\end{table}
Die reellen Nullstellen
von $f(X)=X^3+9X^2+9X +8$ sollen mit Hilfe des
Newton-Verfahrens gefunden werden.
Die Tabelle~\ref{buch:table:hornernewton} zeigt die vom Horner-Schema
berechneten Funktions- und Ableitungswerte sowie das Quotientenpolynom.
\index{Quotient!Polynom}%
Die Konvergenz ist quadratisch und liefert die Nullstelle $x=-8$
sowie den Quotienten $q(X)=X^2+X+1$,
tatsächlich ist
\[
(X+8)q(X) = (X+8)(X^2+X+1) = X^3+9X^2+9X+8 = f(X).
\]
Die Diskriminante von $q(X)$ ist
$b^2-4ac= 1^2 -4\cdot1\cdot 1= - 3<0$, $q(X)$ hat also keine weiteren
reellen Nullstellen.
\end{beispiel}

%
% normalverteilung.tex
%
% (c) 2020 Prof Dr Andreas Müller, Hochschule Rapeprswil
%
\subsection{Inverse der Normalverteilungsfunktion
\label{buch:subsection:inversenormal}}
Das Integral der Standardnormalverteilungsdichte
\index{Standardnormalverteilungsdichte}%
\index{Normalverteilung}%
\[
\Phi(x) = \int_{-\infty}^x e^{-t^2/2}\,dt
\]
kann nicht in geschlossener Form berechnet werden und erst recht
nicht invertiert werden.
Für die Anwendung wird jedoch die Umkehrfunktion benötigt, zu einem Wert
$p\in[0,1]$ ist dasjenige $x$ zu finden, für welches $F(x)=p$ gilt.
\index{Umkehrfunktion}%
Im Beispiel auf Seite~\pageref{buch:beispiel:erfc} wurde gezeigt,
wie die Fehlerfunktion
\index{Fehlerfunktion}%
\index{erfx@$\operatorname{erf}(x)$}%
\[
\operatorname{erf}(x) = \frac{2}{\sqrt{\pi}}\int_0^x e^{-t^2}\,dt
\]
dazu verwendet werden kann, die Normalverteilungsfunktion
\index{Verteilungsfunktion}%
\index{Normalverteilungsfunktion}%
\[
\Phi(x)
=
\frac12\biggl(1+\operatorname{erf}\biggl(\frac{x}{\sqrt{2}}\biggr)\biggr)
\]
zu berechnen.
In diesem Abschnitt soll untersucht werden, wie zu gegebenen Funktionswert
$p$ das zugehörige $x$ bestimmt werden kann.
Es soll also die Gleichung
\[
\Phi(x)=p
\qquad\Rightarrow\qquad
f(x)=\frac12+\frac12\operatorname{erf}\biggl(\frac{x}{\sqrt{2}}\biggr)-p=0
\]
gelöst werden.

\subsubsection{Sekantenverfahren}
\index{Sekantenverfahren}%
Die mit dem Sekantenverfahren gewonnen Iterationsfolge ist in der rechten
Spalte in Tabelle~\ref{buch:table:normalnewton} dargestellt.
Da die erste Ableitung von $f$ relativ langsam ändert, ist die Konvergenz
zwar zunächst offenbar nur linear, beschleunigt sich dann aber auf fast
quadratische Konvergenz, weil die Sekante die Tangente sehr gut approximiert,
sich das Sekantenverfahren in ihrem Konvergenzverhalten also dem
Newton-Verfahren anzunähern beginnt.
\index{Sekante}%
\index{Tangente}%
Wegen der unvermeidbaren Auslöschung bei der Berechnung der Sekantensteigung
wird das Verfahren dann aber instabil, die Iteration bricht ab.
\index{Auslöschung}%
Es können für den Typ \texttt{long double} nur etwa 18 signifikante
Stellen gefunden werden, während das Newton-Verfahren noch drei weitere
Stellen ermitteln kann.
\index{Newton-Verfahren}%
\index{long double@\texttt{long double}}%

\subsubsection{Newton-Verfahren}
Das Newton-Verfahren benötigt ausser dem Funktionswert auch noch die 
Ableitung
\index{Ableitung}%
\[
f'(x)
=
\frac{d}{dx}\frac{1}{\sqrt{\pi}}\int_0^{x/\sqrt{2}} e^{-t^2}\,dt
=
\frac{1}{\sqrt{2\pi}}e^{-x^2/2}.
\]
Damit wird die Iterationsformel für das Newton-Verfahren:
\index{Iterationsfolge}%
\begin{equation}
x_{n+1}
=
x_n - \sqrt{2\pi} e^{x_n^2/2}
\biggl(\frac12+\operatorname{erf}\biggl(\frac{x_n}{\!\sqrt{2}}\biggr)-p\biggr).
\end{equation}
Wie erwartet konvergiert die Iterationsfolge quadratisch für geeignete
Startwerte (siehe Tabelle~\ref{buch:table:normalnewton}).
Der Startwert $x_0=0$ funktioniert für jedes beliebige $p$.
\index{Startwert}%
Bei weiter von $0$ entfernten Starwerten läuft man Gefahr, dass die Iteration
zu betragsmässig grossen Werten $x$ springt, was dann zu einem Überlauf führt.

\begin{table}
\centering
\begin{tabular}{|>{$}r<{$}|>{$}r<{$}|>{$}r<{$}|}
\hline
 k &\textrm{$x_n$ nach Newton-Verfahren}&\textrm{$x_n$ nach Sekantenverfahren}\\
\hline
 0 &             0.00000000000000000000 &             0.00000000000000000000 \\
 1 & \underline{1}.12798272358394999736 & \underline{1}.00000000000000000000 \\
 2 & \underline{1.}50523868934241237280 & \underline{1}.31831529614238960517 \\
 3 & \underline{1.6}3077255164383121513 & \underline{1.}53246084663801306026 \\
 4 & \underline{1.644}69272791792957300 & \underline{1.6}2023672225157423321 \\
 5 & \underline{1.6448536}0566334308020 & \underline{1.64}272141855681471658 \\
 6 & \underline{1.644853626951472}02343 & \underline{1.6448}1100638619422225 \\
 7 & \underline{1.64485362695147239662} & \underline{1.644853}55229014184382 \\
 8 & \underline{1.64485362695147239662} & \underline{1.6448536269}4885539842 \\
 9 &                                    & \underline{1.64485362695147239}597 \\
\hline
\end{tabular}
\caption{Newton-Iteration und Sekantenverfahren zur Bestimmung der Inversen
der Verteilungsfunktion $\Phi(x)$ der Normalverteilung, berechnet mit dem Typ
\texttt{long double}.
Das Sekantenverfahren ist auch sehr schnell, da die Sekante bereits sehr gut
mit der vom Newton-Verfahren verwendeten Tangente übereinstimmt.
Wegen Auslöschung kann es allerdings nicht die volle Genauigkeit erreichen.
\label{buch:table:normalnewton}}
\end{table}








