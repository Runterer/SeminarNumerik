%
% homotopie.tex
%
% (c) 2020 Prof Dr Andreas Müller, Hochschule Rapperswil
%
\section{Homotopie-Verfahren
\label{buch:section:homotopie}}
\rhead{Homotopie-Verfahren}
Der Erfolg des Newton-Verfahrens hängt entscheidend von der Qualität der
Anfangsschätzung $x_0$ ab.
\index{Newton-Verfahren}%
\index{Anfangsschätzung}%
Allerdings ist es oft nicht einfach, eine solche Schätzung zu produzieren.
Die folgende Idee kann dabei helfen.

Oft ist ein schwieriges Problem ein ``deformierte'' Variante eines
weniger schwierigen Problems.
\index{Deformation}%
\index{deformiert}%
Der Begriff der {\em Homotopie} gibt dieser Idee eine klare Bedeutung.
\index{Homotopie}

\begin{definition}
Zwei Funktionen $f_0(x)$ und $f_1(x)$ heissen homotop, wenn es
eine stetige Funktion
\[
F\colon \mathbb R\times I : (x,t)\mapsto F(x,t)
\]
mit $I=[0,1]$
gibt derart, dass $f_0(x)=F(x,0)$ und $f_1(x)=F(x,1)$.
Die partielle Funktion $x\mapsto F(x,t)$ für $t\in I$ wird auch mit
$f_t$ bezeichnet: $f_t(x)=F(x,t)$.
\end{definition}

Die Funktionen $f_t(x) = F(x,t)$ sind Funktionen ``zwischen'' 
$f_0(x)$ und $f_1(x)$.
Lässt man den Parameter $t$ von $0$ nach $1$ laufen, wird der Graph
von $f_0(x)$ deformiert in den Graphen von $f_1(x)$.

\begin{beispiel}
Die Kepler-Gleichung ist 
\index{Kepler-Gleichung}%
\[
M=E-e\sin E,
\]
wobei $M$ gegeben und $E$ gesucht ist.
Dazu gehört die Funktion
\[
f(E)=M-E+e\sin E.
\]
Der Fall $e=0$ ist ein trivial einfaches Problem, $E=M$ ist Nullstelle
der Funktion
\[
f_0(E)=M-E.
\]
Eine Homotopie zwischen $f_0$ und $f_1=f$ ist
\[
F(x,t) = M-E+et\sin E.
\qedhere
\]
\end{beispiel}

Eine Homotopie kann dazu verwendet werden, Startwerte für das Newton-Verfahren
zu liefern.
Ist $x_0(t)$ eine Nullstelle der partiellen Funktion $x\mapsto F(x,t)$,
dann kann $x_0(t)$ als Startwert zur Bestimmung einer Nullstelle von
der partiellen Funktion $F(x,t')$ für $|t-t'|<\varepsilon$ dienen.
Ist $F$ differenzierbar bezüglich $x$, dann können einige Iterationen
des Newton-Verfahrens aus dem Startwert $x_0(t)$ eine gute Lösung für
$x_0(t')$ sein.
Damit lässt sich der folgende Algorithmus konstruieren:

\begin{enumerate}
\item 
Starte mit der exakten Lösung $x_0=x_0(0)$ und $t=0$
\item
Inkrementiere $t$ um $\Delta t$
\item
verbessere $x_0$ durch einige Iterationen des Newton-Verfahrens
zu einer Nullstelle von $f_t(x)$.
\index{Newton-Verfahren}%
\item 
Wiederhole Schritte 2 und 3 bis $t=1$.
\end{enumerate}
\index{Homotopie-Verfahren}%
Auf diese Weise kann sichergestellt werden, dass jede Iteration
des Newton-Verfahrens mit einem guten Schätzwert startet, wenn auch
nur für eine immer bessere Approximation des eigentlichen Problems.
\index{Schätzwert}%
\index{Approximation}%







