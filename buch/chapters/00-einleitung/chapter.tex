%
% einleitung.tex
%
% (c) 2020 Prof Dr Andreas Müller
%
\chapter*{Einleitung\label{chapter:einleitung}}
\lhead{Einleitung}
\rhead{}
\addcontentsline{toc}{chapter}{Einleitung}
Die Mathematik stellt Werkzeuge zur Verfügung, mit denen natürliche
Prozesse exakt beschrieben werden können.
Die Ingenieurwissenschaften haben über die Jahrhunderte deren
Anwendungbarkeit immer wieder mit unglaublichen Leistungen
demonstriert.
Riesige Bauwerke, Ozeanriesen, Flugzeuge, das Internet mit seiner
unüberschaubaren Transportkapazität für Datenströme, die Landung auf
dem Mond und auf anderen Planeten, die Messung winziger Elementarteilchen,
die nur wenige Zeptosekunden überleben im Large Hadron Collider, die 
Detektion von Gravitationswellen, die die Raumzeit um nur einen Tausendstel
des Durchmessers eines Protons verzerren, zeugen von der unvorstellbaren
Präzision, mit der mathematische Modelle die Realität modellieren können.

Die Realität spielt aber trotzdem oft nicht mit.
Messungen sind in der Realität nur mit eingeschränkter Genauigkeit
möglich.
Wie verändert sich die Qualität einer mathematischen Vorhersage,
wenn die Eingabedaten nicht perfekt sind?
Auch stehen nicht für jede mathematische Problemlösung analytische
Formeln zur Verfügung.
Es bleibt dann nichts anderes übrig, als die Berechnungen auf einem
Computer durchzuführen, der unvermeidliche Rundungsfehler einführt,
die die Resultate ebenfalls beeinträchtigen können.
\index{Rundungsfehler}%
Wie kann man auch unter diesen erschwerten Bedingungen die Präzision
der Ergebnisse sicherstellen, die die genannten technischen Meisterleistungen
erst möglich gemacht haben?

Die numerische Mathematik befasst sich mit der Konstruktion und Analyse
von Algorithmen für kontinuierliche mathematische Probleme.
\index{Algorithmus}%
Die von einem Computer zur Verfügung gestellten Zahlensysteme sind nur
Approximationen für die reellen Zahlen.
\index{Zahlensystem}%
\index{Approximation}%
Sie besser zu verstehen und die Auswirkungen ihre Unzulänglichkeiten 
besser in den Griff zu bekommen ist der Inhalt von
Kapitel~\ref{chapter:berechnung}.
Das wichtigste Werkzeug, die Iteration, wird darin einer gründlichen
Analyse unterzogen und es werden Bedingungen ermittelt, unter denen
der Erfolg garantiert werden kann.
\index{Iteration}%
Es lassen sich Gesetzmässigkeiten für die Fehler finden, die einerseits
instabiles Verhalten vorhersagen können, aber andererseits auch dazu
dienen können, den Fehler zu reduzieren und die Konvergenz eines
Verfahrens zu beschleunigen.
\index{instabil}%

Kapitel~\ref{chapter:gleichungen} kümmert sich um eine der ältesten
Aufgaben der Numerik, nämlich die Lösung von Gleichungen.
\index{Lösung!einer Gleichung}%
\index{Gleichung}%
Da es sich dabei um eine sehr grundlegende Aufgabenstellung handelt,
die immer wieder als Baustein umfassender Problemlösungen auftritt,
ist hohe Genauigkeit und Geschwindigkeit besonders wichtig.
Es wird gezeigt, wie der Newton-Algorithmus sich durch sogenannte
{\em quadratische Konvergenz} auszeichnet.
\index{quadratische!Konvergenz}%

Die Mathematik stellt eine grosse Zahl von nützlichen Funktionen zur
Verfügung, deren exakte Berechnung jedoch oft umständlich und
zeitaufwendig ist.
\index{Funktion}%
Kapitel~\ref{chapter:interpolation} zeigt, wie solche Funktionen
mit hoher Genauigkeit durch Polynome approximiert werden können,
die meist viel einfacher zu berechnen sind.
\index{Polynom}%
Besonderes Augenmerk wird dabei darauf gelegt, die Approximationsfehler
abzuschätzen, so dass ein Interpolationsverfahren immer den
Genaugikeitsanforderungen der Anwendung angepasst werden kann.
\index{Approximationsfehler}%
Spline-Interpolation und Bézier-Kurven runden als in der Praxis besonders
erfolgreiches Approximationsverfahren das Kapitel ab.
\index{Spline-Interpolation}%
\index{Bézier-Kurve}%

In der Analysis lernt man unter anderem, Integrale in geschlossener
Form auszuwerten.
\index{Integral}%
Dabei entsteht oft der falsche Eindruck, mit genügend Ausdauer sei praktisch
jedes Integral lösbar.
Nichts könnte weiter von der Realität entfernt sein.
Tatsächlich ist die Berechnung von Integralen eine weitere Grundaufgabe,
für die die Numerik effiziente Methoden bereitstellen soll.
Kapitel~\ref{chapter:integration} diskutiert die Mittelpunktsformel und
die Trapezregel und zeigt, wie sich die Genauigkeit dieser Verfahren
sehr erfolgreich dank einer sorgfältigen Fehleranalyse steigern lässt.
\index{Mittelpunktformel}%
\index{Trapezregel}%

Zu den wichtigsten mathematischen Modellen für natürliche Prozesse
gehören gewöhnliche Differentialgleichungen.
\index{Differentialgleichung!gewöhnliche}%
\index{gewöhnliche Differentialgleichung}%
Hier ist die Verfügbarkeit analytischer Lösungen noch prekärer als
bei den Integralen.
\index{analytische Lösung}%
\index{Lösung!analytische}%
Glücklicherweise stehen Verfahren sehr hoher Präzision in weit
verbreiteten Softwarepaketen für die numerische Analyse zur Verfügung.
\index{Software}%
Die Techniken, mit denen sich diese Präzision erreichen lässt,
werden in Kapitel~\ref{chapter:ode} dargestellt.
Neben dem universell einsetzbaren Runge-Kutte-Verfahren, welches zur
Klasse der Einschrittverfahren gehört, werden auch Mehrschrittverfahren
vorgestellt sowie Techniken.
\index{Runge-Kutta-Verfahren}%
\index{Einschrittverfahren}%
\index{Mehrschrittverfahren}%
Es wird auch gezeigt, wie man die Abhängigkeit der Lösung einer 
Differentialgleichung nach Anfangsbedingungen und Parameter modellieren
kann.
\index{Anfangsbedingung}%
\index{Parameter}%
Die Lösung von Randwertproblemen wird damit effizient möglich.
\index{Randwertproblem}%

Im Kapitel~\ref{chapter:gleichungen} wurde bewusst auf die Behandlung
linearer Gleichungssysteme verzichtet.
\index{Gleichungssystem!linear}%
\index{lineares!Gleichungssystem}%
Zwar kennt man den Gauss-Algorithmus aus den Grundvorlesungen, er benötigt
\index{Gauss-Algorithmus}%
aber für $n$ Unbekannte eine Laufzeit von der Grössenordnung $O(n^3)$,
was für die sehr grosse Zahl von Unbekannten, mit der man es zum Beispiel
bei der numerischen Lösung von partiellen Differentialgleichungen zu tun hat,
zu langsam ist.
\index{partielle Differentialgleichung}
\index{Differentialgleichung!partielle}
Die besonderen Eigenschaften linearer Gleichungssysteme ermöglichen
aber alternative Ansätze zu deren Lösung.
Das Kapitel~\ref{chapter:linsys} beginnt mit einer Diskussion von 
in der Praxis häufig vorkommenden Spezialfällen von Gleichungssystemen,
die sich schneller als in $O(n^3)$ lösen lässt.
Es diskutiert dann eine Reihe von iterativen Verfahren zur Gleichungslösung
\index{iterativ}%
\index{Gleichungssystem!iteratives Verfahren}%
und zeigt, wie der Spektralradius einer Matrix die zentrale
Masszahl für den Erfolg oder Misserfolg solcher Verfahren ist.
\index{Spektralradius}%
Es schliesst mit einer Diskussion zweier einfacher Verfahren für die
QR-Zerlegung und für das Eigenwertproblem, die im zweiten Teil vertieft
wird.
\index{QR-Zerlegung}%
\index{Eigenwertproblem}%

Partielle Differentialgleichungen stellen mathematische Modelle
für ausgedehnte System wie elektromagnetische Felder, Temperaturverteilung,
Dichte etc.~bereit.
\index{Differentialgleichung!partielle}%
\index{partielle Differentialgleichung}%
\index{Feld!elektromagnetisches}%
\index{elektromagnetisches Feld}%
\index{Temperatur}%
\index{Dichte}%
Für deren Lösung sind die Methoden für gewöhnliche Differentialgleichungen
nicht geeignet.
\index{Differentialgleichung!gewöhnliche}%
\index{gewöhnliche Differentialgleichung}%
Kapitel~\ref{chapter:pde} führt die grundlegenden Definitionen aus
der Theorie der partiellen Differentialgleichungen ein und vermittelt
die Grundideen der wichtigsten Diskretisations- und Lösungsverfahren.
\index{Diskretisation}%
Zur Sprache kommen finite Differenzen, finite Volumina und finite Elemente.
\index{finite Differenzen}%
\index{finite Volumina}%
\index{finite Elemente}%








