%
% chapter.tex
%
% (c) 2020 Prof Dr Andreas Müller
%
\chapter{Partielle Differentialgleichungen\label{chapter:pde}}
\lhead{Partielle Differentialgleichungen}
\rhead{}
Elektrische oder magnetische Felder sind Funktionen der Ortskoordinaten
und der Zeit.
\index{Feld}%
\index{elektrisch}%
\index{magnetisch}%
\index{Ladung}%
\index{Maxwellsche Gleichungen}%
Die Veränderung dieser Felder mit der Zeit hängt ab von der Anwesenheit
oder der Bewegung von Ladungen, wie sie von den Maxwellschen Gleichungen
\index{nabla@$\nabla$}%
\begin{align*}
\nabla\cdot\vec{E} &= \frac{\varrho}{\varepsilon_0}
\\
\nabla\cdot\vec{B} &= 0
\\
\nabla\times\vec{E} &= -\frac{\partial\vec{B}}{\partial t}
\\
\nabla\times\vec{B} &= \mu_0\vec{\jmath}
	+ \frac{1}{c^2}\frac{\partial\vec{E}}{\partial t}
\end{align*}
beschrieben werden.
Diese Gleichungen zwischen den Komponenten der Felder enthalten
die partiellen Ableitungen der gesuchten Funktionen nach Ortskoordinaten
und der Zeit.
\index{partielle Differentialgleichung}
\index{Differentialgleichung!partiell}
Solche {\em partiellen Differentialgleichungen} können nicht mit den Methoden
gelöst werden, die für gewöhnliche Differentialgleichungen entwickelt 
worden sind.
In diesem Kapitel sollen die wichtigsten Ideen zusammengetragen werden,
die besser geeigneten Verfahren zu Grunde liegen.

%
% problem.tex
%
% (c) 2020 Prof Dr Andreas Müller, Hochschule Rapperswil
%
\section{Problemstellung
\label{section:pde:problem}}
\rhead{Problemstellung}
\index{Differentialgleichung!gewöhnlich}%
\index{gewöhnliche Differentialgleichung}%
Gewöhnliche Differentialgleichungen werden durch eine einzige Funktion
$f\colon \mathbb R\times\mathbb R^n: (t,x)\mapsto f(t,x)$ beschrieben,
werden.
Eine Lösung der Differentialgleichung
\begin{equation}
\frac{dx}{dt} = f(t,x)
\label{pde:eqn:ode}
\end{equation}
mit der Anfangsbedingung
\[
x(t_0) = x_0
\]
ist eine Funktion $x(t)$ mit $x(t_0)=x_0$ derart, dass 
\[
\frac{dx(t)}{dt} = f(t, x(t)).
\]
Das Definitionsgebiet der Lösungsfunktion $x(t)$ ist ein Intervall
der Form $[t_0,a]$ mit $a>t_0$.
\index{Definitionsgebiet}%

Für eine partielle Differentialgleichung ist die Situation wesentlich
komplizierter.
Zunächst gibt es Ableitungen der gesuchten Funktion $u(x_1,\dots,x_n)$
nach allen unabhängigen Variablen zu
berücksichtigen, so dass eine explizite Form der Differentialgleichung
wie in~\eqref{pde:eqn:ode} grundsätzlich nicht mehr möglich ist.
Das Definitionsgebiet ist eine fast beliebige Teilmenge
$\Omega\subset\mathbb R^n$ eines $n$-dimensionalen Raumes.
Insbesondere kann das Definitionsgebiet sehr viel komplizierter sein
als im Falle einer gewöhnlichen Differentialgleichungen.
Die Form des Gebietes hat einen wesentlichen Einfluss auf die Lösungen
der Differentialgleichung.
Schliesslich wird es nicht mehr genügen, Werte in nur einem Randpunkt
des Gebietes zu kennen, wie das bei einer gewöhnlichen Differentialgleichung
der Fall war.
Vielmehr ist es eine nichttriviale Frage, auf welchem Teil des Randes
$\partial\Omega$ von $\Omega$ welche Funktions- oder Ableitungswerte
vorgegeben werden müssen, damit die Lösung der Differentialgleichung
eindeutig bestimmt ist.

Der Einfachheit halber betrachten wir in diesem Kapitel nur partielle
Differentialgleichungen für eine skalare Funktion $u=u(x_1,\dots,x_n)$.
In diesem Abschnitt geht es darum zu klären, wie genau ein solches
Problem gestellt werden muss.
In Abschnitt~\ref{subsection:pde:gebiet} studieren wir ein sinnvolles
Definitionsgebiet für die Differentialgleichung.
In Abschnitt~\ref{subsection:pde:klassifikation} klassifizieren wir
mögliche Differentialgleichungen bevor wir in
Abschnitt~\ref{subsection:pde:randbedingungen} Randbedingungen und
in Abschnitt~\ref{subsection:pde:loesungen} Lösungen beschreiben.

\subsection{Gebiet und Rand
\label{subsection:pde:gebiet}}
Eine partielle Differentialgleichung sucht eine Funktion
\index{partielle Differentialgleichung}%
\index{Differentialgleichung!partiell}%
\[
u\colon \Omega\to\mathbb R
\]
$u(x_1,\dots,x_n)$, für die Beziehungen zwischen den partiellen
Ableitungen nach den unabhängigen Variablen gelten.
Für alle Punkte einer Menge $\Omega\subset\mathbb R^n$ muss
als eine Gleichung ähnlich wie zum Beispiel
\[
\Delta u
=
\frac{\partial^2 u}{\partial x_1^2}
+
\dots
+
\frac{\partial^2 u}{\partial x_n^2}
=
0
\qquad
\forall \;
(x_1,\dots,x_n)\in \Omega
\]
gelten, $\Delta$ ist der {\em Laplace-Operator}.
\index{Laplace-Operator}%
Diese Beobachtung schränkt die Art der Menge $\Omega$ bereits ein,
wie im Folgenden diskutiert werden soll.

\begin{figure}
\centering
\includegraphics{chapters/70-pde/images/offen.pdf}
\caption{Jeder innere Punkt einer Menge hat eine $\varepsilon$-Umgebung, 
die ebenfalls in der Menge enthalten ist.
Jede Umgebung des Punktes $P$ enthält aber auch Punkte ausserhalb der
Menge $\Omega$, $P$ ist daher ein Randpunkt.
\label{buch:pde:figure:offen}}
\end{figure}

Die Ableitung einer Funktion $u(x_1,\dots,x_n)$ in einem Punkt
$x_0=(x_{0,1},\dots,x_{0,n})$ ist eine lineare Funktion 
$Du(x_{0,1},\dots,x_{0,n}) $ derart, dass
\[
u(x_1,\dots,x_n) - u(x_{0,1},\dots,u_{0,n})
=
Du(x_{0,1},\dots,x_{0,n})\cdot (x-x_0) + o(|x-x_0|).
\]
Das Symbol $o(|x-x_0|)$ beschreibt eine Funktion, die schneller als
ihr Argument gegen $0$ geht, so dass für den Grenzwert des
Quotienten 
\[
\lim_{x\to x_0}\frac{o(|x-x_0|)}{|x-x_0|} = 0
\]
gilt.
Der Grenzwert bedeutet, dass es für jedes $\varepsilon>0$ eine Umgebung
\index{Umgebung}%
$U_\delta(x_0)=\{x\;|\; |x-x_0|<\delta\}$ gibt derart, dass
\index{Udeltax0@$U_\delta(x_0)$}
\[
\bigl|
u(x)-u(x_0) - Du(x_0)\cdot (x-x_0)
\bigr|
< \varepsilon\cdot |x-x_0|
\qquad\forall x\in U_\delta(x_0)
\]
ist.
Dies ist nur sinnvoll, wenn die ganze Umgebung $U_{\delta}(x_0)\subset \Omega$
im Definitionsgebiet $\Omega$ der Differentialgleichung vorhanden ist.
Dies führt auf die folgende Definition (siehe auch Abbildung~\ref{buch:pde:figure:offen}.

\begin{definition}
\index{offene Menge}%
\index{Gebiet}%
\index{Menge!offen}%
Eine Menge $\Omega$ heisst {\em offen}, wenn mit jedem Punkt $x\in \Omega$
auch eine offene Umgebung $U_{\delta}(x)\subset\Omega$ darin enthalten ist.
Ein {\em Gebiet} ist eine offene Menge in $\mathbb R^n$.
\end{definition}

Gebiete sind also genau die sinnvollen Definitionsgebiete für eine partielle
Differentialgleichung.
\index{Definitionsgebiet}%
Es reicht allerdings nicht, dass die Funktion $u$ auf $\Omega$ definiert
ist, da ja auch noch Randwerte erfüllt werden müssen.

\begin{figure}
\centering
\includegraphics{chapters/70-pde/images/gebiet.pdf}
\caption{Inneres und Rand einer Punktmenge $A$ in der Ebene.
Die sechs Strahlen der Menge $A$ links können nichts zu den Randwerten
einer partiellen Differentialgleichunb beitragen, weil sie keinen
Einfluss auf Werte in inneren Punkten (Mitte) haben können.
\index{Strahlen}%
Randwerte müssen daher nur auf einem Teil des Randes $\partial\mathring{A}$
des Inneren (rechts) spezifiziert werden.
\label{buch:pde:figure:gebiet}}
\end{figure}

\begin{definition}
Der {\em Abschluss} $\bar{\Omega}$ einer Menge $\Omega\subset\mathbb R^n$ ist
die Menge aller Punkte in $\mathbb R^n$, die Grenzwerte von Folgen in
$\Omega$ sind.
\index{Abschluss}%
Das {\em Innere} $\mathring{A}$ einer Menge $A$ ist die Menge aller Punkte
$x$ derart, dass es eine Umgebung $U_\delta(x)$ gibt, die ganz in $A$
enthalten ist: $U_{\delta}(x)\subset A$.
\index{Inneres}%
\end{definition}

Alternativ kann man den Abschluss auch charakterisieren als die Menge
aller Punkte $x\in\mathbb R^n$, für die jede beliebige Umgebung
$U_\delta(x)$ auch Punkte von $\Omega$ enthält, also
$U_\delta(x)\cap \Omega\ne \emptyset$.
Ein Gebiet ist offen, daher ist $\mathring\Omega=\Omega$.

Die Lösung einer partiellen Differentialgleichung wird im Allgemeinen erst
festgelegt sein, wenn zusätzlich Werte auf Teilen des ``Randes'' des
Gebietes festgelegt worden sind.
Nur Punkte, die als Grenzwerte von Punkten in $\Omega$ erreicht werden
können, können zu diesem Zweck hinzugezogen werden.

\begin{definition}
Der {\em Rand} $\partial A$ einer Menge $A\subset\mathbb R^n$ ist
$\partial A=\bar{A}\setminus\mathring{A}$.
\index{Rand}
\index{d@$\partial A$}
\end{definition}

Während also eine Differentialgleichung typischerweise auf einem
Gebiet $\Omega$ definiert ist, muss die gesuchte Lösungsfunktion
sogar auf auf dem Abschluss $\bar{\Omega}$ definiert sein.
Die Lösung wird im Allgemeinen erst dadurch festgelegt, dass zusätzlich
Werte oder Ableitungen auf Teilen des Randes $\partial\Omega$ 
vorgegeben werden.

\subsection{Klassifikation der partiellen Differentialgleichungen
\label{subsection:pde:klassifikation}}
Eine partielle Differentialgleichung beschreibt eine Beziehung 
zwischen den partiellen Ableitungen der Funktion $u(x_1,\dots,x_n)$.
Dies ist auf sehr vielfältige Arten möglich, in die dieser Abschnitt
etwas Ordnung bringen soll.

\subsubsection{Ordnung}
Wie bei den gewöhnlichen Differentialgleichungen klassifizieren wir
auch partielle Differentialgleichungen nach der Ordnung.
\index{Ordnung!einer partiellen Differentialgleichung}%
\index{partielle Differentialgleichung!Ordnung}

\begin{definition}
Die {\em Ordnung} einer partiellen Differentialgleichung ist die
Ordnung der höchsten Ableitung, die in der Differentialgleichung
vorkommt.
\index{Ordnung}%
\end{definition}

Für die folgende Diskussion reicht es, partielle Differentialgleichungen
zweiter Ordnung zu betrachten.
Die meisten in den Anwendungen vorkommenden Differentialgleichungen
sind von dieser Art, so dass dies eine unwesentliche Einschränkung ist.
Die Erweiterungen für höhere Ordnung sind offensichtlich.

\subsubsection{Linearität}
Die Beziehung zwischen den Ableitungen kann beschrieben werden durch
eine Funktion
\[
F(x_1,\dots,x_n,u,p_1,\dots,p_n,t_{11},t_{12},\dots,t_{nn})
\]
derart, dass nach der Substitution
\begin{align*}
u&\to u(x_1,\dots,x_n)
\\
p_i&\to \frac{\partial u}{\partial x_i}
\\
t_{ij} &Ò\to \frac{\partial^2 u}{\partial x_i\,\partial x_j}
\end{align*}
die Differentialgleichung
\[
F\biggl(
x_1,\dots,x_n,u(x_1,\dots,x_n),
\frac{\partial u}{\partial x_1},\dots,\frac{\partial u}{\partial x_n},
\frac{\partial^2u}{\partial x_1^2},
\frac{\partial^2u}{\partial x_1\,\partial x_2},\dots,
\frac{\partial^2u}{\partial x_n^2}
\biggr)
=0
\]
entsteht.
Für höhere Ordnung werden weitere Variablen benötigt, die für die
höheren Ableitungen stehen.
Die Funktion $F$ kann also dazu verwendet werden, die verschiedenen
möglichen partiellen Differentialgleichungen zu klassifizieren.

Eine partielle Differentialgleichung heisst {\em linear}, wenn die
Funktion $F$ linear ist in den Argumenten $u$, $p_i$ und $t_{ij}$,
wenn also gilt
\begin{align}
F(\dots,\lambda u' + \mu u'',\dots)
&=
\lambda F(\dots,u',\dots) + \mu F(\dots,u'',\dots)
\label{pde:eqn:linearu}
\\
F(\dots,\lambda p'_i+\mu p''_i,\dots)
&=
\lambda F(\dots,p'_i,\dots) + \mu F(\dots,p''_i,\dots)
&&\forall i
\label{pde:eqn:linearp}
\\
F(\dots,\lambda t'_{ij}+\mu t''_{ij},\dots)
&=
\lambda F(\dots,t'_{ij},\dots) + \mu F(\dots,t''_{ij},\dots)
&&\forall i,j.
\label{pde:eqn:lineart}
\end{align}
Eine partielle Differentialgleichung heisst {\em quasilinear}, 
wenn die Funktion $F$ linear ist in den Argumenten $p_i$ und $t_{ij}$,
wenn also die Bedingungen~\eqref{pde:eqn:linearp} und \eqref{pde:eqn:lineart}
gelten.
\index{quasilinear}%
Linearität in $u$, also die Bedingung~\eqref{pde:eqn:linearu} ist
für eine quasilineare partielle Differentialgleichung nicht verlangt.

\begin{beispiel}
Die Differentialgleichung
\[
\frac{\partial^2 u}{\partial x_1^2}
+
\frac{\partial^2 u}{\partial x_2^2}
=
0
\]
zweiter Ordnung wird beschrieben durch die Funktion
\[
F(x_1,x_2,u,p_1,p_2,t_{11},t_{12},t_{22})
=
t_{11} + t_{22}.
\]
Diese Funktion ist linear in $t_{11}$ und $t_{22}$.
\end{beispiel}

\begin{beispiel}
Die partielle Differentialgleichung erster Ordnung von Burgers
(Siehe auch Kapitel~\ref{chapter:burgers})
\index{Burgers}%
\index{Gleichung!von Burgers}%
\begin{equation}
\frac{\partial u}{\partial x_1}
+
u
\frac{\partial u}{\partial x_2}
=
0
\label{buch:eqn:burgers}
\end{equation}
wird beschrieben durch die Funktion
\[
F(x_1,x_2,u,p_1,p_2)
=
p_1+up_2.
\]
Diese Funktion ist nicht linear in $u$ und $p_2$, aber linear in
$p_1$ und $p_2$.
\eqref{buch:eqn:burgers} ist also eine quasilineare partielle
Differentialgleichung.
\end{beispiel}

\subsubsection{Lineare Differentialgleichungen zweiter Ordnung}
Ein für die Anwendungen besonders wichtiger Fall sind lineare
Differentialgleichungen zweiter Ordnung.
Für eine solche Gleichung ist die Funktion $F$ immer von der Form
\[
F(x_i,u,p_i,t_{ij})
=
\sum_{i,j=1}^n a_{ij}(x_1,\dots,x_n)t_{ij} + \sum_{i=1}^n b_i(x_1,\dots,x_n) p_i + c(x_1,\dots,x_n) u 
\]
Die Koeffizienten $a_{ij}$, $b_i$ und $c$ dürfen also von den Variablen
$x_1,\dots,x_n$ abhängen, nicht aber von $u$ oder den Ableitungen.
Die Differentialgleichung hat also die Form
\begin{equation}
\sum_{i,j=1}^n
a_{ij}\frac{\partial^2 u}{\partial x_i\,\partial x_j}
+
\sum_{i=1}^n
b_i\frac{\partial u}{\partial x_i}
+
cu
=
f
\label{pde:eqn:eqn2nd}
\end{equation}
Da es in den zweiten Ableitungen nicht auf die Reihenfolge ankommt,
kann die Matrix $a_{ij}$ immer symmetrisch gewählt werden.
Die Matrix $A=(a_{ij})$ heisst die {\em Symbolmatrix} der Differentialgleichung.
\index{Symbolmatrix}%

Es stellt sich heraus, dass die Terme zweiter Ordnung, also die 
Matrix $A$, das Verhalten der Lösung wesentlich beeinflussen.
Eine symmetrische Matrix kann durch eine Drehung immer in eine
Diagonalmatrix transformiert werden.
Durch Wechsel des Koordinatensystems kann man also erreichen, dass
\[
A
=
\begin{pmatrix}
\lambda_1&         &      &         \\
         &\lambda_2&      &         \\
         &         &\ddots&         \\
         &         &      &\lambda_n
\end{pmatrix}
\]
ist.
In diesen Koordinaten sind es nur noch die Eigenwerte
$\lambda_1,\dots,\lambda_n$ der Matrix $A$, die das Verhalten der Lösung
bestimmen.

\begin{definition}
Die Differentialgleichung
\eqref{pde:eqn:eqn2nd}
heisst {\em elliptisch}, wenn alle Eigenwerte positiv sind.
Sie heisst {\em hyperbolisch}, wenn alle Eigenwerte bis auf einen negativen
positiv sind.
Sie heisst {\em parabolisch}, wenn alle Eigenwerte bis auf einen
verschwindenden positiv sind.
\end{definition}
\index{elliptische partielle Differentialgleichung}%
\index{elliptisch}%
\index{hyperbolische partielle Differentialgleichung}%
\index{hyperbolisch}%
\index{parabolische partielle Differentialgleichung}%
\index{parabolisch}%

\begin{beispiel}
Die {\em Wellengleichung}
\index{Wellengleichung}%
\[
\frac{1}{a^2}\frac{\partial^2 u}{\partial t^2}
=
\Delta u 
\quad\Rightarrow\quad
\Delta u 
-
\frac{1}{a^2}\frac{\partial^2 u}{\partial t^2}
=
0
\]
hat die Symbolmatrix
\[
A=
\begin{pmatrix}
1& &      &              \\
 &1&      &              \\
 & &\ddots&              \\
 & &      &\displaystyle-\frac{1}{a^2}
\end{pmatrix}
\]
$n$ positive und einem negativen Eigenwert, diese Gleichung ist also
hyperbolisch.
Die Lösungen zeigen Wellencharakter und die Werte und ersten
Ableitungen von $u$ zu einem Zeitpunkt $t_0$ bestimmen das Verhalten
der Lösung vollständig.
Eine hyperbolische Differentialgleichung hat also eine natürlich
``Entwicklungsrichtung'', das Gebiet kann in Richtung ``Zukunft''
offen sein, ohne dass die Eindeutigikeit der Lösung dadurch gefährdet wird.
\end{beispiel}

\begin{beispiel}
Die Gleichung für das Potential einer Ladungsverteilung
\index{Ladungsverteilung}%
\[
\Delta u = f
\qquad\Rightarrow\qquad
A=
\begin{pmatrix}
1& &      & \\
 &1&      & \\
 & &\ddots& \\
 & &      &1
\end{pmatrix}
=E
\]
hat nur positive Eigenwerte, sie ist also elliptisch.
Die Lösungen dieser Gleichung sind nur dann bestimmt, wenn Werte von $u$
auf dem gesamten Rand des Gebietes vorgegeben werden.
Es ist also nicht möglich, eine ``Zeitrichtung'' für die Entwicklung
der Lösung einer solchen Differentialgleichung zu finden.
\index{Zeitrichtung}%
\end{beispiel}

Die Verschiedenartigkeit der Lösungen in Abhängigkeit vom Typ der
Differentialgleichung hat zur Folge, dass verschiedene Lösungsverfahren
zum Einsatz kommen müssen.
Die Klassifikation einer Differentialgleichung ist also Vorbedingung
für die Wahl eines geeigneten Lösungsverfahrens.
\index{Klassifikation!partielle Differentialgleichung}

\subsection{Randbedingungen
\label{subsection:pde:randbedingungen}}
Die Lösung einer partiellen Differentialgleichung ist erst eindeutig
bestimmt, wenn Werte von $u$ oder von Ableitungen von $u$ auf geeignet
gewählten Teilen das Randes $\partial\Omega$ des Gebietes $\Omega$
vorgegeben sind.
Die Theorie der partiellen Differentialgleichungen studiert ausführlich,
welche Art von Randbedingungen wo auf dem Rand zu spezifizieren sind.
\index{Randbedingung}%

\subsubsection{Dirichlet-Randbedingungen}
Werte der Funktion auf dem Rand können vorgegeben werden, indem eine
Funktion $g\colon \partial\Omega\to\mathbb R$ spezifiziert wird, so
dass für alle Punkte $x\in\partial\Omega$ die Gleichung
$
u(x) = g(x) 
$
gilt.
Diese Art von Randbedingungen heisst {\em Dirichlet-Randbedingungen}.
\index{Dirichlet-Randbedingung}%
\index{Randbedingung!Dirichlet}%

\subsubsection{Neumann-Randbedingungen}
Die Theorie des Anfangswertproblems für gewöhnliche Differentialgleichungen
besagt, dass die Lösung einer Differentialgleichung $n$-ter Ordnung erst
festgelegt ist, wenn der Anfangswert und $n-1$-Ableitungen vorgegeben sind.
Es ist zu erwarten, dass es auch bei gewissen partiellen
Differentialgleichungen der Ordnung $n\ge 2$ nötig sein wird, 
Ableitungen auf dem Rand vorzugeben.

Im Allgemeinen wird es nicht genügen, nur Ableitungen vorzugeben,
da sie Funktionen nur bis auf eine Konstante festlegen können.
Es wird also nötig sind, mindestens in einem Punkt zusätzlich einen
Funktionswert vorzugeben.

Es ist jedoch nicht sinnvoll, beliebige Ableitungen auf dem Rand
vorzugeben.
Wir illustrieren dies an einem einfachen Beispiel.
Als Gebiet wählen wir $\Omega = \{(x,y)\in\mathbb R^2\;|\; x > 0\}$,
der Rand ist also die $y$-Achse.
Nehmen wir an, dass Werte der Ableitung $\partial u/\partial y$ 
auf der $y$-Achse vorgegeben sind, also
\[
\frac{\partial u}{\partial y}(0,y)  = g(y).
\]
Ausserdem nehmen wir an, dass der Funktionswert $u(0,0)=u_0$ vorgegeben ist.
Für die Funktion $f(y) = u(0,y)$ gilt daher $f'(y) = g(y)$ und $f(0)=u_0$.
Daraus lässt sich die Funktion $f$ aber durch das Integral
\[
f(y) = u_0 + \int_0^y g(\eta)\,d\eta
\]
bestimmen.
Die Vorgabe der Ableitung $\partial u/\partial y$ auf dem Rand und eines
Wertes ist also gleichbedeutend mit der Vorgabe aller Werte $f(y)$ auf dem
Rand.
Statt der Ableitungen $\partial u/\partial y$ hätten wird daher auch
Dirichlet-Randbedingungen $u(0,y) = f(y)$ vorgeben können.
Es ist also nur sinnvoll, die Ableitung $\partial u/\partial x$ 
auf dem Rand vorzugeben.

Weiter oben hat ein spezielles Beispiel bezeigt,
dass Ableitungen entlang des Randes gegenüber Dirichlet-Randbedingungen
keine neue Information liefern können.
Nur die Ableitung in die Richtung senkrecht auf den Rand kann zusätzliche
Information liefern.
Dazu muss der Rand des Gebietes ausreichend glatt sein, so dass die
Normale auf den Rand wohldefiniert ist.

\begin{definition}
Ist $u$ eine Funktion, die auf $\bar\Omega$ definiert ist.
Sei $n$ die Normale auf den Rand $\partial\Omega$ in einem Punkt
$x\in\partial\Omega$.
Die {\em Normalableitung}
\[
\frac{\partial u}{\partial n}
=
\lim_{t\to 0+} \frac{u(x+tn)-u(x)}{t}
\]
in $x$ ist die Richtungsableitung 
der Funktion $u$ in Richtung $n$.
\end{definition}
\index{Normalableitung}%

Im Falle der Wellengleichung
\[
\frac{\partial^2 u}{\partial x^2}
-
\frac{1}{a^2}
\frac{\partial^2 u}{\partial t^2}
=0
\]
beschreibt $u(t,x)$ zum Beispiel die Auslenkung einer gespannten Saite
aus der Ruhelage.
Es ist aus physikalischen Überlegungen klar, dass die Bewegung der Saite
erst dann festgelegt ist, wenn Auslenkung und Geschwindigkeit der Saite
zur Zeit $t=0$ vorgegeben werden.
Die Vorgabe der Auslenkung zur Zeit $t=0$
\[
u(0,x) = f(x)
\]
ist eine Dirichlet-Randbedingung.
Die Richtung $n$ der Zeitachse ist senkrecht auf dem Rand $t=0$,
die Zeitableitung von $u$ ist also genau eine Normalableitung.
Die Vorgabe der Geschwindigkeit
\[
\frac{\partial u}{\partial n}(0,x)
=
\frac{\partial u}{\partial t}(0,x)
=
g(x)
\]
ist also eine Vorgabe der Normalableitung.

\begin{definition}
Die Vorgabe der Normalableitung auf dem Rand $\partial\Omega$
\[
\frac{\partial u}{\partial n}(x) = h(x)\qquad \forall x\in\partial\Omega
\]
heisst eine {\em Neumann-Randbedingung}.
\end{definition}
\index{Neumann-Randbedingung}%
\index{Randbedingung!Neumann}%

\subsection{Lösungen
\label{subsection:pde:loesungen}}
Ein vollständig gestelltes Problem mit partiellen Differentialgleichungen
beginnt also immer mit einer Definition des Gebietes $\Omega$, also
einer offenen Menge in $\mathbb R^n$.
Die Differentialgleichung wird gegeben durch eine Funktion $F$ wie in
Abschnitt~\ref{subsection:pde:klassifikation} dargestellt.
Ausserdem müssen Randbedingungen vorgegeben werden, wie in
Abschnitt~\ref{subsection:pde:randbedingungen} dargestellt.
Gesucht ist dann eine Funktion $u$, welche alle diese Bedingungen
erfüllen soll.
Dazu muss die Funktion nicht nur in $\Omega$ definiert sein, 
sondern auch auf dem Rand.

\begin{definition}
Eine Lösung einer partiellen Differentialgleichung ist eine Funktion
\[
u\colon \bar{\Omega} \to \mathbb R:(x_1,\dots,x_n)\mapsto u(x_1,\dots,x_n)
\]
derart, dass die Differentialgleichung im Inneren, also in $\Omega$,
erfüllt ist, und die Randbedingungen auf $\partial\Omega$.
\end{definition}







%
% fdm.tex
%
% (c) 2020 Prof Dr Andreas Müller, Hochschule Rapperswil
%
\section{Finite Differenzen
\label{section:finite-differenzen}}
\rhead{Finite Differenzen}
Jedes Lösungsverfahren für partielle Differentialgleichungen muss die
unendlich vielen Freiheitsgrade, die eine Funktion
$u\colon\Omega\to\mathbb R$ enthalten kann, auf eine endlich Zahl von
Variablen reduzieren, für die sich ein Gleichungssystem aufstellen 
lässt, welches dann mit einer der früher studierten Methoden
gelöst werden kann.

%
% Gitter und Ableitungen
%
\subsection{Gitter und Ableitungen
\label{pde:subsection:gitter}}
Eine einfache Methode, die Funktion $u$ eine endliche Menge von Parametern
zu reduzieren, ist, nur die Werte in einzelnen Punkten des Gebietes 
$\Omega$ zu verwenden.
Als Beispiel betrachten wir ein Gebiet $\Omega\subset\mathbb R^2$
in der Ebene.
Dazu betrachten wir die Punkte
\[
x_{ik} = (ih_x, kh_y) \in \mathbb R^2
\qquad
i,k\in\mathbb Z,
\]
sie bilden ein Gitter $\Gamma$ mit Gitterkonstante oder Schrittweite
$h_x$ in $x$-Richtung und $h_y$ in $y$-Richtung.
\index{Gitter}%
\index{Gitterkonstante}%
\index{Schrittweite}%

Die Funktion $u(x,y)$ nimmt in den Punkten $x_{ik}$ die Werte
$u_{ik} = u(x_{ik})$ an.
Eine approximative Lösung der Differentialgleichung ist also
die Bestimmung der Werte $u_{ik}$ für die Punkte $x_{ik}$, die in
$\Omega$ liegen, für die also $x_{ik}\in\Omega$ gilt.
Dazu müssen jetzt die Randbedingungen und die Differentialgleichung
in Gleichungen für die Unbekannten $u_{ik}$ übersetzt werden.

\subsubsection{Dirichlet-Randbedingungen}
Dirichlet-Randbedingungen geben die Werte auf dem Rand vor.
\index{Dirichlet-Randbedingung}%
Liegt ein Punkt $x_{ik}$ des Gitters $\Gamma$ auf dem Rand $\partial\Omega$
von $\Omega$, dann geben die Dirichlet-Randbedingungen den Wert
dieser Variablen vor.
Es ist daher anzustreben, das Gitter $\Gamma$ so zu wählen, dass 
der Rand durch Gitterpunkte verläuft.
Ein gekrümmter Rand wird daher im Allgmeinen durch eine Kurve durch
die Gitterpunkte approximiert werden müssen.

\subsubsection{Erste Ableitungen}
\begin{figure}
\centering
\includegraphics{chapters/70-pde/images/derivatives.pdf}
\caption{Approximation der ersten Ableitung mit verschiedenen
Differenzausdrücken
\label{buch:pde:1abldiff}}
\end{figure}
Die naheliegenste Approximation für die Differentialgleichung besteht
darin, die Ableitungen durch Differenzenquotienten zu ersetzen (siehe 
Abbildung~\ref{buch:pde:1abldiff}).
Mit der oben eingeführten Notation können die ersten Ableitungen durch
die sogenannten {\em Vorwärtsdifferenzen}
\index{Vorwärtsdifferenz}%
\begin{align}
\frac{\partial u}{\partial x} (x_{ik}) 
&\approx
\frac{u(x_{i+1,k}) - u(x_{ik})}{h_x}
&
&\text{und}
&
\frac{\partial u}{\partial y} (x_{ik}) 
&\approx
\frac{u(x_{i,k+1}) - u(x_{ik})}{h_y}
\label{chapter:pde:approx1st}
\end{align}
approximiert werden (Abbildung~\ref{buch:pde:1abldiff} Mitte).
Die Genauigkeit der Approximation kann offenbar verbessert werden, 
indem $h_x$ und $h_y$ verkleinert werden.

Der Fehler der Approximation~\eqref{chapter:pde:approx1st} ergibt sich
aus dem Mittelwertsatz der Differentialrechnung.
\index{Mittelwertsatz}%
Es gibt Zahlen $\xi$ und $\eta$ zwischen $ih_x$ und $(i+1)h_x$
bzw.~$kh_y$ und $(k+1)h_y$ derart, dass
\begin{align*}
\frac{u(x_{i+1,k}) - u(x_{ik})}{h_x}
&=
\frac{\partial u}{\partial x}(\xi, kh_y)
&
&\text{und}
&
\frac{u(x_{i,k+1}) - u(x_{ik})}{h_y}
&=
\frac{\partial u}{\partial x}(ih_x, \eta).
\end{align*}
Dies zeigt, dass die Approximation~\eqref{chapter:pde:approx1st}
nicht für die Werte der Ableitungen im Punkt $x_{ik}$ repräsentativ 
sein kann.
Alternativ könnten statt der Vorwärtsdifferenzen die {\em Rückwärtsdifferenzen}
\index{Rückwärtsdifferenz}%
\begin{align}
\frac{\partial u}{\partial x} (x_{ik}) 
&\approx
\frac{u(x_{ik}) - u(x_{i-1,k})}{h_x}
&
&\text{und}
&
\frac{\partial u}{\partial y} (x_{ik}) 
&\approx
\frac{u(x_{ik}) - u(x_{i,k-1})}{h_y}
\label{chapter:pde:approxrueckwaerts}
\end{align}
verwendet werden (Abbildung~\ref{buch:pde:1abldiff} unten).
Die Genauigkeit wird dadurch jedoch nicht verbessert, die Punkte
$(\xi,kh_y)$ und $(ih_x,\eta)$, an dem diese Ableitungswerte angenommen
werden, liegen jetzt einfach links bzw.~unterhalb von von $x_{ik}$.

Die Vorwärtsdifferenzen~\eqref{chapter:pde:approx1st} sind also
genauso fehlerhaft wie
die Rückwärtsdifferenzen~\eqref{chapter:pde:approxrueckwaerts},
wenngleich in eine andere Richtung.
Ein Mittelweg könnte ein Differenzenquotient
\begin{align*}
\frac{\partial u}{\partial x}(x_{ik})
&=
\frac{u_{i+1,k}-u_{i-1,k}}{2h_x}
&&\text{und}
&
\frac{\partial u}{\partial y}(x_{ik})
&=
\frac{u_{i,k+1}-u_{i,k-1}}{2h_y}
\end{align*}
über ein symmetrisches Interval der doppelten Länge.
\index{symmetrische Differenz}%
Diese sogenannten {\em symmetrische Differenzen}
(Abbildung~\ref{buch:pde:1abldiff}) sind eher repräsentativ
für die Steigung im Punkt $x_{ik}$, dafür ist die Genauigkeit wegen des
doppelt so langen Intervals kleiner.

\begin{beispiel}
\begin{figure}
\centering
\includegraphics{chapters/70-pde/images/diffex.pdf}
\caption{Differenzenquotienten für die Funktion $f(x)=x^2$.
Die Vorwärts- und Rückwärtsdifferenzen im Punkt $x_0$ ergeben Approximationen
für die Ableitungen, die mit der wahren Ableitung im Punkt $x_0\pm \frac{h}2$
übereinstimmen.
Die symmetrische Differenz im Punkt $x_0$ ergibt genau die Steigung von $f(x)$
im Punkt $x_0$.
\label{buch:pde:diffex}}
\end{figure}
Um die Unterschiede zwischen den Fehlern der verschiedenen
Differenzapproximationen besser zu verstehen, approximieren wir die
Ableitungen der Funktion $f(x)=x^2$ im Punkt $x_0$ und bestimmen den
Fehler sowie den Punkt, in dem die Approximation den korrekten Ableitungswert
annimmt.
Die Vorwärts-, Rückwärts- und symmetrischen Differenzen sind
\begin{align*}
f'(x_0)
&\approx
\frac{f(x_0+h)-f(x_0)}{h}
=
\frac{(x_0+h)^2-x_0^2}{h}
=
2x_0+h
&&=
f'(x_0 + {\textstyle\frac12}h)
\\
&\approx
\frac{f(x_0)-f(x_0-h)}{h}
=
\frac{x_0+^2-(x_0-h)^2}{h}
=
2x_0-h
&&=
f'(x_0 - {\textstyle\frac12}h)
\\
&\approx
\frac{f(x_0+h)-f(x_0-h)}{2h}
=
\frac{(x_0+h)^2-(x_0-h)^2}{2h}
=
\frac{4x_0h}{2h}=2x_0
&&=
f'(x_0).
\end{align*}
Für eine quadratische Funktion liefert also die symmetrische Differenz
den exakten Wert der Ableitung im Punkt $x_0$, während die Vorwärts-
und Rückwertsdifferenzen die Ableitungen in Punkten genau in der Mitte
zwischen den Gitterpunkten rechts bzw.~links von $x_0$
(Abbildung~\ref{buch:pde:diffex}).
\index{quadratische!Funktion}%
\end{beispiel}

\subsubsection{Zweite Ableitungen}
\begin{figure}
\centering
\includegraphics{chapters/70-pde/images/diff2.pdf}
\caption{Differenzenquotienten für die zweiten Ableitungen.
Links die zweite Ableitungen nach $x$, rechts die gemischte Ableitung.
In den roten Kreisen das Gewicht des zugehörigen Wertes der Funktion
im Differenzenquotienten.
\label{buch:pde:diff2}}
\end{figure}
Eine Approximation für die zweite Ableitung können wir als
Differenzenquotient aus der Vorwärts- und der Rückwärtsdifferenz
erhalten:
\begin{align*}
\frac{\partial^2 u}{\partial x^2}(x_{ik})
&\approx
\frac{1}{h_x}\biggl(
\frac{\partial u}{\partial x}(x_{i-\frac12,k})
-
\frac{\partial u}{\partial x}(x_{i+\frac12,k})
\biggr)
=
\frac{1}{h_x}
\cdot
\biggl(
\frac{u_{i+1,k}-u_{ik}}{h_x}
-
\frac{u_{ik}-u_{i-1,k}}{h_x}
\biggr)
\\
&=
\frac{u_{i+1,k}-2u_{ik}+u_{i-1,k}}{h_x^2}.
\end{align*}
Das Problem wird aber schwieriger, wenn eine gemischte Ableitung
approximiert werden soll.
Hier kann jede beliebige Kombination von Vorwärts- und Rückwärtsdifferenzen
verwendet werden.
Mit Vorwärtsdifferenzen erhält man zum Beispiel
\begin{align*}
\frac{\partial^2u}{\partial x\,\partial y}(x_{ik})
&\approx
\frac{1}{h_x}\biggl(
\frac{\partial u}{\partial y}(x_{i+1,k})
-
\frac{\partial u}{\partial y}(x_{ik})
\biggr)
\\
&=
\frac{1}{h_x}\biggl(
\frac{u_{i+1,k+1}-u_{i+1,k}}{h_y}
-
\frac{u_{i,k+1}-u_{ik}}{h_y}
\biggr)
\\
&=
\frac{1}{h_xh_y}(
u_{i+1,k+1}-u_{i+1,k}
-
u_{i,k+1}+u_{ik}
).
\end{align*}
Die in Abbildung~\ref{buch:pde:diff2} dargestellten ``Schablonen''
zeigen schematisch, wie die Ableitungen berechnet werden.
\index{Schablone}%
In den roten Kreisen um die beteiligten Knotenvariablen stehen die
Gewichte, mit denen die Knotenvariablen multipliziert werden müssen.
\index{Knotenvariable}%

\subsubsection{Neumann-Randbedingungen}
Neumann-Randbedingungen geben die Ableitungen in Richtung der Normalen
auf dem Rand vor.
Die Diskretisation auf ein Gitter führt dazu, dass der Rand aus geraden
Teilstücken besteht, wo eine Normalenrichtung leicht zu definieren ist,
und Teilstücken, wo zusätzlicher Aufwand getrieben werden muss, überhaupt
die Richtung der Normalen zu definieren.
Für gerade Teilstücke des Randes können für die Approximation der
Normalableitung Vorwärts- oder Rückwärtsdifferenzen verwendet
werden.

\begin{beispiel}
\begin{figure}
\centering
\includegraphics{chapters/70-pde/images/neumann.pdf}
\caption{Neumann-Randbedingungen sind in einem diskretisierten Gebiet 
einfach auszudrücken, solange die Gitterlinien mit dem Rand des 
Gebietes zusammenfallen.
Sie ergeben eine zusätzliche Gleichung zwischen den rot hinterlegten
Punkten.
Für die runden Teile des Randes gibt es keine einfachen Lösungen.
\label{buch:pde:neumann-analysis}}
\end{figure}
Zur Illustration des Vorgehens approximieren wir die Normalableitungen
für das Rechteckgebiet mit Rändern $x=0$, $x=Nh_x$, $y=0$ und $y=Mh_y$,
welches in Abbildung~\ref{buch:pde:neumann-analysis} dargestellt ist.
In Punkten $x_{0k}$ und $x_{Nk}$ auf den vertikalen Rändern
oder in Punkten $x_{i0}$ und $x_{iM}$ auf den horizontalen Rändern
können wir Vorwärts- bzw.~Rückwärtsdifferenzen verwenden:
\begin{align*}
\frac{\partial u}{\partial n}(x_{0k})
=
\frac{\partial u}{\partial x}(x_{0k})
&\approx
\frac{u_{1k}-u_{0k}}{h_x}
&&\text{und}&
\frac{\partial u}{\partial n}(x_{Nk})
=
\frac{\partial u}{\partial x}(x_{Nk})
&\approx
\frac{u_{Nk}-u_{N-1,k}}{h_x},
\\
\frac{\partial u}{\partial n}(x_{i0})
=
\frac{\partial u}{\partial y}(x_{i0})
&\approx
\frac{u_{i1}-u_{i0}}{h_y}
&&\text{und}&
\frac{\partial u}{\partial n}(x_{iM})
=
\frac{\partial u}{\partial y}(x_{iM})
&\approx
\frac{u_{iM}-u_{i,M-1}}{h_y}.
\end{align*}
Natürlich gelten die oben formulierten Vorbehalte bezüglich der
Zuverlässigkeit dieser Approximation, wir haben aber nicht die Möglichkeit,
symmetrische Differenzen zu verwenden, da keine Funktionswerte ausserhalb
des Gebietes bekannt sind.
\end{beispiel}


\subsubsection{Das Poisson-Problem}
\index{Poisson-Problem}%
\begin{figure}
\centering
\includegraphics{chapters/70-pde/images/laplace.pdf}
\caption{Approximation des Laplace-Operators mit Summen von symmetrischen
Differenzen.
\index{Laplace-Operator}%
In den roten Kreisen die Gewichte, mit denen die Knotenvariablen 
multipliziert werden müssen, um den Ausdruck
\eqref{pde:eqn:poissongl} zu ergeben.
\label{buch:pde:laplace}}
\end{figure}
Das Poisson-Problem ist die Differentialgleichung
\begin{equation}
\Delta u
=
\frac{\partial^2 u}{\partial x^2}
+
\frac{\partial^2 u}{\partial y^2}
=
f
\label{buch:pde:poissondgl}
\end{equation}
auf einem Gebiet $\Omega$, wobei wir als Beispiel ein Quadrat
$\Omega = (0,1) \times (0,1)$ wählen.
Die abstrakte Theorie sagt, dass die Lösung der Differentialgleichung
eindeutig bestimmt ist, Randwerte $u(x)=g(x)$ für $x\in\partial\Omega$
vorgegeben werden.

\begin{figure}
\centering
\includegraphics{chapters/70-pde/images/poisson.pdf}
\caption{Diskretisiertes Gebiet für die Differentialgleichung.
Die Werte in den blauen Punkten auf dem Rand sind durch die Randbedingungen
gegeben, nur die Werte in den roten Punkten müssen bestimmt werden.
\eqref{buch:pde:poissondgl}.
\label{buch:pde:poissongebiet}}
\end{figure}
Wir diskretisieren das Gebiet (Abbildung~\ref{buch:pde:poissongebiet})
mit Hilfe des Gitters mit Gitterkonstanten
$h=h_x=h_y=1/N$.
Wir erhalten die $(N+1)^2$ Unbekannten $u_{ik}$ mit $0\le i\le N$ und 
$0\le k\le N$.
Die Randbedingungen legen die Werte
\begin{align*}
u_{0k}&= g(0, kh)
&
u_{Nk}&= g(1, kh)
&&1\le k\le N
\\
u_{i0}&=g(ih,0)
&
u_{iN}&=g(ih,N)
&&
1\le i\le N
\end{align*}
fest.
$4N$ Unbekannte sind also bereits bestimmt, es bleiben noch
$(N+1)^2-4N = N^2-2N+1=(N-1)^2$ innere Werte zubestimmen.

Die Differentialgleichung kann für jeden Punkt im Inneren des Gebietes
$\Omega$ aufgestellt werden (die roten Punkte in
Abbildung~\ref{buch:pde:poissongebiet}), es gilt
\begin{align}
\Delta u
=
\frac{\partial^2 u}{\partial x^2}(u_{ik})
+
\frac{\partial^2 u}{\partial y^2}(u_{ik})
&=
\frac{u_{i+1,k}-2u_{ik}+u_{i-1,k}}{h^2}
+
\frac{u_{i,k+1}-2u_{ik}+u_{i,k-1}}{h^2}
\notag
\\
f_{ik}=f(x_{ik})
&=
\frac{1}{h^2} ( u_{i+1,k} + u_{i-1,k} + u_{i,k+1} + u_{i,k-1} - 4u_{ik}).
\label{pde:eqn:poissongl}
\end{align}
Alle diese $(N-1)^2$ Gleichungen sind linear.
Insbesondere haben wir gleich viele Gleichungen wie Unbekannte und 
dürfen daher davon ausgehen, dass, wie sich auch beweisen lässt, das
lineare Gleichungssystem
\eqref{pde:eqn:poissongl} 
für die verbleibenden Unbekannten regulär ist.
Die Diskretisation führt also die Lösung der partiellen Differentialgleichung
auf die Lösung eines linearen Gleichungssystems zurück.

%
% waermeleitung.tex -- Beispiel für Problemlösung mit finiten Differenzen
%
% (c) 2020 Prof Dr Andreas Müller, Hochschule Rapperswio
%
\subsection{Wärmeleitungsgleichung
\label{buch:subsection:waermeleitung}}
Als etwas ausführlicheres Beispiel soll in den folgenden Abschnitten
das Wärmeleitungsproblem auf einem Stab mit verschiedenen Randbedingungen
und Methoden gelöst werden.
\index{Stab}%
\index{Wärmeleitung}%
Dieser Abschnitt beginnt damit, das Problem und seine Diskretisierung
zu formulieren.
In den folgenden Abschnitten werden die Gleichungen dann für verschiedene
Randbedingungen numerisch gelöst.

\subsubsection{Die Wärmeleitungsgleichung}
Die Temperaturverteilung $u(x,t)$ auf einem Stab mit $x$-Koordinaten
zwischen $0$ und $1$ zur Zeit $t$ wird durch eine partielle
Differentialgleichung auf dem Gebiet
\index{Temperaturverteilung}%
\[
\Omega = \{ (x,t)\;|\; 0 < x < 1\wedge 0<t\}
\]
beschrieben.
In der Wärmeleitungsgleichung
\begin{equation}
\frac{\partial u}{\partial t}
=
\kappa\frac{\partial^2 u}{\partial x^2}
\label{buch:pde:waerme:gleichung}
\end{equation}
ist die $\kappa$ eine Konstante, die Wärmeleitfähigkeit und
Wärmekapazität des Materials charaktersiert.
\index{Wärmeleitfähigkeit}%
\index{Wäermekapazität}%

\subsubsection{Randbedingungen}
Zudem müssen Randbedingungen zur Zeit $t=0$ und an den Enden
des Stabes bei $x=0$ oder $x=1$ erfüllt sein.
\index{Randbedingung}%
Zur Zeit $t=0$ muss die initiale Temperaturverteilung $f(x)$ 
spezifiziert werden.
\index{Temperaturverteilung}%

Dirichletrandbedingungen
\[
\begin{aligned}
u(x,0)&=f(x)&&x\in[0,1]
\end{aligned}
\]
am Rand des Intervals bedeuten, dass die Enden des Stabes mit Wärmereservoirs
verbunden sind, welche ihn auf einer vorgegebenen Temperatur halten.
\index{Wärmereservoir}%

Neumann-Randbedingungen
\[
\begin{aligned}
\frac{\partial u}{\partial n}(x,t)=\frac{\partial u}{\partial x}(x,t)&=g(x)&&x\in \{0,1\}.
\end{aligned}
\]
spezifizieren den Temperaturgradienten am Rande und legen damit fest,
wieviel Wärmeenergie in den Stab hineinfliesst oder ihn verlässt.
\index{Temperaturgradient}%
Im Falle $g(x)=0$ verschwindet der Gradient und Wärmefluss ist unterbunden.
Dies entspricht einem thermisch isolierten Stab.
\index{Wärmefluss}%

\subsubsection{Energie}
Die physikalische Interpretation der
Gleichung~\eqref{buch:pde:waerme:gleichung}
erlaubt, das Verhalten der Lösung abzuschätzen.
Dazu berechnen wir die Gesamtenergie
\[
U(t) = \int_0^1 u(x,t)\,dx,
\]
die im Intervall enthalten ist.
\index{Energie}%
Die Wärmeleitungsgleichung~\ref{buch:pde:waerme:gleichung} erlaubt nun,
die Änderung der Energie mit der Zeit zu bestimmen.
Es gilt
\begin{align*}
\frac{dU(t)}{dt}
&=
\frac{d}{dt} \int_0^1 u(x,t)\,dx
=
\int_0^1 \frac{\partial u}{\partial t} (x,t)\,dx
=
\kappa \int_0^1 \frac{\partial^2u}{\partial x^2}(x,t)\,dx
=
\kappa \biggl[\frac{\partial u}{\partial x}\biggr]_0^1.
\end{align*}
Daraus lässt sich zum Beispiel ablesen, dass sich die Energie in
einem isolierten Stab, wo die Ableitungen an den Intervallenden
verschwinden, nicht ändert.
Anders ausgedrückt:
die Energie bleibt dann konstant, wenn die Steigung
in beiden Enden gleich gross ist, was gleichbedeutend ist damit,
dass der Wärmefluss durch beide Intervallenden gleich gross ist.
\index{Wärmefluss}%

In allen Fällen kann man das Verhalten der Lösung für $t\to\infty$
bestimmen.
Dirichlet-Rand\-be\-dingun\-gen 
\[
u(0,t) = u_0 \qquad\text{und}\qquad u(1,t) = u_1
\]
sagen zum Beispiel, dass die beiden Enden des Stabes auf Temperatur
$u_0$ und $u_1$ gehalten werden.
Mit der Zeit wird sich eine stationäre Temperaturverteilung einstellen,
also eine Temperaturverteilung, die sich mit der Zeit nicht mehr ändert.
Eine solche hat zweite Ableitung
\[
\frac{\partial^2u}{\partial x^2}
=
\frac{1}{\kappa} \frac{\partial u}{\partial t} = 0.
\]
Die Funktion $x\mapsto u(x,t)$ muss also linear sein, was auf die
stationäre Temperaturverteilung
\[
u_\infty(x) = xu_1 + (1-x)u_0
\]
führt.

Für konstante Neumann-Randbedingungen
\[
\frac{\partial u}{\partial t}(0) = v_0
\qquad\text{und}\qquad
\frac{\partial u}{\partial t}(1) = v_1
\]
ist der Wärmefluss in den Stab konstant, es gilt
\[
\frac{dU(t)}{dt} = (v_1-v_0)t + U(0).
\]
Insbesondere kann man nicht erwarten, dass es eine stationäre Lösung
gibt.
Vielmehr erwartet man eine Lösung, die linear mit der Zeit anwächst.
Eine lineare Funktion von $x$ kann nicht funktionieren, weil diese
auf gleichen Wärmefluss durch die beiden Intervallenden führen würde.
Wir versuchen daher den quadratischen Ausdruck.
\[
u_s(x,t) = ax^2 + bx + c + d\cdot t.
\]
Tatsächlich sind die Ableitungen
\begin{align*}
\frac{\partial u_s}{\partial t}
&=
d
\\
\frac{\partial^2 u_s}{\partial x^2}
&=
2a.
\end{align*}
Eingesetzt in die Differentialgleichung finden wir den Wert von $a$
\begin{equation*}
d=2a\kappa
\qquad\Leftrightarrow\qquad
a=\frac{d}{2\kappa}.
\end{equation*}
Um die Werte von $a$ und $b$ zu bestimmen, müssen wir die Randbedingungen
bemühen:
\begin{align*}
v_0=\frac{\partial u}{\partial x}(0) &= b &&\Rightarrow& b &= v_0
\\
v_1=\frac{\partial u}{\partial x}(1) &= 2a + b &&\Rightarrow& a &= \frac{v_1-v_0}2
\end{align*}
Damit ist auch $d=\kappa (v_1-v_0)$ bestimmt.
Einzig $c$ lässt sich auf diesem Weg nicht bestimmen, dazu sind weitere
Dirichlet-Randbedingungen erfoderlich.

\subsubsection{Diskretisation}
Zur Diskretisation verwenden wir ein Gitter mit Gitterkonstanten
$h_x=1/N$ und $h_t$.
\index{Gitterkonstante}%
Die zu bestimmenden Unbekannten sind die $u_{ik}$ mit
$0\le i\le N$ und $k\ge 0$.
Die Randbedingungen auf dem Rand $t=0$ geben die Werte 
\[
u_{i0} = u(ih_x,0) = f(ih_x) =: f_i
\]
vor.

Für Dirichlet-Randbedingungen an den Intervallenden legen die Werte von
$u_{ik}$ für $i=0$ und $i=N$ fest.
Die Variablen $u_{0k}$ und $u_{Nk}$ sind also nicht Unbekannte sondern
vorgegebene Werte.

Die Neumann-Randbedingungen am linken und rechten Rand können mit Hilfe
der Vorwärts- bzw.~Rückwärts-Differenzen approximiert werden:
\begin{align*}
\frac{\partial u}{\partial n}(x_{0k})
&\approx
\frac{u_{1k}-u_{0k}}{h_x}
=
0
&&\text{und}&
\frac{\partial u}{\partial n}(x_{0k})
&\approx
\frac{u_{Nk}-u_{N-1,k}}{h_x}
=
0
\intertext{Dies bedeutet, dass die beiden Werte nahe des Randes
gleich sind}
u_{0k}&=u_{1k}&&\text{und}&u_{Nk}&=u_{N-1,k}.
\end{align*}
Die Differentialgleichung verwendet die zweite Ableitung nach $x$, 
für die wir die Approximation
\index{Approximation}%
\[
\frac{\partial^2u}{\partial x^2}(x_{ik})
=
\frac{u_{i+1,k}-2u_{ik}+u_{i-1,k}}{h_x^2}
\]
verwenden können.

\subsubsection{Matrixform}
Da die Werte $u(x,0)=f(x)$ von $u$ zur Zeit $t=0$ bereits bekannt
sind, wird die numerische Lösung nacheinander die Variablen
$u_{ik}$ mit $k=1,2,3,\dots$ bestimmen.
\index{Matrixform}%
Der Schritt $k\to k+1$ kann etwas kompakter beschrieben und vor allem
leichter analysiert werden, wenn wir die Werte $u_{ik}$ für gegebenes $k$
in den Vektor
\[
u_k = \begin{pmatrix}
u_{0k}\\
u_{1k}\\
\vdots\\
u_{ik}\\
\vdots\\
u_{Nk}
\end{pmatrix}
\]
zusammenfassen.
Für homogene Randbedingungen ist der Zusammnenhang zwischen $u_k$
und $u_{k+1}$ eine lineare Funktion, es muss also eine Matrix geben,
welche $u_{k+1}=Au_k$.

Sind die Randbedingungen nicht homogen, kann man nicht mehr unbeding
eine zeitunabhängige Matrix $A$ finden, vielmehr kann die Matrix
in jedem Zeitschritt verschieden sein.
Ausserdem kann in jedem Zeitschritt ein konstanter Wert hinzukommen.
Insgesamt gibt es also eine Matrix $A_k$ und ein Vektor $b_k$ derart,
dass $u_{k+1} = A_ku_k + b_k$.
\index{Zeitschritt}%

Für die erste Ableitung nach der Zeit könnten Vorwärts- oder
Rückwärtsdifferenzen verwendet werden, in beiden Fällen entsteht
ein unvermeidbarer Fehler und ein jeweils anderes numerisches
Lösungsverfahren.
\index{Vorwärtsdifferenz}%
\index{Rückwärtsdifferenz}%

\subsubsection{Euler-Verfahren}
\begin{figure}
\centering
\includegraphics{chapters/70-pde/images/euler.pdf}
\caption{Euler-Verfahren für das Wärmeleitungsproblem.
\index{Euler-Verfahren}%
\label{buch:pde:figure:euler}}
\end{figure}
Wir müssen jetzt die Differentialgleichung des Wärmeleitungsproblems
diskretisieren.
Für die zweite Ableitung verwenden wir zweite Differenzen.
Für die ersten Ableitungen haben wir verschiedene Optionen, wir
verwenden Vorwärtsdifferenzen, also
\begin{align}
\frac{\partial u}{\partial x}(x_{ik})
\approx
\frac{u_{i,k+1}-u_{ik}}{h_t}
&=
\kappa
\frac{\partial^2u}{\partial x^2}(x_{ik})
\approx
\kappa
\frac{u_{i,k-1}-2u_{ik}+u_{i,k+1}}{h_x^2}
\notag
\\
u_{i,k+1}
&=
u_{ik} + \frac{h_t\kappa}{h_x^2} (u_{i-1,k}-2u_{ik}+u_{i+1,k})
\label{buch:pde:waerme:euler}
\end{align}
für die inneren Punkte.
In Matrixform kann dies als
\[
\begin{pmatrix}
\vdots\\
u_{i,k+1}\\
\vdots
\end{pmatrix}
=
\begin{pmatrix}
&&&&\\
0&\dots
	&\displaystyle\frac{h_t\kappa}{h_x^2}
		&\displaystyle1-\frac{2h_t\kappa}{h_x^2}
			&\displaystyle\frac{h_t\kappa}{h_x^2}
				&\dots
					&0\\
&&&&
\end{pmatrix}
\begin{pmatrix}
\vdots\\
u_{i-1,k}\\
u_{ik}\\
u_{i+1,k}\\
\vdots
\end{pmatrix}
\]
geschrieben werden.
Wir kürzen den gemeinsamen Faktor $c=h_t\kappa/h_x^2$ ab und erhalten
die Form
\[
u_{k+1}
=
\begin{pmatrix}
\ddots&\ddots&\ddots&    &      &      &      \\
      &     c&  1-2c&  c &      &      &      \\
      &      &    c &1-2c&  c   &      &      \\
      &      &      &  c &1-2c  &  c   &      \\
      &      &      &    &\ddots&\ddots&\ddots
\end{pmatrix}
u_k.
\]
Die Lösung wird sich durch Iteration dieser Matrix finden lassen,
Konvergenz wird allein von $c$ abhängen.
Wir erwarten also, ein Konvergenzkriterium basierend auf $c$.
\index{Konvergenzkriterium}

Die Gleichung~\eqref{buch:pde:waerme:euler} gilt nur für innere Punkte.
Für die Variablen $u_{0k}$, $u_{1k}$, $u_{N-1,k}$ und $u_{Nk}$ müssen
wir aus den Randbedingungen zusätzliche Gleichungen für die Randwerte
ableiten.

\subsubsection{Rückwärts-Verfahren}
\begin{figure}
\centering
\includegraphics{chapters/70-pde/images/rueckwaerts.pdf}
\caption{Verwendung von Rückwärts-Differenzen für die Wärmeleitungsgleichung
\label{buch:pde:figure:rueckwaerts}}
\end{figure}
Statt der Vorwärts-Differenz kann man auch die Rückwärts-Differenz für
die erste Ableitung nach der Zeit verwenden.
Die diskretisierte Gleichung wird dann
\begin{align*}
\frac{u_{ik}-u_{i,k-1}}{h_t}
\approx
\frac{\partial u}{\partial x}(x_{ik})
&=
\kappa\frac{\partial^2 u}{\partial x^2}(x_{ik})
\approx
\frac{u_{i-1,k}-2u_{ik} + u_{i+1,k}}{h_x}.
\\
-cu_{i-1,k}
+
(1+2c)u_{ik}
-cu_{i+1,k}
&=u_{i,k-1}.
\end{align*}
In Matrixform kann man dies als
\[
\begin{pmatrix}
%\ddots&\ddots&\ddots&    &      &      &      \\
      &    -c&  1+2c& -c &      &      &      \\
      &      &   -c &1+2c& -c   &      &      \\
      &      &      & -c &1+2c  & -c   &      \\
      &      &      &    &\ddots&\ddots&\ddots
\end{pmatrix}
u_k
=
u_{k-1}
\]
schreiben.
Wieder gelten diese Gleichungen nur für innere Punkte.
Abbildung~\ref{buch:pde:figure:rueckwaerts} zeigt, wie die Gleichungen
die Knotenwerte untereinander verknüpfen.

\subsubsection{Crank-Nicholson-Verfahren}
\begin{figure}
\centering
\includegraphics{chapters/70-pde/images/cngrid.pdf}
\caption{Verwendung der Knotenvariablen im Crank-Nicholson-Verfahren.
\label{buch:pde:figure:cngrid}}
\end{figure}
\index{Crank-Nicholson-Verfahren}%
Sowohl Vorwärts- wie auch Rückwärts-Differenzen bestimmen die erste
Ableitung nach der Zeit eigentlich für einen Zeitpunkt zwischen 
den Vielfachen von $h_t$.
Für diese Zeitpunkte stehen aber keine Funktionswerte zur Verfügung, 
um damit die Gleichung aufzustellen.
Das Crank-Nicholson-Verfahren schlägt daher vor, den Mittelwert der
zweiten Ableitungen zu benachbarten Zeitpunkten als Wert für den
Zwischenzeitpunkt zu verwenden.
Damit wird die Wärmeleitungsgleichung approximiert durch
\begin{align*}
\frac{u_{i,k+1}-u_{ik}}{h_t}
\approx
\frac{\partial u}{\partial x}(x_{ik})
&=
\kappa
\frac{\partial^2 u}{\partial x^2}(x_{ik})
\approx
\frac{\kappa}2\biggl(
\frac{u_{i-1,k+1}-2u_{i,k+1}+u_{i+1,k+1}}{h_x^2}
+
\frac{u_{i-1,k}-2u_{i,k}+u_{i+1,k}}{h_x^2}
\biggr)
\\
\Rightarrow\hspace{1cm}
2u_{i,k+1}-2u_{ik}
&=
c(
u_{i-1,k+1}-2u_{i,k+1}+u_{i+1,k+1}
+
u_{i-1,k}-2u_{i,k}+u_{i+1,k}
)
\end{align*}
Verschiebt man die Terme nach Zeitpunkt auf die beiden Seiten
der Gleichung erhält man
\begin{equation}
-cu_{i-1,k+1}+2(1+c)u_{i,k+1}-cu_{i+1,k+1}
=
cu_{i-1,k}+2(1-c)u_{i,k}+cu_{i+1,k},
\end{equation}
was sich leichter in Matrixform 
\begin{equation}
\begin{pmatrix}
&\ddots& \ddots &        &        &        &        \\
&  -c  & 2(1+c) &  -c    &        &        &        \\
&      &  -c    & 2(1+c) &  -c    &        &        \\
&      &        &  -c    & 2(1+c) &  -c    &        \\
&      &        &        & \ddots & \ddots &        
\end{pmatrix}
u_{k+1}
=
\begin{pmatrix}
&\ddots&\ddots  &        &        &        &        \\
&   c  & 2(1-c) &   c    &        &        &        \\
&      &   c    & 2(1-c) &   c    &        &        \\
&      &        &   c    & 2(1-c) &   c    &        \\
&      &        &        &\ddots  & \ddots &        
\end{pmatrix}
u_k
\end{equation}
bringen lässt.

\subsection{Wärmeleitungsgleichung mit Dirichlet-Randbedingungen
\label{buch:pde:subsection:waerme:dirichlet}}
Die bisher formulierten Gleichungen berücksichtigen die
Randbedingungen nicht.
Dirichlet-Rand\-bedingungen am linken und rechten Rand des Intervalls
legen die Werte $u_{0k}$ und $u_{Nk}$ fest.
Wir möchten dies wieder in Matrixform schreiben.
Da die Randbedingungen nicht von $u$ abhängen, muss dies in der Form
$u_{k+1}=Au_k + b_k$ möglich sein, was nur geht, wenn
\begin{equation}
u_{k+1}
=
\underbrace{
\begin{pmatrix}
 0  &   0  &  0   &      &      &      &      &      &      \\
 c  & 1-2c &  c   &      &      &      &      &      &      \\
    &  c   & 1-2c &  c   &      &      &      &      &      \\
    &      &\ddots&\ddots&\ddots&      &      &      &      \\
    &      &      &  c   & 1-2c &  c   &      &      &      \\
    &      &      &      &\ddots&\ddots&\ddots&      &      \\
    &      &      &      &      &  c   & 1-2c &  c   &      \\
    &      &      &      &      &      &  c   & 1-2c &  c   \\
    &      &      &      &      &      &  0   &  0   &  0   
\end{pmatrix}}_{\displaystyle A\mathstrut}
u_k
+
\underbrace{
\begin{pmatrix}
g_{0k}\\
0     \\
0     \\
\vdots\\
0     \\
\vdots\\
0     \\
0     \\
g_{Nk}
\end{pmatrix}}_{\displaystyle b_k\mathstrut}
\end{equation}
für die Randwerte $g$.

Für homogene Randbedingungen wird die Situation etwas einfacher zu
analysieren, denn dann ist $b_k=0$.
\index{homogene Randbedingung}%
\index{Randbedingung!homogen}%
Wir erwarten, dass die Lösung exponentiell gegen $0$ konvergiert,
das System ``kühlt aus''.
\index{exponentiell}%
Da die Randbedingungen $u_{0k}=u_{Nk}=0$ erzwingen, kann man diese 
Variablen weglassen, die Vektoren $u$ und die Matrix $A$ verkürzen sich auf
\begin{equation}
\tilde{u}_{k+1}
=
\begin{pmatrix}
u_{1,k+1}\\
u_{2,k+1}\\
\vdots\\
u_{i,k+1}\\
\vdots\\
u_{N-1,k+1}\\
u_{N,k+1}
\end{pmatrix}
=
\underbrace{
\begin{pmatrix}
 1-2c &  c   &      &      &      &      &      \\
  c   & 1-2c &  c   &      &      &      &      \\
      &\ddots&\ddots&\ddots&      &      &      \\
      &      &  c   & 1-2c &  c   &      &      \\
      &      &      &\ddots&\ddots&\ddots&      \\
      &      &      &      &  c   & 1-2c &  c   \\
      &      &      &      &      &  c   & 1-2c \\
\end{pmatrix}}_{\displaystyle \tilde{A}\mathstrut}
\begin{pmatrix}
u_{1k}\\
u_{2k}\\
\vdots\\
u_{ik}\\
\vdots\\
u_{N-1,k}\\
u_{Nk}
\end{pmatrix}
=
A
\tilde{u}_k.
\end{equation}
Die Lösung $u$ entsteht also durch Iteration mit der Matrix $\tilde{A}$.
Der Spektralradius der Matrix gibt Auskunft darüber, ob das Verfahren
konvergiert.
\index{Spektralradius}%
Da die Matrix $\tilde{A}$ symmetrisch ist, sind alle Eigenwerte reell.
\index{symmetrisch}%
\begin{figure}
\centering
\includegraphics{chapters/70-pde/images/explizitspektrum.pdf}
\caption{Spektrum der Matrix $\tilde{A}$ für $N=16$,
die sich für das Euler-Verfahren mit
\index{Euler-Verfahren}%
homogenen Dirichlet-Randwerte ergibt.
Nur für $c<2$ ist der Spektralradius $<1$.
\label{buch:pde:waerme:explizitspektrum}}
\end{figure}
In Abbildung~\ref{buch:pde:waerme:explizitspektrum} sind die Eigenwerte
von $\tilde{A}$ dargestellt.
Man kann erkennen, dass nur für $c<\frac12$ der Spektralradius $<1$
wird.
\index{Spektralradius}%
Daraus ergibt sich das Kriterium
\[
\frac{2\kappa h_t}{h_x^2} < 1
\]
für die Konvergenz des Verfahrens.

Für das Rückwärtsverfahren sieht die Matrix etwas anders aus, es gilt
\index{Rückwärtsverfahren}%
\begin{equation}
\begin{pmatrix}
   0  &    0  &   0  &      &      &      &      &      &      \\
  -c  &  1+2c &  -c  &      &      &      &      &      &      \\
      &   -c  & 1+2c &  -c  &      &      &      &      &      \\
      &       &\ddots&\ddots&\ddots&      &      &      &      \\
      &       &      &  -c  & 1+2c &  -c  &      &      &      \\
      &       &      &      &\ddots&\ddots&\ddots&      &      \\
      &       &      &      &      &  -c  & 1+2c &  -c  &      \\
      &       &      &      &      &      &  -c  & 1+2c &  -c  \\
      &       &      &      &      &      &   0  &   0  &   0  
\end{pmatrix}
u_{k+1}
=
Bu_{k+1}
=
u_k + b_k.
\end{equation}
Für den Fall homogener Randbedingungen werden auch hier wieder
nur die ``inneren'' Koeffizienten
\[
\tilde{B}
=
\begin{pmatrix}
 1+2c &  -c  &      &      &      &      &      \\
  -c  & 1+2c &  -c  &      &      &      &      \\
      &\ddots&\ddots&\ddots&      &      &      \\
      &      &  -c  & 1+2c &  -c  &      &      \\
      &      &      &\ddots&\ddots&\ddots&      \\
      &      &      &      &  -c  & 1+2c &  -c  \\
      &      &      &      &      &  -c  & 1+2c 
\end{pmatrix}
\]
der Matrix $B$ nötig.
Die Lösung findet man dann aus $\tilde{B}\tilde{u}_{k+1}=\tilde{u}_k$
durch Iteration $\tilde{u}_{k+1}=\tilde{B}^{-1} \tilde{u}_k$.
Konvergenz wird wieder bestimmt durch den Spektralradius $\varrho(\tilde{B})$,
der, wie Abbildung~\ref{buch:pde:waerme:implizitspektrum} zeigt, immer
$<1$ ist.
Das Rückwärtsverfahren ist also immer konvergent, ganz unabhängig von
den relativen Schrittweiten.
\begin{figure}
\centering
\includegraphics{chapters/70-pde/images/implizitspektrum.pdf}
\caption{Eigenwertspektrum der Matrix $\tilde{B}^{-1}$ für das 
Rückwärts-Verfahren.
Der Spektralradius ist unabhängig von $c$ immer $<1$.
\label{buch:pde:waerme:implizitspektrum}}
\end{figure}

Für das Crank-Nicholson-Verfahren ist, welches wir hier nur für homogene
Randbedingungen betrachten wollen.
Aus der Gleichung
\begin{gather*}
\tilde{C}u_{k+1}
=
\begin{pmatrix}
2(1+c)&  -c  &      &      &      &      &      \\
   -c &2(1+c)&  -c  &      &      &      &      \\
      &   -c &2(1+c)&  -c  &      &      &      \\
      &      &\ddots&\ddots&\ddots&      &      \\
      &      &      &\ddots&\ddots&\ddots&      \\
      &      &      &      &   -c &2(1+c)&  -c  \\
      &      &      &      &      &  -c  &2(1+c)
\end{pmatrix}
u_{k+1}
\qquad
\qquad
\qquad
\\
\qquad
\qquad
\qquad
=
\tilde{D}u_{k}
=
\begin{pmatrix}
2(1-c)&   c  &      &      &      &      &      \\
    c &2(1-c)&   c  &      &      &      &      \\
      &    c &2(1-c)&   c  &      &      &      \\
      &      &\ddots&\ddots&\ddots&      &      \\
      &      &      &\ddots&\ddots&\ddots&      \\
      &      &      &      &    c &2(1-c)&   c  \\
      &      &      &      &      &   c  &2(1-c)
\end{pmatrix}
u_k
\end{gather*}
folgern wir, dass die Iteration $u_{k+1} = \tilde{C}^{-1}\tilde{D} u_k$
die Lösung der Gleichung liefert.
\begin{figure}
\centering
\includegraphics{chapters/70-pde/images/cnspektrum.pdf}
\caption{Eigenwertspektrum für das Crank-Nicholson-Verfahren
\label{buch:pde:waerme:cnspektrum}}
\end{figure}
Das Spektrum von $\tilde{C}^{-1}\tilde{D}$ ist in
Abbildung~\ref{buch:pde:waerme:cnspektrum} dargestellt und zeigt,
dass auch das Crank-Nicholson-Verfahren immer konvergiert unabhängig von
der Schrittweite $h_t$.

\subsection{Wärmeleitungsgleichung mit Neumann-Randbedingungen
\label{buch:pde:subsection:waerme:neumann}}
Die Neumann-Randbedingungen legen nicht Werte in den Randpunkten fest,
sondern liefern nur Gleichungen zwischen den Variablen nahe dem Rand.
Sie ändern damit die Koeffizientenmatrix des Gleichungssystems und haben
nicht nur Einfluss auf konstante Vektoren $b_k$ wie in
Abschnitt~\ref{buch:pde:subsection:waerme:dirichlet}.

\subsubsection{Neumann-Randbedingungen}
Wir betrachten in diesem Abschnitt nur homogene Neumann-Randbedingungen.
Die Randbedingung am linken und rechten Rand verlangt, dass $u_{0k}=u_{1k}$
und $u_{N-1,k}=u_{Nk}$ gelten muss.
Dies bedeutet, dass die Randwerte $u_{0k}$ und $u_{Nk}$ mit der
gleichen Formel berechnet werden können wie die Werte für $u_{1k}$ und
$u_{N-1,k}$.
Wir können damit die Variablen $u_{0k}$ und $u_{Nk}$ aus allen
Gleichungen eliminieren, indem wir sie durch $u_{1k}$ und $u_{N-1,k}$ 
ersetzen.

\subsubsection{Euler-Verfahren}
Aus den Gleichungen
\begin{align*}
u_{1,k+1} &= c u_{0k} + (1-2c) u_{1k} + c u_{2k} \\
u_{N-1,k+1} &= c u_{N-2,k} + (1-2c) u_{N-1,k} + c u_{Nk} 
\end{align*}
am Rand des Gebietes werden nach den Ersetzungen
$u_{0k}\to u_{1k}$
und
$u_{Nk}\to u_{N-1,k}$
die Gleichungen
\begin{align*}
u_{1,k+1}
&=
(1-c) u_{1k} + c u_{2k}
\\
u_{N-1,k+1}
&=
c u_{N-2,k} + (1-c) u_{N-1,k}.
\end{align*}
Die Matrix des Euler-Verfahrens wird damit zu
\begin{equation}
\tilde{u}_{k+1}
=
\underbrace{
\begin{pmatrix}
 1- c &  c   &      &      &      &      &      \\
  c   & 1-2c &  c   &      &      &      &      \\
      &\ddots&\ddots&\ddots&      &      &      \\
      &      &  c   & 1-2c &  c   &      &      \\
      &      &      &\ddots&\ddots&\ddots&      \\
      &      &      &      &  c   & 1-2c &  c   \\
      &      &      &      &      &  c   & 1- c 
\end{pmatrix}}_{\displaystyle A}
\tilde{u}_k.
\end{equation}
Die Konvergenz des Verfahrens hängt wieder vom Spektralradius ab.
Diesmal können wir aber nicht erwarten, dass alle Eigenwerte
Betrag $<1$ haben werden.
Die Zeilen- und Spaltensumme der Matrix $A$ ist immer $1$, d.~h.~die
ein konstanter Vektor wird auf sich selbst abgebildet.
Insbesondere gibt es einen Eigenvektor zum Eigenwert $1$,
der Spektralradius kann also nicht kleiner als $1$ sein. 
\begin{figure}
\centering
\includegraphics{chapters/70-pde/images/explizitneumann.pdf}
\caption{Eigenwertspektrum des Euler-Verfahrens für
Neumann-Randbedingungen.
Der Eigenwert $1$ ist einfach,
für $c<\frac12$ ist der Spektralradius $1$.
\label{buch:pde:waerme:explizit:neumannspektrum}}
\end{figure}
Für $c<1$ konvergiert das Verfahren daher gegen die konstante
Temperaturverteilung.
\begin{figure}
\centering
\includegraphics[width=\hsize]{chapters/70-pde/images/explizit.jpg}
\caption{Lösung der Wärmeleitungsgleichung mit dem Euler-Verfahren.
\label{buch:pde:waerme:figure:euler}}
\end{figure}
Die numerische Durchführung des Verfahrens mit geeigneten Werten von
$h_x$ und $h_t$ derart, dass $c<\frac12$ ist, führt auf die
Lösungsfläche in Abbildung~\ref{buch:pde:waerme:figure:euler}.

\subsubsection{Rückwärtsverfahren}
Auch für das Rückwärtsverfahren können wir für Neumann-Randbedinungen
die Matrixgleichung
\begin{equation}
Bu_k
=
\begin{pmatrix}
 1+c & -c   &      &      &      &      &      &     \\
 -c  & 1+2c & -c   &      &      &      &      &     \\
     &  -c  & 1+2c & -c   &      &      &      &     \\
     &      &\ddots&\ddots&\ddots&      &      &     \\
     &      &      &\ddots&\ddots&\ddots&      &     \\
     &      &      &      &  -c  & 1+2c & -c   &     \\
     &      &      &      &      &  -c  & 1+2c & -c  \\
     &      &      &      &      &      &  -c  & 1+c 
\end{pmatrix}
u_k
=
u_{k-1}
\end{equation}
finden.
Auch diese Matrix hat den konstanten Vektor als einzigen Eigenvektor
zum Eigenwert $1$.
Wie bei Dirichlet-Randbedingungen ist das Spektrum, dargestellt in
Abbildung~\ref{buch:pde:waerme:implizit:neumannspektrum}, bis auf den
einen Eigenwert $1$ im Intervall $(0,1)$ enthalten, so dass das
Verfahren konvergiert.
Die Lösung ist dargestellt in
Abbildung~\ref{buch:pde:waerme:figure:rueckwaerts}.
\begin{figure}
\centering
\includegraphics{chapters/70-pde/images/implizitneumann.pdf}
\caption{Das Eigenwertspektrum für das Rückwärtsverfahren für
Neumann-Randbedingungen enthält eine einzelnen Eigenwert $1$, alle
anderen Eigenwerte haben Betrag $<1$.
\label{buch:pde:waerme:implizit:neumannspektrum}}
\end{figure}
\begin{figure}
\centering
\includegraphics[width=\hsize]{chapters/70-pde/images/implizit.jpg}
\caption{Lösung der Wärmeleitungsgleichung mit dem Rückwärtsverfahren.
\label{buch:pde:waerme:figure:rueckwaerts}}
\end{figure}

\subsubsection{Crank-Nicholson-Verfahren}
Für das Crank-Nicholson-Verfahren sind die Gleichungen am Rand
des Intervals
\begin{align*}
-cu_{0,k+1}+2(1+c)u_{1,k+1}-cu_{2,k+1}
&=
cu_{0k}+2(1-c)u_{1k}+cu_{2,k}
\\
(2+c)u_{1,k+1}-cu_{2,k+1}
&=
(2-c)u_{1k}+cu_{2,k},
\end{align*}
so dass die Matrixgleichung
\begin{gather*}
Cu_{k+1}
=
\begin{pmatrix}
 2+c &  -c  &      &      &      &      &      \\
  -c &2(1+c)&  -c  &      &      &      &      \\
     &   -c &2(1+c)&  -c  &      &      &      \\
     &      &\ddots&\ddots&\ddots&      &      \\
     &      &      &\ddots&\ddots&\ddots&      \\
     &      &      &      &   -c &2(1+c)&  -c  \\
     &      &      &      &      &  -c  &  2+c
\end{pmatrix}
u_{k+1}
\qquad
\qquad
\qquad
\\
\qquad
\qquad
\qquad
=
\begin{pmatrix}
 2-c &   c  &      &      &      &      &      \\
   c &2(1-c)&   c  &      &      &      &      \\
     &    c &2(1-c)&   c  &      &      &      \\
     &      &\ddots&\ddots&\ddots&      &      \\
     &      &      &\ddots&\ddots&\ddots&      \\
     &      &      &      &    c &2(1-c)&   c  \\
     &      &      &      &      &   c  &  2-c
\end{pmatrix}
u_k
=
Du_k
\end{gather*}
wird.
Daraus erhält man als Iterationsverfahren:
\[
u_{k+1} = C^{-1}Du_k.
\]
Das Eigenwertspektrum (Abbildung~\ref{buch:pde:waerme:cranknicholson:spektrum})
hat wieder einen einfachen Eigenwert $1$,
alle anderen Eigenwerte haben Betrag $<1$, das Verfahren ist
konvergent obwohl der Spektralradius exakt $1$ ist.
\begin{figure}
\centering
\includegraphics{chapters/70-pde/images/cnspektrum.pdf}
\caption{Das Eigenwertspektrum des Crank-Nicholson-Verfahrens zeigt
einen einzelnen Eigenwert $1$ 
\label{buch:pde:waerme:cranknicholson:spektrum}}
\end{figure}
\begin{figure}
\centering
\includegraphics[width=\hsize]{chapters/70-pde/images/cranknicholson.jpg}
\caption{Lösung der Wärmeleitungsgleichung mit dem Crank-Nicholson-Verfahren
\label{buch:pde:waerme:figure:cranknicholson}}
\end{figure}

\begin{figure}
\centering
\includegraphics[width=\hsize]{chapters/70-pde/images/combined.jpg}
\caption{Alle drei Lösungsverfahren liefern ähnliche Lösungen,
doch das Euler-Verfahren (rot) weicht stark von den beiden anderen
Verfahren ab.
Das Rückwärts-Verfahren und das Crank-Nicholson-Verfahren stimmen
weitgehend überein.
\label{buch:pde:waerme:figure:combined}}
\end{figure}
In Abbildung~\ref{buch:pde:waerme:figure:combined} sind die drei
Lösungen im gleichen Graphen dargestellt. 
Die Lösung des Euler-Verfahrens weicht stark von den anderen Verfahren ab.
Die Wärmeleitungsgleichung hat Lösungen, bei denen sich Wärme mit
unendlicher Geschwindigkeit ausbreitet.
\index{Euler-Verfahren}%
\index{Ausbreitungsgeschwindigkeit}%
Im Euler-Verfahren kann sich ein Temperatursprung nur um $h_x$ in
jeder Iteration ausbreiten, also maximal mit der Geschwindigkeit $h_x/h_t$.
Daher verhält sich das Euler-Verfahren so, wie wenn die Wärmeleitfähigkeit
reduziert wäre, was zu dem ausgeprägteren ``Buckel'' der Euler-Lösung
führt.
\index{Wärmeleitfähigkeit}%



%
% stabilitaet.tex -- Computational mode
%
% (c) 2020 Prof Dr Andreas Müller, Hochschule Rapperswio
%
\subsection{Stabilität und Computational Mode
\label{pde:subsection:stabilitaet}}
In der Diskussion des Wärmeleitungsproblems haben wir für die Zeitableitung
nicht versucht, symmetrische Differenzen zu verwenden.
Tatsächlich gibt es einen wichtigen Grund dafür, der in diesem Abschnitt
untersucht werden soll.
Es wird sich zeigen, dass symmetrische Differenzen zusätzliche instabile
Lösungen erzeugen können, die nichts mit realen Lösungen der
Differentialgleichung zu tun haben.
\index{instabil}%
\index{instabile Lösung}%
Dieser sogenannte {\em Computational Mode}
\index{Computational Mode}%
führt auf nutzlose Resultate und dürfte mit ein Grund für die Misserfolge
von Lewis Fry Richardson in seinen Pionierversuchen zur numerischen
Wettervorhersage gewesen sein
\cite{buch:richardson}.
\index{Lewis Fry Richardson}%
\index{Richardson, Lewis Fry}%

\subsubsection{Vorwärtsdifferenzen}
Zur Illustration des Problem verwenden wir die einfache Differentialgleichung
\[
y' = -y,\quad y(0)=y_0
\]
welche die wohlbekannte Lösung $y(x)=y_0e^{-x}$ hat.
Zur Diskretisation verwenden wir die Gitterpunkte $x_i=ih$, $i\in\mathbb Z$
und versuchen die Werte $y_i = y(x_i)$ zu berechnen.
Diskretisation mit Vorwärtsdifferenzen führt auf 
\index{Vorwärtsdifferenz}
\[
y'(x_i) \approx \frac{y_{i+1}-y_i}{h} = -y(x_i) = -y_i
\quad\Rightarrow\qquad
y_{i+1} = y_u-hy_i = (1-h)y_i.
\]
Damit kann man die Lösung als 
\[
y_i = y_0(1-h)^i
\]
schreiben.
Wählt man $h=x/n$, dann liefert diese Approximation
\[
y(x) = y_0\biggl(1-\frac{x}n\biggr)^n 
\to
y_0e^{-x}
\]
für $n\to\infty$.
Dieses Verfahren konvergiert also, wenn auch nicht besonders schnell,
da das Euler-Verfahren nur ein Verfahren erster Ordnung ist.
\index{Euler-Verfahren}%

\subsubsection{Symmetrische Differenzen}
Eine Schwierigkeit der Verwendung von Vorwärts-Differenzen ist, dass
die Vorwärts-Differenz eher eine Approximation für $y'(x_i+h/2)$ ist,
nicht für $y'(x_i)$.
Wie früher gezeigt wurde, können symmetrische Differenzen die Ableitung
im Punkt $x_i$ besser darstellen.
Die zugehörige Differenzengleichung wird jetzt
\[
y'(x_i)
=
\frac{y_{i+1}-y_{i-1}}{2h}
=
-y(x_i)
=
-y_i
\qquad\Rightarrow\qquad
y_{i+1}+2hy_i-y_{i-1}=0.
\]
Ein Potenzansatz $x_i=\lambda^i$ liefert die charakteristische Gleichung
\[
\lambda^{i+1} +2h\lambda^i -\lambda^{i-1}
=
\lambda^{i-1}(\lambda^2 + 2h\lambda -1)
=
0
\]
mit den Lösungen
\[
\lambda_\pm = -h \pm \sqrt{h^2+1}.
\]
Die Quadratwurzel kann in die Taylor-Reihe
\index{Taylor-Reihe}%
\index{Quadratwurzel}%
\[
\sqrt{1+h^2}
\approx
1 + \frac{h^2}2 + O(h^4)
\]
entwickelt werden.
Es folgt, dass $\lambda$-Wert für das positive Vorzeichen
\[
\lambda_+
=
-h+\sqrt{h^2+1}
\approx
1-h+\frac{h^2}{2}+\dots
<
1
\]
kleiner als $1$ ist, insbesondere ist die Folge $\lambda_+^i$ monoton fallend.

Verwenden wir wieder die Schrittweite $h=x/n$, dann ist
\[
y_i
=
y_0 \lambda_+^i
=
y_0 \biggl(1-h+\frac{h^2}2\biggr)^n
=
y_0\biggl(1-\frac{x}n + o(h)\biggr)^n
\to
y_0e^{-x}
\]
für $n\to\infty$, die Lösung für positives Vorzeichen liefert also genau
die gleiche Lösung, die wir bereits mit Vorwärtsdifferenzen gefunden haben.

Die zweite Lösung der charakteristischen Gleichung mit dem negativen
Vorzeichen ist
\[
\lambda_-
=
-\frac{h}2 - \sqrt{\frac{h^2}{4}+1}
<
-1.
\]
Die Lösungsfolge $\lambda_-^i$ hat alternierende Vorzeichen, aber die 
Beträge $|\lambda_-^i|$ wachsen exponentiell schnell an.
Die Verwendung symmetrischer Differenzen führt also dazu, dass die
Differenzengleichung eine zweite Lösung bekommt, die exponentiell
schnell anwächst.
Diese zweite Lösung heisst der {\em Computational Mode}.
\index{Computational Mode}

\subsubsection{Numerische Anregung des Computational Mode}
\index{Computational Mode!numerische Anregung}%
Man muss sich fragen, warum man sich über die zweite Lösung überhaupt
Gedanken machen soll, immerhin wird ja für die Lösung der
Differentialgleichung nur die Lösung $\lambda_+$ benötigt.
Der Grund ist, dass bei der Berechnung der Lösung mit Hilfe der
Rekursionsformel
\[
x_{i+1} = x_{i-1}-2hx_i
\]
die zweite Lösung trotzdem auftritt.
Zu zwei aufeinanderfolgenden Werten $y_0$ und $y_1$ kann man mit der
Rekursionsformel alle weiteren Werte berechnen.
Aber selbst wenn man $y_1 = \lambda+y_0$ setzt, wird die numerische
Rechnung Rundungsfehler einführen, so dass der Wert $y_1$, der für die
Rekursion verwendet wird, nicht mit dem exakten Wert $\lambda_+y_0$
übereinstimmt.
\index{Rekursion}%
Wir nehmen an, dass dieser Rundungsfehler $\delta$ der einzige
Fehler ist, der im Laufe der Berechnung eingeführt wird.
\index{Rundungsfehler}%
Wir müssen also die exakte Lösung für die Startwerte $y_0$ und
$\lambda_+y_0+\delta$ bestimmen.
Es sind also Koeffizienten $a_\pm$ zu finden mit
\begin{align*}
a_+ + a_- &= y_0 \\
a_+\lambda_+ + a_-\lambda_-&=y_0\lambda_++\delta
\end{align*}
Subtrahiert man das $\lambda_+$-fache der ersten Gleichung von der
zweiten Gleichung, erhält man
\[
a_-(\lambda_-\lambda_+)=\delta
\qquad\Rightarrow\qquad
a_- = \frac{\delta}{\lambda_--\lambda_+} = -\frac{\delta}{h}.
\]
Es folgt, dass selbst ein einziger minimaler Fehler $\delta$
bei der Berechnung von $y_1$ die Lösung die zweite Lösung sichtbar
macht. 

Man kann auch ausrechnen, wie gross $i$ werden muss, bis der
Computational Mode von der gleichen Grössenordnung wie die
gesuchte Lösung geworden ist.
Dieser Fall tritt ein wenn
\[
|y_0\lambda_+^i| < |\delta\lambda_-^i|
\qquad\Rightarrow\qquad
\biggl|
\frac{\lambda_-}{\lambda_+}
\biggr|^i
> \biggl|\frac{y_0}{\delta}\biggr|
\quad\Rightarrow\quad
i
>
\frac{\log |y_0/\delta|}{\log|\lambda_-/\lambda_+|}.
\]
Für kleine Werte von $h$ ist
\[
\log |\lambda_\pm|
=
\log(1\mp h+o(h)) 
\approx
\mp h.
\]
Die Bedingung an $i$ kann man damit mit
\[
i \gtrsim \frac{\log |y_0/\delta|}{2h}
\]
approximieren.
Der $x$-Wert, bei dem der Computational Mode überhand nimmt, ist daher
\[
x \gtrsim h \frac{\log|y_0/\delta|}{2h} = \frac12(\log|y_0|-\log|\delta|)
\]
Für den \texttt{double}-Typ ist $|\delta| \approx 10^{-15}$,
das Verfahren ist also grundsätzlich nicht in der Lage, vernünftige
Resultate für $x>7.5\log 10\approx 17.269$ zu liefern.

\begin{figure}
\centering
\includegraphics{chapters/70-pde/experiments/computationalmode.pdf}
\caption{Betrag der Lösung der Differentialgleichung $y'=-y$ berechnet
mit Hilfe symmetrischer Differenzen.
\label{buch:pde:cm:fig}}
\end{figure}
Die Annahme, der einzige Fehler trete bei der Rundung von $y_1$ auf,
ist natürlich viel zu optimistisch.
In Wahrheit tritt in jedem Schritt ein Fehler auf, so dass der 
Computational Mode schon viel früher angeregt wird.
In Abbildung~\ref{buch:pde:cm:fig} ist der Betrag der Lösung
für verschiedene Schrittweiten $h$ gezeigt.
Ganz zu Beginn, für $x<1$, scheint die Lösung einigermassen genau dem
theoretisch zu erwartenden Verlauf zu entsprechen.
Dann beginnt jedoch der Computational Mode überhand zu nehmen,
die Lösung wächst exponentiell schnell an und wird unbrauchbar.
\index{exponentiell}%

Mehr Information zum Computational Mode insbesondere auch zu einer
Filtertechnik, mit der der Computational Mode auch wieder gedämpft
werden kann, ist im Kapitel~\ref{chapter:burgers} zu finden.
\index{Burgers}%
\index{Filter}%








%
% fvm.tex
%
% (c) 2020 Prof Dr Andreas Müller, Hochschule Rapperswi
%
\section{Finite Volumina
\label{section:finite-volumina}}
\rhead{Finite Volumina}
\index{finite Volumina}%
Die Diskretisierung mit Hilfe eines Gitters hat auf Variablen $u_{ik}$
geführt, die Werte der Funktion $u$ an den Gitterpunkten waren.
\index{Gitterpunkte}%
Die Werte der Funktion zwischen diesen Punkten haben für die
diskretisierten Gleichungen keine Rolle gespielt.
Ausserdem hat sich gezeigt, dass Gleichungen erster Ordnung
nur mit Kompromissen auf diese Weise diskretisiert werden.

Am Beispiel der zweidimensionalen Kontinuitätsgleichung soll
in diesem Abschnitt illustriert werden, wie die gleiche Information,
die in der Differentialgleichung steckt, auch in einer Integralform
dargestellt werden kann.
\index{Kontinuitätsgleichung}%
Dies führt auf eine alternative Diskretisation, in der wir nicht
auf Differenzenquotienten angewiesen sind.
\index{Differenzenquotient}%

\subsection{Die Kontinuitätsgleichung}
Die Kontinuitätsgleichung beschreibt die Tatsache, dass in einem
strömenden Fluid Materie nicht einfach aus dem Nichts entstehen kann
oder verschwinden kann.
\index{Fluid}%
\index{Materie}%
Sie stellt eine Beziehung her zwischen der Dichte $\varrho$ und der
Strömungsgeschwindigkeit $v$.
\index{Strömungsgeschwindigkeit}%
Der Einfachheit halber untersuchen wir das Problem nur in einer
Dimension.
Die Dichte $\varrho(x,t)$ wie auch die Geschwindigkeit $v(x,t)$
ist eine Funktion von Ort und Zeit.
\index{Geschwindigkeit}%

Zur Herleitung der Kontinuitätsgleichung betrachten wir ein
Intervall $[x,x+\Delta x]$.
Zur Zeit $t$ enthält es ungefähr die Masse $\varrho(x,t)\cdot\Delta x$.
Die Masse kann sich während eines Zeitintervalls $\Delta t$ dadurch
ändern, dass Materie durch das linke und rechte Intervallende strömt.
Die Menge, die durch das rechte Ende strömt ist
$\varrho(x+\Delta x,t)\cdot v(x+\Delta x,t)\cdot \Delta t$,
durch das linke Ende strömt
$\varrho(x,t)\cdot v(x,t)\cdot \Delta t$.
Die Masseänderung ist die Differenz, also
\[
\varrho(x+\Delta x,t)\cdot v(x+\Delta x,t)\cdot \Delta t
-
\varrho(x,t)\cdot v(x,t)\cdot \Delta t
=
\varrho(x,t+\Delta t)\Delta x
\varrho(x,t)\Delta x.
\]
Nach Division durch $\Delta t\Delta x $ bleibt die Gleichung
\[
\frac{\varrho(x+\Delta x,t)v(x+\Delta x,t) - \varrho(x,t)v(x,t)}{\Delta x}
=
\frac{\varrho(x,t+\Delta t)-\varrho(x,t)}{\Delta t}.
\]
Im Grenzwert $\Delta t\to 0$ un d$\Delta x\to 0$ entsteht die
die partielle Differentialgleichung
\begin{equation}
\frac{\partial (\varrho v)}{\partial x} = \frac{\partial \varrho}{\partial t}
\qquad\Leftrightarrow\qquad
\frac{\partial\varrho}{\partial t} - \frac{\partial j}{\partial x}=0,
\label{pde:eqn:kontinuitaetsgleichung}
\end{equation}
wobei wir in der letzten Umformung $j=\varrho v$ für den Massefluss
geschrieben haben.
Die Gleichung \eqref{pde:eqn:kontinuitaetsgleichung} heisst die
{\em Kontinuitätsgleichung}.
\index{Kontinuitätsgleichung}%

Die Kontinuitätsgleichung kann ganz analog auch in höheren Dimensionen
hergeleitet werden.
Für den Stromvektor $\vec{\jmath}=\varrho\vec{v}$ gilt
\[
\frac{\partial \varrho}{\partial t}
-
\operatorname{div}(\varrho \vec{v})
=
\frac{\partial \varrho}{\partial t}
-
\operatorname{div}\vec{\jmath}
=
0,
\]
wobei die {\em Divergenz}
\index{Divergenz}%
eines Vektorfeldes $\vec{a}$ durch
\[
\operatorname{div}\vec{a}
=
\frac{\partial a_x}{\partial x}
+
\frac{\partial a_y}{\partial y}
+
\frac{\partial a_z}{\partial z}
\]
gegeben ist.

\subsection{Integralform}
\index{Integralform}
Die Herleitung der Kontinuitätsgleichung basiert auf dem Vergleich
von Funktionswerten von $\varrho$ und $j$ in den Eckpunkten eines
Rechtecks mit Kantenlänge $\Delta x$ und $\Delta t$ in der $x$-$t$-Ebene.
Alternativ hätten wir aber auch Integrale entlang der Kanten verwenden
können, um die Veränderung der Masse im Intervall $[x,x+\Delta x]$ 
zu berechnen.

Die Masse im Intervall $[x,x+\Delta x]$ ist gegeben durch das Integral
\[
m(t) = \int_x^{x+\Delta x} \varrho(\xi, t) \,d\xi
\]
der Dichte über das Interval.
\index{Dichte}%
Die Änderung der Masse zwischen den Zeiten $t$ und $t+\Delta t$ entsteht
durch den Fluss durch die Endpunkte.
\index{Masse}%
Durch den linken und rechten Endpunkt des Intervalls fliesst im Zeitintervall
$[t,t+\Delta t]$ die Masse
\[
\int_t^{t+\Delta t} j(x,\tau)\,d\tau
\qquad\text{bzw.}\qquad
\int_t^{t+\Delta t} j(x+\Delta x,\tau)\,d\tau.
\]
Die Änderung der Masse zwischen den Zeitpunkten $t$ und $t+\Delta t$
ist die Differenz, also
\begin{align}
m(t+\Delta t)-m(t)
&=
\int_t^{t+\Delta t}j(x+\Delta x,\tau)\,d\tau
-
\int_t^{t+\Delta t}j(x,\tau)\,d\tau
\notag
\\
\Rightarrow\qquad
\int_x^{x+\Delta x} \varrho(\xi,t+\Delta t)\,d\xi
-
\int_x^{x+\Delta x} \varrho(\xi,t)\,d\xi
&=
\int_t^{t+\Delta t}j(x+\Delta x,\tau)\,d\tau
-
\int_t^{t+\Delta t}j(x,\tau)\,d\tau
\label{pde:eqn:integralform}
\end{align}
Diese Gleichung gilt für beliebig grosse Schritte $\Delta x$ und
$\Delta t$, sie ist also nicht nur eine Approximation wie die
Differenzenquotienten, die für die Herleitung der Kontinuitätsgleichung
verwendet wurden. 
Die Differenzenquotienten ergaben erst im Grenzwert eine exakte
Gleichung.
Wir nennen 
\eqref{pde:eqn:integralform}
die {\em Integralform der Kontinuitätsgleichung}.

Die Herleitung der Integralform lässt sich auch in der mehrdimensionalen
Situation durchführen.
Dazu muss man einerseits die Masse $m(t)$ in einem Volumn $V$ berechnen,
was mit dem Dreifachintegral
\[
m(t)
=
\iiint_V \varrho(x,y,z,t) \,dV
\]
geschehen kann.
Andererseits muss man den Materiefluss durch die Oberfläche von $V$
berechnen können, was das Flussintegral
\[
\oint_{\partial V} \vec{\jmath}(x,y,z,t) \,d\vec{n}
\]
tut.
\index{Materiefluss}%
\index{Flussintegral}%
Die Integralform der Kontinuitätsgleichung wird damit
\[
\iiint_V \varrho(x,y,z,t+\Delta t) \,dV
-
\iiint_V \varrho(x,y,z,t) \,dV
=
\oint_{\partial V} \vec{\jmath}(x,y,z,t+\Delta t)\,d\vec{n}
-
\oint_{\partial V} \vec{\jmath}(x,y,z,t)\,d\vec{n}.
\]

Die Integralsätze von Green, Gauss und Stokes ermöglichen ganz allgemein,
viele der Differentialgleichungen, die Naturphänomene beschreiben, in
eine Integralform zu bringen.
\index{Gauss!Divergenzsatz}%
\index{Divergenzsatz von Gauss}%
\index{Stokes!Integralsatz}%
\index{Satz von Stokes}%
\index{Green!Satz von}%
\index{Satz von Green}

\subsection{Diskretisation}
\begin{figure}
\centering
\includegraphics{chapters/70-pde/images/kont.pdf}
\caption{Diskretisation der Kontinuitätsgleichung~\eqref{pde:fvm:1dim}
mit Hilfe diskreter Volumina. 
Die Variablen $m_{ik}$ stehen für die im Intervall $[ih_x,(i+1)h_x]$
enthaltene Masse, $j_{ik}$ steht für die im Zeitintervall $[kh_t,(k+1)h_t]$
durch das Intervallende $ih_x$ fliessende Masse.
\label{buch:pde:fvdisk}}
\end{figure}
Die Integralform~\eqref{pde:eqn:integralform}
der Kontinuitätsgleichung suggeriert, dass die Integrale
in dieser Gleichung bessere Variablen für eine Diskretisation
sein könnten.

\subsubsection{Eine Raumdimension}
Wir verwenden wieder ein Gitter in der $x$-$t$-Ebene, aber 
als gesuchte Variablen verwenden wir die Integrale
\begin{align*}
m_{ik}
&=
\int_{ih_x}^{(i+1)h_x} \varrho(\xi,kh_t)\,d\xi
\\
j_{ik}
&=
\int_{kh_t}^{(k+1)h_t} j(ih_x,\tau) \,d\tau
\end{align*}
der Funktionen über Kanten im Gitter, wie in Abbildung~\ref{buch:pde:fvdisk}
dargestellt.
Die Kontinuitätsgleichung wird damit zu
\begin{equation}
m_{i,k+1}-m_{ik}
=
j_{i+1,k}-j_{ik}.
\label{pde:fvm:1dim}
\end{equation}
Man beachte auch hier wieder, dass dies keine Approximation ist, sondern
dass diese Gleichungen exakt gelten.

\subsubsection{Höhere Dimension}
Um diese Idee auf höhere Dimensionen zu verallgemeinern zerlegt man
das Gebiet in kleine Volumina $V_i$.
Die Variablen
\[
m_{ik} = \int_{V_i} \varrho(x, kh_t)\, dV
\]
berechnen die Masse, die im Volumen $V_i$ enhalten sind.
Die Masseänderung in einem Zeitintervall setzt sich aus den
Masseflüssen durch die verschiedenen Seitenflächen $V_i$ 
zusammen.
Seien $\sigma_{il}$ die Seitenflächen des Volumens $V_i$
Daher verwenden wir 
\[
j_{ilk} = \int_{\sigma_{il}} \vec{\jmath}\,(\xi, kh_t)\,d\vec{n}.
\]
Für jedes Volumen nimmt die Kontinuitätsgleichung die Form
\begin{equation}
m_{i,k+1}-m_{ik}
=
\sum_{\text{$l$ Seitenfläche von $V_i$}} (j_{il,k+1}-j_{ilk}).
\label{pde:fvm:ndim}
\end{equation}
an.
Wieder sind lineare Gleichungen entstanden.
Haben zwei Volumina $V_{i_1}$ und $V_{i_2}$ die Seite
$\sigma_{i_1j_1}=\sigma_{i_2j_2}$ gemeinsam, dann sind die 
zugehörigen Flüsse entgegengesetzt:
\[
j_{i_1l_1k}=j_{i_2l_2k}
\qquad\forall k\in\mathbb Z,
\]
denn was das Volumen $V_{i_1}$ durch die Seite $\sigma_{i_1l_1}$ an
Masse verliergt gewinnt das Volumn $V_{i_2}$ durch die Seite
$\sigma_{i_2l_2}$.

Die Gleichungen \eqref{pde:fvm:1dim} und \eqref{pde:fvm:ndim} für
sich alleine reichen nicht, weitere Gleichungen wie die
Navier-Stokes-Gleichung werden zusätzlich benötigt, um ein Strömungsfeld
vollständig zu beschreiben.
\index{Navier-Stokes-Gleichung}%
Es ist allerdings nicht das Ziel dieses Abschnitts, die so
begründete Methode
der {\em finiten Volumina} in voller Allgemeinheit zu entwickeln.
\index{finite Volumina}%






%
% fem.tex
%
% (c) 2020 Prof Dr Andreas Müller, Hochschule Rapperswil
%
\section{Finite Elemente
\label{section:finite-elemente}}
\rhead{Finite Elemente}
Die Technik der finiten Volumina basierte darauf, dass die gesuchte
Funktion durch Integrale über Volumina oder Flächenstücke ersetzt 
werden konnte, zwischen denen lineare Gleichungen gelten.
Dies ist jedoch nicht die einzige denkbare Vorgehensweise.
In diesem Abschnitt zeigen wir, wie gewisse partielle Differentialgleichungen
in ein äquivalentes Minimalprinzip umgewandelt werden können.
Approximiert man die gesuchte Funktion anschliessend durch geeignete
Interpolationspolynome, wird das Minimalproblem zu einem quadratischen
Minimalproblem für die Koeffizienten der Interpolationspolynome,
welches mit Methoden der linearen Algebra gelöst werden kann.
\index{Minimalproblem}%

\subsection{Das äquivalente Minimalproblem}
Zur Illustration des Prinzips soll in diesem Abschnitt das Eigenwertproblem
für eine elliptische Differentialgleichung zweiter Ordnung betrachtet werden.

\subsubsection{Ein eindimensionales Problem}
Wir betrachten die Differentialgleichung
\begin{equation}
u''(x) = \lambda u(x)
\label{pde:fem:1dgl}
\end{equation}
auf dem Interval $[a,b]$ mit Randwerten  $u(a)=u(b)=0$
und möchten zeigen, dass eine Lösung gleichzeitig ein stationärer
Punkt des Integrals
\begin{equation}
I(u)
=
\int_a^b u'(x)^2 + \lambda u(x)^2 \,dx
\label{pde:fem:1minimal}
\end{equation}
ist.

\begin{satz}
\label{pde:satz:minimal1}
Eine Funktion $u(x)$ ist genau dann eine Lösung der Differentialgleichung
\eqref{pde:fem:1dgl}, wenn sie das Funktional
$I(u)$ von \eqref{pde:fem:1minimal} minimiert.
\end{satz}
\index{Funktional}%
\index{Variation}%

\begin{proof}[Beweis]
Ein Minimum der Funktion $I(u)$ erfüllt die Bedingung, dass jede 
Variation $u(x)+\varepsilon h(x)$ von $u$, die den Randbedingungen genügt,
einen grösseren Wert von $I$ liefert.
Die Ableitung nach $\varepsilon$ verschwindet also an der Stelle
$\varepsilon=0$.
Wir setzen
\[
I(\varepsilon) = I(u(x) + \varepsilon h(x))
\]
mit beliebigen Funktionen $h(x)$, die am Rand verschwinden: $h(a)=h(b)=0$.
\begin{align}
0
=
\frac{dI(\varepsilon)}{d\varepsilon}\bigg|_{\varepsilon=0}
&=
\frac{d}{d\varepsilon}
\int_a^b (u'(x)+\varepsilon h'(x))^2  + (u(x)+\varepsilon h(x))^2\,dx
\bigg|_{\varepsilon=0}
\notag
\\
&=
\int_a^b
2u'(x)h'(x) + 2\varepsilon h'(x)^2
+
2\lambda u(x)h(x) + 2\lambda \varepsilon h(x)^2
\,dx
\bigg|_{\varepsilon=0}
\notag
\\
&=
2
\int_a^b
u'(x)h'(x)
+
\lambda u(x)h(x)
\,dx
\notag
\intertext{Um das Integralprinzip von Lemma~\ref{buch:lemma:integralprinzip}
anwenden zu können, darf nur $h(x)$ vorkommen.
\index{Integralprinzip}%
Wir können die Ableitung $h'(x)$ mit Hilfe von partieller Ableitung zum
Verschwinden bringen.}
&=
2\biggl[u'(x)h(x)\biggr]_a^b
-
2\int_a^b u''(x) h(x)\,dx
+
2\int_a^b \lambda u(x) h(x)\,dx.
\notag
\intertext{Der erste Term verschwindet, da $h(x)$ an den Intervallenden
verschwindet:}
&=
2\int_a^b \bigl(-u''(x) +\lambda u(x)\bigr)\,h(x)\,dx.
\label{pde:fem:1integral}
\end{align}
Jetzt kann Lemma~\ref{buch:lemma:integralprinzip} angewendet werden: das
Integral~\eqref{pde:fem:1integral} kann nur dann für alle Funktionen
$h(x)$ verschwinden, wenn
\[
u''(x)=-\lambda u(x)
\]
gilt.
\end{proof}


\subsubsection{Partielle Differentialgleichung auf einem Rechteck}
Das gleiche Prinzip ist auch anwendbar für das Eigenwertproblem
\begin{equation}
\Delta u(x) = \lambda u(x),
\qquad \Delta
=
\frac{\partial^2}{\partial x^2} + \frac{\partial^2}{\partial y^2}
\label{pde:fem:2dgl}
\end{equation}
auf einem Rechteck $\Omega = (a,b)\times (c,d)$, es ist nur nicht
ganz so klar, wie das Problem formuliert werden muss.
Der zentrale Schnitt im Beweis von Satz~\ref{pde:satz:minimal1}
war partielle Integration und die Ausnutzung der Randwerte von $h(x)$.
Indem wir dieses Beispiel auf ein Rechteck verallgemeinern, können
wird die richtige Verallgemeinerung von Satz~\ref{pde:satz:minimal1}
finden.

Da die Funktion von zwei Variablen abhängt, gibt es jetzt nicht nur
eine erste Ableitung sondern deren zwei.
Statt dem Quadrat der ersten Ableitung $u'(x)^2$ werden daher
einen analogen Terme für beide ersten Ableitungen benötigen.
\index{Quadrant}%
\index{Quadratsumme}%
Die Quadratsumme der Ableitungen
\[
(\nabla u)^2
=
\biggl(\frac{\partial u}{\partial x}\biggr)^2
+
\biggl(\frac{\partial u}{\partial y}\biggr)^2
\]
liegt auf der Hand.
In Analogie zum eindimensionalen Problem verwenden wir daher
als Minimalproblem das Funktional
\index{Funktional}%
\[
I(u)
=
\int_a^b\int_c^d \nabla u(x)^2 + \lambda u(x)^2\,dy \,dx
\]
und schreiben wieder
\[
I(\varepsilon)
=
I(u + \varepsilon h)
\]
für eine Funktion $h\colon \Omega\to\mathbb R$, die auf dem Rand
verschwindet: also
\[
u(a,y) = u(b,y) = u(x,c) = u(x,d) = 0
\]
für beliebige $x\in[a,b]$ und $y\in[c,d]$.

Die Ableitung nach $\varepsilon$ an der Stelle $\varepsilon=0$ ist
\begin{align*}
\frac{dI(\varepsilon)}{d\varepsilon}\bigg|_{\varepsilon=0}
&=
\frac{d}{d\varepsilon}
\int_a^b\int_c^d
(\nabla u(x)+\varepsilon \nabla h(x))^2
+
\lambda (u(x) + \varepsilon h(x))^2
\,dy \,dx
\bigg|_{\varepsilon=0}
\\
&=
2
\int_a^b\int_c^d
\nabla u(x)\cdot \nabla h(x) +\varepsilon \nabla h(x)^2
+
\lambda u(x) h(x) + \lambda \varepsilon h(x)^2
\,dy \,dx
\bigg|_{\varepsilon=0}
\\
&=
2
\int_a^b\int_c^d
\nabla u(x)\cdot \nabla h(x) + \lambda u(x) h(x)
\,dy \,dx.
\end{align*}

Das erste Term im Integranden enthält wieder Ableitungen der Funktion
$h$, die wir loswerden müssen.

\begin{lemma}
\label{pde:lemma:partint2}
Ist $v(x,y)$ eine beliebige Funktion $v\colon\Omega\to\mathbb R^2$ und
$h\colon\Omega\to\mathbb R$. 
Dann gilt für das Integral von $v\cdot\nabla u$ die partielle
Integrationsformel
\index{partielle Integration}
\index{Integration!partiell}
\begin{align}
\int_a^b\int_c^d v(x,y)\cdot \nabla h(x,y)\,dy\,dx
&=
\int_c^d
v_x(b,y) h(b,y)
-
v_x(a,y) h(a,y)
\,dy
\notag
\\
&\qquad
+
\int_a^b
v_y(x,d) h(x,d)
-
v_y(x,c) h(x,c)
\,dx
\notag
\\
&\qquad
-\int_a^b\int_c^d
\frac{\partial v_x}{\partial x}+\frac{\partial v_y}{\partial y}
\,dy\,dx
\label{pde:fem:part2d}
\end{align}
\end{lemma}

\begin{proof}[Beweis]
\definecolor{orange}{rgb}{1,0.6,0.2}
\definecolor{gruen}{rgb}{0.2,0.6,0.2}
\definecolor{magenta}{rgb}{0.9,0.2,0.4}
\definecolor{azure}{rgb}{0,0.2,1}
Wir teilen das Integral mit Hilfe von
\[
v(x,y)\cdot\nabla h(x,y) = 
v_x(x,y) \, \frac{\partial h}{\partial x}(x,y)
+
v_y(x,y) \, \frac{\partial h}{\partial x}(x,y)
\]
in zwei Summanden
\[
\int_a^b\int_c^dv\cdot\nabla h \,dy\,dx
=
\int_a^b\int_c^dv_x \frac{\partial h}{\partial x} \,dy\,dx
+
\int_a^b\int_c^dv_y \frac{\partial h}{\partial y} \,dy\,dx
=
I_1+I_2
\]
auf, die wir separat berechnen können.
Für den ersten Summanden erhalten wir
\begin{align}
I_1
&=
\int_a^b\int_c^d v_x(x,y) \frac{\partial h}{\partial x}(x,y)\,dy\,dx
\notag
\\
&=
\int_c^d
\int_a^b
v_x(x,y) \frac{\partial h}{\partial x}(x,y)
\,dx
\,dy
\notag
\\
&=
\int_c^d
\biggl[
v_x(x,y) h(x,y)
\biggr]_a^b
-
\int_a^b \frac{\partial v_x}{\partial x}(x,y) h(x,y)
\,dx
\,dy
\notag
\\
&=
\int_c^d
v_x(b,y) h(b,y)
-
v_x(a,y) h(a,y)
-
\int_a^b \frac{\partial v_x}{\partial x}(x,y) h(x,y)
\,dx
\,dy
\notag
\\
&=
\int_c^d
{\color{magenta}
v_x(b,y) h(b,y)}
{\color{azure}
\mathstrut
-
v_x(a,y) h(a,y)}
\,dy
-
\int_a^b
\int_c^d
\frac{\partial v_x}{\partial x}(x,y) h(x,y)
\,dy
\,dx.
\label{buch:pde:part2dI1}
\intertext{%
Die zweite Summe ist noch einfacher, weil es gar nicht erst notwendig ist,
die Integrationsreihenfolge zu ändern:
\index{Integrationsreihenfolge}}
I_2
&=
\int_a^b\int_c^d v_y(x,y)\frac{\partial h}{\partial y}\,dy\,dx
\notag
\\
&=
\int_a^b
\biggl[ v_y(x,y) h(x,y)\biggr]_c^d
-
\int_c^d
\frac{\partial v_y}{\partial y} h(x,y)
\,dy
\,dx
\notag
\\
&=
\int_a^b
{\color{gruen}
v_y(x,d)h(x,d)}
{\color{orange}\mathstrut
-v_y(x,c)h(x,c)}
\,dx
-
\int_a^b
\int_c^d
\frac{\partial v_y}{\partial y} h(x,y)
\,dy
\,dx
\label{buch:pde:part2dI2}
\end{align}
Die beiden Terme zusammen geben genau die im Lemma behauptete Formel.
\end{proof}

\begin{figure}
\centering
\includegraphics{chapters/70-pde/images/2dpart.pdf}
\caption{Die Randterme des Integrals \eqref{pde:fem:part2d}  können
als ein Flussintegral über den Rand des Rechtecks geschrieben werden.
\index{Flussintegral}%
Dazu müssen die Integrale über die einzelnen Kanten des Rechtecks einzeln
als Flussintegrale über die Kante geschrieben werden.
Die Terme werden auch in den Gleichungen
\eqref{buch:pde:part2dI1} und \eqref{buch:pde:part2dI2}
mit den gleichen Farben hervorgehoben.
\label{buch:pde:pfadintegral}}
\end{figure}
Der erste Integral auf der rechten Seite von Lemma~\ref{pde:lemma:partint2}
kombiniert Integrale von $v_x(x,y) h(x,y)$ über die beiden vertikalen Kanten 
des Rechtecks, während das zweite Integral die Integrale von
$v_y(x,y)h(x,y)$ über die horizontalen Kanten kombiniert (siehe auch
Abbildung~\ref{buch:pde:pfadintegral}).
Schreibt man $\vec{n}(x,y)$ für die nach aussen zeigende Normale auf den Rand 
des Rechtecks, dann können diese Integrale alle einheitlich geschrieben
werden:
\begin{center}
\definecolor{orange}{rgb}{1,0.6,0.2}
\definecolor{gruen}{rgb}{0.2,0.6,0.2}
\definecolor{magenta}{rgb}{0.9,0.2,0.4}
\definecolor{azure}{rgb}{0,0.2,1}
\begin{tabular}{l >{$}c<{$} >{$}l<{$}}
Kante& &\text{Integrand}
\\[2pt]
\hline
\\[-7pt]
\color{orange}untere Kante
	&y=c
	&         - v_x(x,c)h(x,c) = h(x,c) \, \vec{v}(x,c) \cdot \vec{n}(x,c)
\\[4pt]
\color{gruen}obere Kante
	&y=d
	&\phantom{-}v_x(x,d)h(x,d) = h(x,d) \, \vec{v}(x,d) \cdot \vec{n}(x,d)
\\[4pt]
\color{azure}linke Kante
	&x=a
	&         - v_y(a,y)h(a,y) = h(a,y) \, \vec{v}(a,y) \cdot \vec{n}(a,y)
\\[4pt]
\color{magenta}rechte Kante
	&x=b
 	&\phantom{-}v_y(b,y)h(b,y) = h(b,y) \, \vec{v}(b,y) \cdot \vec{n}(b,y)
\\[4pt]
\hline
\end{tabular}
\end{center}
Es folgt also, dass die einfachen Integrale in 
Lemma~\ref{pde:lemma:partint2} das Integral
\[
\int_{\partial\Omega} \vec{v}(x,y) h(x,y) \cdot d\vec{n}
\]
ist.

Nach diesen Vorarbeiten können wir jetzt das zur Differentialgleichung
\eqref{pde:fem:2dgl} gehörige äquivalente Minimalprinzip formulieren.

\begin{satz}
Die Funktion $u\colon\Omega\to\mathbb R$ ist genau dann eine Lösung der
Differentialgleichung
\[
\Delta u =\lambda u
\]
wenn sie das Funktional
\[
I(u)
=
\int_{\Omega} \nabla u(x,y)^2 + \lambda u(x,y)^2 \,dx\,dy
\]
minimiert.
\index{Funktional}%
\end{satz}

\begin{proof}[Beweis]
Die Ableitung von $I(\varepsilon)$ an der Stelle $\varepsilon=0$
wurde früher schon berechnet, das Integral~\eqref{pde:fem:part2d}
muss jetzt mit Hilfe der Formel für die partielle Integration
von Lemma~\ref{pde:lemma:partint2} umgeformt werden.
Dazu setzen wir $\vec{v}=\nabla u$ und erhalten für
\[
\frac{\partial v_x}{\partial x}
+
\frac{\partial v_y}{\partial y}
=
\frac{\partial^2 u}{\partial x^2}
+
\frac{\partial^2 u}{\partial y^2}
=
\Delta u.
\]
Es folgt
\begin{align*}
0
&=
\int_{\partial\Omega} \nabla u(x,y) h(x,y) \cdot d\vec{n}
-
\int_{\Omega} \Delta u(x,y) h(x,y)\,dx\,dy
+
\int_{\Omega} \lambda u(x,y) h(x,y)\,dx\,dy
\\
&=
-
\int_{\Omega} \bigl(\Delta u(x,y)-\lambda u(x,y)\bigr) h(x,y)\,dx\,dy.
\end{align*}
Dies gilt genau dann für jede Funktion $h$, wenn
\[
\Delta u = \lambda u,
\]
wenn also $u$ eine Lösung des Eigenwertproblems ist.
\index{Eigenwertproblem}%
\end{proof}

%
% loesung.tex -- Beispiel-File für die Beschreibung der Loesung
%
% (c) 2020 Prof Dr Andreas Müller, Hochschule Rapperswil
%
\section{Approximation
\label{kettenbruch:section:Approximation}}
\rhead{Approximation}

In der Einleitung wurde erwähnt, dass die Bestimmung von guten
Näherungsbrüchen eine wichtige Anwendung von Kettenbrüchen ist. Es
gilt nämlich, dass jeder Näherungsbruch der Kettenbruchentwicklung
einer reellen Zahl eine besonders gute rationale Näherung dieser
Zahl ist.

\subsection{Definition}

Eine rationale Zahl $\frac{a}{b}$ mit $b>0$ heisst beste Näherung
erster Art an eine reelle Zahl $x$, wenn es keine von $\frac{a}{b}$
verschiedene rationale Zahl mit gleichem oder kleinerem Nenner gibt,
die bezüglich des Absolutbetrages näher bei $x$ liegt.
Das heisst, dann gilt für alle rationalen Zahlen $\frac{c}{d} \ne
\frac{a}{b}$ mit $0<d\le b$:
\begin{equation}
\biggl|x-\frac{a}{b}\biggr| < \biggl| x-\frac{c}{d}\biggr|.
\end{equation}

\subsection{Näherungsgesetz}
Ziel dieses Abschnitt ist es, eine genügend gute Approximation der
Näherungsbrüche nachzuweisen. Gibt man sich eine beliebige Zahl $x$
vor, so kann man sich die Frage stellen, welche "unkürzbaren" Brüche
$\frac{p}{q}$ mit vorgegebenem Höchstnenner sich gut approximieren
lässt.

\begin{beispiel}
Näherung von $\pi$ mit dem unendliche Dezimalbruch:
$\pi = [3;7,15,1,292,1,1,1,2,1,3,1,14,2,\cdots]$
Die Näherung $3.14 = \frac{314}{100}$ ist eine Näherung. Aber
$\frac{22}{7} = 3.14285714\dots$ hat einen viel kleineren Nenner und
ist eine deutlich bessere Näherung von $\pi$.
Eine noch bessere Näherung ist der Kettenbruch
\begin{equation}
\frac{355}{113} = 3 + \cfrac{1}{7+\cfrac{1}{15+\cfrac{1}{1}}} = 3.1415\overline{92}
\end{equation}
Folgende Näherungswerte von $\pi$ können schnell und einfach gerechnet werden:
\begin{equation}
3,\frac{22}{7} \approx 3.143 ; \frac{333}{106} \approx 3.14151 ; \frac{355}{113} 
\approx 3.1415929 ; \frac{103993}{33102} \approx 3.1415926530 ; \cdots.
\end{equation}
\end{beispiel}
Die Bestapproximation ist einfach formuliert durch die Bestimmung
derjenigen rationalen Brüchen, welche von einer gegebenen rationalen
oder irrationalen Zahl einen festgelegten minimalen Abstand haben
und dabei einen möglichst kleinen positiven Nenner besitzen.

\subsubsection{Beispiel}
\begin{beispiel}
Die Funktion $\tan^{-1}(x)$ spielt bei der Berechnung von $\pi$ an vielen Stellen eine Rolle. 
Es gilt die Leibnizsche Reihe
\begin{equation}
\frac{\pi}{4} = \tan^{-1}(1)
=
1 - \frac{1}{3} + \frac{1}{5} - \frac{1}{7} + \frac{1}{9} - \frac{1}{11} +\dots
\end{equation}
Dank der Euler-Transformation gilt folgender Kettenbruch
\begin{equation}
\tan^{-1}(x) = x - \frac{x^3}{3} + \frac{x^5}{5} - \frac{x^7}{7} + \frac{x^9}{9} - 
\frac{x^{11}}{11}\dots,		\qquad x \leq 1
\end{equation}
Somit kann der Kettenbruch von $\tan^{-1}(x)$ folgendermassen dargestellt werden.
\begin{equation}
\tan^{-1}(x)
=
\cfrac{x}{1+\cfrac{x^2}{3+\cfrac{4x^2}{5+\cfrac{9x^2}{7+\cfrac{16x^2}{9+\cdots}}}}}
\qquad	(|x|< 1)
\end{equation}
Das Gleichungssystem kann umgeschrieben werden als Funktion $f_n$
\begin{equation}
f_n(x) = \frac{x}{1+}\frac{x^2}{3+}\frac{4x^2}{5+}\cdots\frac{(n-1)^2 x^2}{2n-1}
\qquad	(|n|\ge 2)
\end{equation}
\end{beispiel}
Hiermit kann nach $n$-te Bildung der Kettenbruchreaktion einen Grenzwert
erreichen:
\begin{equation}
\tan^{-1}(x) = \lim_{n\to\infty} f_n(x), \qquad (|x| < 1)
\end{equation}
Die Konvergenz der Funktion kann an einem Beispiel beurteilt werden. 
\begin{equation}
\tan^{-1}(1) = \pi/4 \approx 0.785398
\end{equation}

\begin{table}
\centering
\begin{tabular}{>{$}c<{$}>{$}l<{$}}
n	& f_n(1) 	\\
\hline
2	& 0.750000 	\\
3	& 0.791667 	\\
4	& 0.784314 	\\
5	& 0.785586 	\\
6	& 0.785366 	\\
7	& 0.785404	\\
8	& 0.785397	\\
9	& 0.785398	\\
\hline
\end{tabular}
\caption{Näherungsstufen mit Kettenbruchentwicklung von der Funktion $\tan^{-1}(1)$
\label{kettenbruch:tabelle}}
\end{table}

In wenigen und einfachen Schritten haben wir mit Hilfe einer
Kettenbruchentwicklung ein System gebildet das die Konvergenz der
Funktion $\tan^{-1}(x)$ vorantreibt und präzise Resultate liefert.


%
% quadratisch.tex
%
% (c) 2020 Prof Dr Andreas Müller, Hochschule Rapperswil
%
\subsection{Quadratische Minimalprobleme
\label{buch:qm:subsection:quadratischeminimalprobleme}}
In den vorangegangenen Beispielen wurde gezeigt, wie sich viele
partielle Differentialgleichungen auf äquivalente Minimalprobleme
zurückführen lassen.
\index{quadratisches Minimalproblem}
Indem man die Funktionen dann als Linearkombinationen approximiert,
kann man daraus quadratische Minimalprobleme für die Koeffizienten
dieser Linearkombinationen gewinnen.
In diesem Abschnitt soll gezeigt werden, wie solche quadratischen
Minimalprobleme algebraisch gelöst werden können.

%
% Least-Squares-Problem als Beispiel
%
\subsubsection{Least-Squares-Problem}
\index{Least-Squares-Problem}%
Das folgende Problem ist ein wohlbekanntes Beispiel für eine einfaches
quadratisches Minimalproblem.
Es soll hier als Einstieg dienen, Ziel ist eine allgemeinere Form von
Minimalproblem zu formulieren und die Analysis hinzuzuziehen, um das
verallgemeinerte Problem zu lösen.
Natürlich springt dabei auch ein neuer Beweis für die bekannte 
Lösung des Least-Squares-Problems heraus.

\begin{problem}
\label{buch:qm:problem:ls}
Sei $A$ eine $n\times m$-Matrix mit $m<n$ und $\operatorname{Rang} A=m$
und $b$ ein $n$-dimensionaler Spaltenvektor.
Finde einen $m$-dimensionalen Vektor $x$ derart, dass $|Ax-b|$ minimal ist.
\end{problem}

Die geometrische Intuition zur Lösung des Problems ist, dass 
$Ax$ die Punkte eines Unterraums sind, der von den Spalten von $A$
aufgespannt wird.
\index{Unterraum}%
Gesucht wird jetzt derjenige Punkt, der möglichst nahe am Punkt $b$
liegt, der möglicherweise ausserhalb des Unterraums ist.
Für diesen Punkt $Ax$ muss die Differenz $Ax-b$ auf dem Unterraum
senkrecht stehen.
Da dieser von den Spalten von $A$ aufgespannt wird, muss $Ax-b$ auf allen
Spalten von $A$ senkrecht stehen.
Das Skalarprodukt aller Spalten von $A$ mit $Ax-b$ muss daher verschwinden,
also
\index{Skalarprodukt}%
\[
A^t(Ax-b) = 0
\qquad\Rightarrow\qquad
A^tAx = A^tb 
\qquad\Rightarrow\qquad
x = (A^tA)^{-1}A^tb.
\]
Diese Argumentation verwendet die geometrischen Eigenschaften des
Skalarproduktes.
Und funktioniert daher nur für den euklidischen Abstand, der von
dem sehr speziellen quadratischen Ausdruck $x_1^2+\dots + x_n^2$
berechnet wird.
Im Folgenden soll das Problem für beliebige quadratische Ausdrücke
verallgemeinert werden.

%
% Problemstellung
%
\subsubsection{Problemstellung}
Das Least Squares Problem\ref{buch:qm:problem:ls} ist ein Spezialfall eines
quadratischen Minimalproblems.

\begin{problem}
\label{buch:qm:problem:ls2}
Sei $A$ eine Matrix und $b$ ein Vektor wie in Problem~\ref{buch:qm:problem:ls}.
Finde einen Vektor $x$ derart, dass die quadratische Funktion
\begin{equation}
Q(x)
=
|Ax-b|^2
=
(Ax-b)^t(Ax-b)
=
x^tA^tAx -x^tA^tb - b^tAx +b^t b
\label{buch:qm:eqn:ls2}
\end{equation}
minimiert wird.
\end{problem}

Während es in der ursprünglichen Formulierung des Least-Squares-Problems
um die Minimierung des Abstands ging, ist jetzt ein Problem entstanden,
in der ein allgemeinerer quadratischer Ausdruck der Form $x^tBx$ 
minimiert werden muss, wobei $B=A^tA$ eine symmetrische Matrix ist.
Allerdings ist immer noch die Beschreibung des Unterraums der zulässigen
Punkte $Ax$ vermischt mit dem quadratischen Ausdruck, der minimiert werden
soll, in dem ebenfalls die Matrix $A$ vorkommt.
Um diese beiden Komponenten klarer zu trennen, formulieren wird das
Problem wie folgt:

\begin{problem}
\label{buch:qm:problem:allg}
Sei $B$ eine positiv definite, symmetrische $n\times n$-Matrix 
und $A$ eine $m\times n$-Matrix mit $m<n$ und $\operatorname{Rang}A=m$.
Sei weiter $b$ ein $m$-dimensionaler Vektor.
Finde einen $n$-dimensionalen Vektor $x$ der $Ax=b$ erfüllt und
$Q(x)=x^tbx$ minimiert.
\end{problem}

%
% Nichtlineare Minimalprobleme
%
\subsubsection{Nichtlineare Minimalprobleme}
\index{Nichtlineares Minimalproblem}%
\index{Minimalproblem!nichtlinear}%
In der Theorie der Funktionen mehrere Variablen lernt man die folgende
Methode zur Lösung nichtlinearer Minimalprobleme.
Sie wird oft auch die Methode der Lagrange-Multiplikatoren genannt,
die Zahlen $\lambda_i$ heissen Lagrange-Multiplikatoren.
\index{Lagrange-Multiplikator}%

\begin{lemma}
\label{buch:qm:lemma:lagrangemultiplikatoren}
Gegeben ist eine Funktion $f\colon \mathbb R^n \to \mathbb R$
und $m$-Funktionen $g_i\colon\mathbb R^n \to \mathbb R$ mit $m<n$.
Ist $x$ ein Punkt, in dem $f(x)$ minimiert wird und gleichzeitig
$g_i(x)=0$ gilt, dann gibt es Zahlen $\lambda_i$ derart dass
\[
\nabla f(x) = \lambda_1 \nabla g_1(x)  + \dots + \lambda_m g_m(x)
\]
gilt.
\end{lemma}

Das Lemma~\ref{buch:qm:lemma:lagrangemultiplikatoren} besagt, dass
$x$ und die Zahlen $\lambda_i$ als Lösung der Gleichungen
\begin{equation}
\begin{aligned}
g_i(x)&=0\qquad i=1,\dots,m\\
\nabla f(x) -\sum_{i=1}^m \nabla g_i(x)&=0
\end{aligned}
\label{label:qm:eqn:lagrangemultiplikatoren2}
\end{equation}
gefunden werden kann.
Die letzte Gleichung ist eine Vektorgleichung mit $n$-Komponenten,
das System~\eqref{label:qm:eqn:lagrangemultiplikatoren2} ist also
ein Gleichungssystem mit $m+n$ Gleichungen für die $n+m$
Unbekannten $x$ und $\lambda=(\lambda_1,\dots,\lambda_m)$, ein Zeilenvektor.
\index{Zeilenvektor}%

Die Funktionen $g_i(x)=0$ können in einen $m$-dimensionalen Vektor
$g(x)$ zusammengefasst werden, für die wir auch den Gradienten
\index{Gradient}%
\[
\nabla g(x)
=
\renewcommand\arraystretch{1.25}
\begin{pmatrix}
\displaystyle\frac{\partial g_1}{\partial x_1} & \dots
	& \displaystyle\frac{\partial g_m}{\partial x_1}\\
\displaystyle\frac{\partial g_1}{\partial x_2} & \dots
	& \displaystyle\frac{\partial g_m}{\partial x_2}\\
\vdots & \ddots & \vdots \\
\displaystyle\frac{\partial g_1}{\partial x_n} & \dots
	& \displaystyle\frac{\partial g_m}{\partial x_n}
\end{pmatrix}
\]
definieren können.
Mit diesen Notation wird das Gleichungssystem
\eqref{label:qm:eqn:lagrangemultiplikatoren2}
schreiben als
\begin{equation}
\begin{aligned}
g(x)&=0\\
\nabla f(x) - \lambda  \nabla g(x) &= 0.
\end{aligned}
\label{label:qm:eqn:lagrangemultiplikatoren3}
\end{equation}
In dieser Form versuchen wir das Problem zu lösen.

%
% Ableitungen
%
\subsubsection{Gradienten von linearen und quadratischen Formen}
Für die Lösung eines quadratischen Minimalproblems mit Hilfe der Gleichungen
\eqref{label:qm:eqn:lagrangemultiplikatoren3} muss der Gradient
von quadratischen und linearen Funktionen berechnet werden können.
In diesem Abschnitt tragen wir die benötigten Formeln zusammen.

\begin{lemma}
\label{buch:qm:lemma:gradlin}
Sei $v$ ein $n$-dimensionaler Spaltenvektor und $w$ ein $n$-dimensionaler
Zeilenvektor.
Der Gradient der Funktionen $f(x)=x^tv$ und $g(x)=wx$ ist
\begin{equation}
\begin{aligned}
\nabla f &= v
&&\text{und}&
\nabla g &= w^t.
\end{aligned}
\end{equation}
\end{lemma}

\begin{proof}[Beweis]
Die Funktionen $f$ und $g$ sind etwas ausführlicher geschrieben
\[
\begin{aligned}
f(x) &= x^tv = \sum_{i=1}^n x_iv_i
&&\text{und}&
g(x) &= wx = \sum_{i=1}^n w_ix_i.
\end{aligned}
\]
Die partiellen Ableitungen von $f$ und $g$ sind
\begin{align*}
\frac{\partial f}{\partial x_k}
&=
\sum_{i=1}^n \frac{\partial x_i}{\partial x_k} v_i
=
\sum_{i=1}^n \frac{\partial x_i}{\partial x_k} v_i
=
\sum_{i+1}^n \delta_{ik} v_i
=
v_k
\\
\frac{\partial f}{\partial x_k}
&=
\sum_{i=1}^n w_i\frac{\partial x_i}{\partial x_k}
=
\sum_{i=1}^n w_i\frac{\partial x_i}{\partial x_k}
=
\sum_{i+1}^n w_i\delta_{ik} =w_k
\end{align*}
also
$\nabla f = v$ und $\nabla g = w$.
\end{proof}

\begin{lemma}
\label{buch:qm:lemma:gradsquare}
Ist $B$ eine $n\times n$-Matrix, dann ist der Gradient der
quadratischen Form $q(x) = x^tBx$
\begin{equation}
\nabla q(x) = (B+B^t)x.
\end{equation}
Falls $B$ symmetrisch ist, ist $\nabla q(x) = 2Bx$.
\end{lemma}

\begin{proof}[Beweis]
Die Funktion $q(x)$ ist ausführlicher geschrieben
\[
q(x) = x^tBx = \sum_{i,j=1}^n x_ib_{ij}x_j.
\]
Die partiellen Ableitungen sind
\begin{align*}
\frac{\partial q}{\partial x_k}
&=
\sum_{i,j=1}^n \frac{\partial}{\partial x_k} x_ib_{ij}x_j
=
\sum_{i,j=1}^n \frac{\partial x_i}{\partial x_k} b_{ij}x_j
+
\sum_{i,j=1}^n x_ib_{ij}\frac{\partial x_j}{\partial x_k}
=
\sum_{i,j=1}^n \delta_{ik} b_{ij}x_j
+
\sum_{i,j=1}^n x_ib_{ij}\delta_{jk}
\\
&=
\sum_{j=1}^n b_{kj}x_j
+
\sum_{i=1}^n x_ib_{ik}.
\end{align*}
Die beiden Terme sind die $k$-Komponente von $Bx$ und die $k$-Komponente von
$B^tx$, es folgt $\nabla q(x) = (B+B^t)x$.
\end{proof}


%
% Ein paar Matrix-Regeln
%
\subsubsection{Invertierung von Blockmatrizen}
Eine $2\times 2$-Matrix ist sehr leicht zu invertieren, es ist
\begin{equation}
A=\begin{pmatrix}
a&b\\
c&d
\end{pmatrix}
\qquad\Rightarrow\qquad
A^{-1}
=
\frac{1}{ad-bc}
\begin{pmatrix}
d&-b\\
-c&a
\end{pmatrix}.
\label{buch:qm:eqn:inverse22}
\end{equation}
Dies lässt sich zum Beispiel mit dem Gauss-Algorithmus sofort
beweisen.
Auf die gleiche Weise kann man aber auch eine Formel für die Inverse
einer Blockmatrix herleiten.
\index{Blockmatrix}%
\index{Inverse!einer Blockmatrix}%

\begin{lemma}
\label{buch:qm:lemma:blockinverse}
Gegeben ist die reguläre Matrix $(n+m)\times(n+m)$-Matrix
\[
M = \begin{pmatrix}
A&B\\
C&D
\end{pmatrix},
\qquad\text{und}\quad\left\{
\quad
\begin{aligned}
&\text{$A$ eine $n\times n$-Matrix}\\
&\text{$B$ eine $n\times m$-Matrix}\\
&\text{$C$ eine $m\times n$-Matrix}\\
&\text{$D$ eine $m\times m$-Matrix}.
\end{aligned}
\right.
\]
Falls $A$ regulär ist, ist auch $D-CA^{-1}B$ regulär und
die Inverse von $M$ ist
\begin{equation}
M
=
\begin{pmatrix}
A^{-1} - A^{-1}B(D-CA^{-1}B)^{-1}CA^{-1}  & -A^{-1}B(D-CA^{-1}B)^{-1} \\
(D-CA^{-1}B)^{-1}CA^{-1}                 & (D-CA^{-1}B)^{-1}
\end{pmatrix}.
\label{buch:qm:eqn:blockinverse}
\end{equation}
\end{lemma}

\begin{proof}[Beweis]
Wir schreiben $E_n$ für die $n\times n$-Einheitsmatrizen und führen den
Gauss-Algorithmus auf einem Tableau von Blockmatrizen durch.
\index{Gauss-Algorithmus}%
\index{Einheitsmatrix}%
Der Leser ist aufgefordert, sich zu überlegen, dass die Operationen
des Gauss-Algorithmus auch funktionieren, wenn man sie in einer Algebra
von Matrizen durchführt, man muss nur sorgfältig darauf achten, die
Reihenfolge der Faktoren nicht zu verändern.

Der erste Schritt im Gauss-Algorithmus ist die Pivot-Division durch
$A$, was in der Matrizenalgebra die Multiplikation von links mit
$A^{-1}$ wird.
Nach Voraussetzung existiert $A^{-1}$, so dass die Zeilenoperation
mit Pivot $A$ durchgeführt werden kann:
\begin{align*}
\renewcommand\arraystretch{1.25}
\begin{tabular}{|>{$}c<{$}>{$}c<{$}|>{$}c<{$}>{$}c<{$}|}
\hline
A&B&E_n&0   \\
C&D&0  &E_m \\
\hline
\end{tabular}
&\rightarrow
\renewcommand\arraystretch{1.25}
\begin{tabular}{|>{$}c<{$}>{$}c<{$}|>{$}c<{$}>{$}c<{$}|}
\hline
E_n&A^{-1}B&A^{-1}&0   \\
C  &D      &0     &E_m \\
\hline
\end{tabular}
\\
&\rightarrow
\renewcommand\arraystretch{1.25}
\begin{tabular}{|>{$}c<{$}>{$}c<{$}|>{$}c<{$}>{$}c<{$}|}
\hline
E_n&A^{-1}B    &A^{-1}   &0   \\
0  &D-CA^{-1}B &-CA^{-1} &E_m \\
\hline
\end{tabular}.
\intertext{Jetzt muss die Pivotdivision durch $D-CA^{-1}B$ durchgeführt
werden, was aber nur möglich ist, wenn $D-CA^{-1}B$ regulär ist.
Wäre $D-CA^{-1}B$ nicht regulär, dann könnte auch $M$ nicht regulär
sein.
Somit kann auch die Pivotdivision durch $D-CA^{-1}B$ und das
\index{Pivotdivision}%
\index{Rückwärtseinsetzen}%
Rückwärtseinsetzen durchgeführt werden, was auf}
&\rightarrow
\renewcommand\arraystretch{1.25}
\begin{tabular}{|>{$}c<{$}>{$}c<{$}|>{$}c<{$}>{$}c<{$}|}
\hline
E_n&A^{-1}B &A^{-1}                    &0   \\
0  &E_m     &-(D-CA^{-1}B)^{-1}CA^{-1} &(D-CA^{-1}B)^{-1} \\
\hline
\end{tabular}
\\
&\rightarrow
\renewcommand\arraystretch{1.25}
\begin{tabular}{|>{$}c<{$}>{$}c<{$}|>{$}c<{$}>{$}c<{$}|}
\hline
E_n&0  &A^{-1}+A^{-1}B(D-CA^{-1}B)^{-1}CA^{-1} &-A^{-1}B(D-CA^{-1}B)^{-1} \\
0  &E_m&-(D-CA^{-1}B)^{-1}CA^{-1}              &(D-CA^{-1}B)^{-1} \\
\hline
\end{tabular}
\end{align*}
führt.
Daraus kann man die inverse Matrix von $M$ ablesen und findet
\eqref{buch:qm:eqn:blockinverse}
\end{proof}

\begin{beispiel}
Im Fall $n=m=1$ sind die Blöcke von $M$ gewöhnliche reelle Zahlen und es
kommt nicht mehr auf die Reihenfolge der Faktoren an, damit bekommt man
für die Inverse der Matrix
\begin{align*}
A&=\begin{pmatrix}a&b\\c&d\end{pmatrix}
&
A^{-1}
&=
\begin{pmatrix}
\displaystyle \frac1a + \frac{b}a\biggl(d-\frac{cb}a\biggr)^{-1}\frac{c}a
	&\displaystyle -\frac{b}a \biggl(d-\frac{cb}{a}\biggr)^{-1}
\\[10pt]
\displaystyle -\frac{c}a\biggl(d-\frac{cb}{a}\biggr)^{-1}
	&\displaystyle \biggl(d-\frac{bc}{a}\biggr)^{-1}
\end{pmatrix}
\\
&&&=
\begin{pmatrix}
\displaystyle \frac1a + \frac{bc}{a}\frac{1}{ad-bc}
	&\displaystyle \frac{-b}{ad-bc}
\\[10pt]
\displaystyle \frac{-c}{ad-bc}
	&\displaystyle \frac{a}{ad-bc}
\end{pmatrix}
\\
&&&=
\frac{1}{ad-bc}
\begin{pmatrix}
\displaystyle \frac{(ad-bc)-ad}a&-b\\
-c&a
\end{pmatrix}
=
\frac{1}{ad-bc}
\begin{pmatrix}
 d&-b\\
-c& a
\end{pmatrix},
\end{align*}
die Formel \eqref{buch:qm:eqn:inverse22}
für die Inverse einer $2\times 2$-Matrix.
\end{beispiel}

\begin{korollar}
\label{buch:qm:korollar:quadr}
Ist $B$ eine symmetrische, positiv definite $n\times n$-Matrix und 
und $A$ eine $m\times n$-Matrix mit $m\le n$ und $\operatorname{Rang}A=m$.
Dann ist 
\begin{equation}
\begin{pmatrix}
2B&-A^t \\
-A & 0
\end{pmatrix}^{-1}
=
\begin{pmatrix}
\frac12B^{-1} - \frac12B^{-1}A^t(AB^{-1}A^t)^{-1}AB^{-1}
      & -B^{-1}A(AB^{-1}A^t)^{-1} \\
-(AB^{-1}A^t)^{-1}AB^{-1} & \frac12 (AB^{-1}A^t)^{-1}
\end{pmatrix}
\label{buch:qm:eqn:quadr}
\end{equation}
\end{korollar}

\begin{proof}[Beweis]
Dies ist der Fall $A=2B$, $B=-A^t$, $C=-A$ und $D=0$ des
Lemmas~\ref{buch:qm:lemma:blockinverse}.
\end{proof}

%
% Lösung des quadratischen Minimalproblems
%
\subsubsection{Die Lösung eines quadratischen Minimalproblems}
Das quadratische Minimalproblem~\ref{buch:qm:problem:allg} sucht
einen Vektor $x$ derart, dass $f(x)=x^tBx$ minimiert wird unter der
Nebenbedingung $Ax=b$.
Die Funktion $g(x)$ für dieses nichtlineare Extremalproblem ist
$g(x)=Ax-b$.
Nach dem Verfahren der Lagrange-Multiplikatoren
\ref{buch:qm:lemma:lagrangemultiplikatoren}
sind Vektoren $x$ und $\lambda$ zu finden derart, dass
\begin{align*}
g(x) &= 0
\\
\nabla f(x) -\lambda \nabla g(x) &=0
\end{align*}
gilt.
Wir verwenden die Lemmata~\ref{buch:qm:lemma:gradlin}
und \ref{buch:qm:lemma:gradsquare}
zur Berechnung der Gradienten von $f(x)$ und $g(x)$:
\begin{align*}
\nabla f(x) &= (B+B^t)x
\\
\nabla g(x) &= A^t.
\end{align*}
So entsteht das lineare Gleichungssystem
\[
\left.
\begin{aligned}
2Bx - A^t\lambda &=0
\\
Ax-b&=0 &&\Rightarrow& Ax&=b
\end{aligned}
\right\}
\quad\text{oder}\quad
\begin{pmatrix}
2B & -A^t \\
 A & 0
\end{pmatrix}
\begin{pmatrix}x\\\lambda\end{pmatrix}
=
\begin{pmatrix}0\\ b\end{pmatrix}
\]
in Matrixform.

Die Matrix hat die in Korollar
\ref{buch:qm:korollar:quadr} untersuchte Form.
Mit der Formel \eqref{buch:qm:eqn:quadr} für die Inverse können wir jetzt 
die Lösung angeben:
\begin{align*}
x       &= B^{-1}A(AB^{-1}A^t)^{-1} b \\
\lambda &= \frac12(AB^{-1}A^t)^{-1}b.
\end{align*}
Insbesondere ist damit die Lösung des quadratischen Minimalproblems
vollständig auf Operationen mit Matrizenoperationen zurückgeführt.

%
% Neue Lösung für das Least-Squares-Problem
%
\subsubsection{Least Squares mit Hilfe des Gradienten}
Zur weiteren Illustration der Rechentechnik mit Gradienten von
Matrixfunktionen lösen wir hier auch noch das Least-Squares-Problem
\ref{buch:qm:problem:ls2} mit dieser Methode.
In diesem Fall haben wir keine Nebenbedingungen.
Wir müssen nur das Minimum des Ausdrucks~\eqref{buch:qm:eqn:ls2}
bestimmen.
Wir tun dies, indem wir Nullstellen der Ableitung suchen.
Der Gradient von $Q(x)$ ist
\begin{align*}
\nabla
Q(x)
&=
2A^tA x -2 A^tb = 0
&&\Rightarrow&A^tAx&=A^tb.
\end{align*}
Daraus leitet man die bekannte Lösung
\[
x = (A^tA)^{-1}A^tb
\]
ab.







