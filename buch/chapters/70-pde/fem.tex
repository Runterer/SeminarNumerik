%
% fem.tex
%
% (c) 2020 Prof Dr Andreas Müller, Hochschule Rapperswil
%
\section{Finite Elemente
\label{section:finite-elemente}}
\rhead{Finite Elemente}
Die Technik der finiten Volumina basierte darauf, dass die gesuchte
Funktion durch Integrale über Volumina oder Flächenstücke ersetzt 
werden konnte, zwischen denen lineare Gleichungen gelten.
Dies ist jedoch nicht die einzige denkbare Vorgehensweise.
In diesem Abschnitt zeigen wir, wie gewisse partielle Differentialgleichungen
in ein äquivalentes Minimalprinzip umgewandelt werden können.
Approximiert man die gesuchte Funktion anschliessend durch geeignete
Interpolationspolynome, wird das Minimalproblem zu einem quadratischen
Minimalproblem für die Koeffizienten der Interpolationspolynome,
welches mit Methoden der linearen Algebra gelöst werden kann.
\index{Minimalproblem}%

\subsection{Das äquivalente Minimalproblem}
Zur Illustration des Prinzips soll in diesem Abschnitt das Eigenwertproblem
für eine elliptische Differentialgleichung zweiter Ordnung betrachtet werden.

\subsubsection{Ein eindimensionales Problem}
Wir betrachten die Differentialgleichung
\begin{equation}
u''(x) = \lambda u(x)
\label{pde:fem:1dgl}
\end{equation}
auf dem Interval $[a,b]$ mit Randwerten  $u(a)=u(b)=0$
und möchten zeigen, dass eine Lösung gleichzeitig ein stationärer
Punkt des Integrals
\begin{equation}
I(u)
=
\int_a^b u'(x)^2 + \lambda u(x)^2 \,dx
\label{pde:fem:1minimal}
\end{equation}
ist.

\begin{satz}
\label{pde:satz:minimal1}
Eine Funktion $u(x)$ ist genau dann eine Lösung der Differentialgleichung
\eqref{pde:fem:1dgl}, wenn sie das Funktional
$I(u)$ von \eqref{pde:fem:1minimal} minimiert.
\end{satz}
\index{Funktional}%
\index{Variation}%

\begin{proof}[Beweis]
Ein Minimum der Funktion $I(u)$ erfüllt die Bedingung, dass jede 
Variation $u(x)+\varepsilon h(x)$ von $u$, die den Randbedingungen genügt,
einen grösseren Wert von $I$ liefert.
Die Ableitung nach $\varepsilon$ verschwindet also an der Stelle
$\varepsilon=0$.
Wir setzen
\[
I(\varepsilon) = I(u(x) + \varepsilon h(x))
\]
mit beliebigen Funktionen $h(x)$, die am Rand verschwinden: $h(a)=h(b)=0$.
\begin{align}
0
=
\frac{dI(\varepsilon)}{d\varepsilon}\bigg|_{\varepsilon=0}
&=
\frac{d}{d\varepsilon}
\int_a^b (u'(x)+\varepsilon h'(x))^2  + (u(x)+\varepsilon h(x))^2\,dx
\bigg|_{\varepsilon=0}
\notag
\\
&=
\int_a^b
2u'(x)h'(x) + 2\varepsilon h'(x)^2
+
2\lambda u(x)h(x) + 2\lambda \varepsilon h(x)^2
\,dx
\bigg|_{\varepsilon=0}
\notag
\\
&=
2
\int_a^b
u'(x)h'(x)
+
\lambda u(x)h(x)
\,dx
\notag
\intertext{Um das Integralprinzip von Lemma~\ref{buch:lemma:integralprinzip}
anwenden zu können, darf nur $h(x)$ vorkommen.
\index{Integralprinzip}%
Wir können die Ableitung $h'(x)$ mit Hilfe von partieller Ableitung zum
Verschwinden bringen.}
&=
2\biggl[u'(x)h(x)\biggr]_a^b
-
2\int_a^b u''(x) h(x)\,dx
+
2\int_a^b \lambda u(x) h(x)\,dx.
\notag
\intertext{Der erste Term verschwindet, da $h(x)$ an den Intervallenden
verschwindet:}
&=
2\int_a^b \bigl(-u''(x) +\lambda u(x)\bigr)\,h(x)\,dx.
\label{pde:fem:1integral}
\end{align}
Jetzt kann Lemma~\ref{buch:lemma:integralprinzip} angewendet werden: das
Integral~\eqref{pde:fem:1integral} kann nur dann für alle Funktionen
$h(x)$ verschwinden, wenn
\[
u''(x)=-\lambda u(x)
\]
gilt.
\end{proof}


\subsubsection{Partielle Differentialgleichung auf einem Rechteck}
Das gleiche Prinzip ist auch anwendbar für das Eigenwertproblem
\begin{equation}
\Delta u(x) = \lambda u(x),
\qquad \Delta
=
\frac{\partial^2}{\partial x^2} + \frac{\partial^2}{\partial y^2}
\label{pde:fem:2dgl}
\end{equation}
auf einem Rechteck $\Omega = (a,b)\times (c,d)$, es ist nur nicht
ganz so klar, wie das Problem formuliert werden muss.
Der zentrale Schnitt im Beweis von Satz~\ref{pde:satz:minimal1}
war partielle Integration und die Ausnutzung der Randwerte von $h(x)$.
Indem wir dieses Beispiel auf ein Rechteck verallgemeinern, können
wird die richtige Verallgemeinerung von Satz~\ref{pde:satz:minimal1}
finden.

Da die Funktion von zwei Variablen abhängt, gibt es jetzt nicht nur
eine erste Ableitung sondern deren zwei.
Statt dem Quadrat der ersten Ableitung $u'(x)^2$ werden daher
einen analogen Terme für beide ersten Ableitungen benötigen.
\index{Quadrant}%
\index{Quadratsumme}%
Die Quadratsumme der Ableitungen
\[
(\nabla u)^2
=
\biggl(\frac{\partial u}{\partial x}\biggr)^2
+
\biggl(\frac{\partial u}{\partial y}\biggr)^2
\]
liegt auf der Hand.
In Analogie zum eindimensionalen Problem verwenden wir daher
als Minimalproblem das Funktional
\index{Funktional}%
\[
I(u)
=
\int_a^b\int_c^d \nabla u(x)^2 + \lambda u(x)^2\,dy \,dx
\]
und schreiben wieder
\[
I(\varepsilon)
=
I(u + \varepsilon h)
\]
für eine Funktion $h\colon \Omega\to\mathbb R$, die auf dem Rand
verschwindet: also
\[
u(a,y) = u(b,y) = u(x,c) = u(x,d) = 0
\]
für beliebige $x\in[a,b]$ und $y\in[c,d]$.

Die Ableitung nach $\varepsilon$ an der Stelle $\varepsilon=0$ ist
\begin{align*}
\frac{dI(\varepsilon)}{d\varepsilon}\bigg|_{\varepsilon=0}
&=
\frac{d}{d\varepsilon}
\int_a^b\int_c^d
(\nabla u(x)+\varepsilon \nabla h(x))^2
+
\lambda (u(x) + \varepsilon h(x))^2
\,dy \,dx
\bigg|_{\varepsilon=0}
\\
&=
2
\int_a^b\int_c^d
\nabla u(x)\cdot \nabla h(x) +\varepsilon \nabla h(x)^2
+
\lambda u(x) h(x) + \lambda \varepsilon h(x)^2
\,dy \,dx
\bigg|_{\varepsilon=0}
\\
&=
2
\int_a^b\int_c^d
\nabla u(x)\cdot \nabla h(x) + \lambda u(x) h(x)
\,dy \,dx.
\end{align*}

Das erste Term im Integranden enthält wieder Ableitungen der Funktion
$h$, die wir loswerden müssen.

\begin{lemma}
\label{pde:lemma:partint2}
Ist $v(x,y)$ eine beliebige Funktion $v\colon\Omega\to\mathbb R^2$ und
$h\colon\Omega\to\mathbb R$. 
Dann gilt für das Integral von $v\cdot\nabla u$ die partielle
Integrationsformel
\index{partielle Integration}
\index{Integration!partiell}
\begin{align}
\int_a^b\int_c^d v(x,y)\cdot \nabla h(x,y)\,dy\,dx
&=
\int_c^d
v_x(b,y) h(b,y)
-
v_x(a,y) h(a,y)
\,dy
\notag
\\
&\qquad
+
\int_a^b
v_y(x,d) h(x,d)
-
v_y(x,c) h(x,c)
\,dx
\notag
\\
&\qquad
-\int_a^b\int_c^d
\frac{\partial v_x}{\partial x}+\frac{\partial v_y}{\partial y}
\,dy\,dx
\label{pde:fem:part2d}
\end{align}
\end{lemma}

\begin{proof}[Beweis]
\definecolor{orange}{rgb}{1,0.6,0.2}
\definecolor{gruen}{rgb}{0.2,0.6,0.2}
\definecolor{magenta}{rgb}{0.9,0.2,0.4}
\definecolor{azure}{rgb}{0,0.2,1}
Wir teilen das Integral mit Hilfe von
\[
v(x,y)\cdot\nabla h(x,y) = 
v_x(x,y) \, \frac{\partial h}{\partial x}(x,y)
+
v_y(x,y) \, \frac{\partial h}{\partial x}(x,y)
\]
in zwei Summanden
\[
\int_a^b\int_c^dv\cdot\nabla h \,dy\,dx
=
\int_a^b\int_c^dv_x \frac{\partial h}{\partial x} \,dy\,dx
+
\int_a^b\int_c^dv_y \frac{\partial h}{\partial y} \,dy\,dx
=
I_1+I_2
\]
auf, die wir separat berechnen können.
Für den ersten Summanden erhalten wir
\begin{align}
I_1
&=
\int_a^b\int_c^d v_x(x,y) \frac{\partial h}{\partial x}(x,y)\,dy\,dx
\notag
\\
&=
\int_c^d
\int_a^b
v_x(x,y) \frac{\partial h}{\partial x}(x,y)
\,dx
\,dy
\notag
\\
&=
\int_c^d
\biggl[
v_x(x,y) h(x,y)
\biggr]_a^b
-
\int_a^b \frac{\partial v_x}{\partial x}(x,y) h(x,y)
\,dx
\,dy
\notag
\\
&=
\int_c^d
v_x(b,y) h(b,y)
-
v_x(a,y) h(a,y)
-
\int_a^b \frac{\partial v_x}{\partial x}(x,y) h(x,y)
\,dx
\,dy
\notag
\\
&=
\int_c^d
{\color{magenta}
v_x(b,y) h(b,y)}
{\color{azure}
\mathstrut
-
v_x(a,y) h(a,y)}
\,dy
-
\int_a^b
\int_c^d
\frac{\partial v_x}{\partial x}(x,y) h(x,y)
\,dy
\,dx.
\label{buch:pde:part2dI1}
\intertext{%
Die zweite Summe ist noch einfacher, weil es gar nicht erst notwendig ist,
die Integrationsreihenfolge zu ändern:
\index{Integrationsreihenfolge}}
I_2
&=
\int_a^b\int_c^d v_y(x,y)\frac{\partial h}{\partial y}\,dy\,dx
\notag
\\
&=
\int_a^b
\biggl[ v_y(x,y) h(x,y)\biggr]_c^d
-
\int_c^d
\frac{\partial v_y}{\partial y} h(x,y)
\,dy
\,dx
\notag
\\
&=
\int_a^b
{\color{gruen}
v_y(x,d)h(x,d)}
{\color{orange}\mathstrut
-v_y(x,c)h(x,c)}
\,dx
-
\int_a^b
\int_c^d
\frac{\partial v_y}{\partial y} h(x,y)
\,dy
\,dx
\label{buch:pde:part2dI2}
\end{align}
Die beiden Terme zusammen geben genau die im Lemma behauptete Formel.
\end{proof}

\begin{figure}
\centering
\includegraphics{chapters/70-pde/images/2dpart.pdf}
\caption{Die Randterme des Integrals \eqref{pde:fem:part2d}  können
als ein Flussintegral über den Rand des Rechtecks geschrieben werden.
\index{Flussintegral}%
Dazu müssen die Integrale über die einzelnen Kanten des Rechtecks einzeln
als Flussintegrale über die Kante geschrieben werden.
Die Terme werden auch in den Gleichungen
\eqref{buch:pde:part2dI1} und \eqref{buch:pde:part2dI2}
mit den gleichen Farben hervorgehoben.
\label{buch:pde:pfadintegral}}
\end{figure}
Der erste Integral auf der rechten Seite von Lemma~\ref{pde:lemma:partint2}
kombiniert Integrale von $v_x(x,y) h(x,y)$ über die beiden vertikalen Kanten 
des Rechtecks, während das zweite Integral die Integrale von
$v_y(x,y)h(x,y)$ über die horizontalen Kanten kombiniert (siehe auch
Abbildung~\ref{buch:pde:pfadintegral}).
Schreibt man $\vec{n}(x,y)$ für die nach aussen zeigende Normale auf den Rand 
des Rechtecks, dann können diese Integrale alle einheitlich geschrieben
werden:
\begin{center}
\definecolor{orange}{rgb}{1,0.6,0.2}
\definecolor{gruen}{rgb}{0.2,0.6,0.2}
\definecolor{magenta}{rgb}{0.9,0.2,0.4}
\definecolor{azure}{rgb}{0,0.2,1}
\begin{tabular}{l >{$}c<{$} >{$}l<{$}}
Kante& &\text{Integrand}
\\[2pt]
\hline
\\[-7pt]
\color{orange}untere Kante
	&y=c
	&         - v_x(x,c)h(x,c) = h(x,c) \, \vec{v}(x,c) \cdot \vec{n}(x,c)
\\[4pt]
\color{gruen}obere Kante
	&y=d
	&\phantom{-}v_x(x,d)h(x,d) = h(x,d) \, \vec{v}(x,d) \cdot \vec{n}(x,d)
\\[4pt]
\color{azure}linke Kante
	&x=a
	&         - v_y(a,y)h(a,y) = h(a,y) \, \vec{v}(a,y) \cdot \vec{n}(a,y)
\\[4pt]
\color{magenta}rechte Kante
	&x=b
 	&\phantom{-}v_y(b,y)h(b,y) = h(b,y) \, \vec{v}(b,y) \cdot \vec{n}(b,y)
\\[4pt]
\hline
\end{tabular}
\end{center}
Es folgt also, dass die einfachen Integrale in 
Lemma~\ref{pde:lemma:partint2} das Integral
\[
\int_{\partial\Omega} \vec{v}(x,y) h(x,y) \cdot d\vec{n}
\]
ist.

Nach diesen Vorarbeiten können wir jetzt das zur Differentialgleichung
\eqref{pde:fem:2dgl} gehörige äquivalente Minimalprinzip formulieren.

\begin{satz}
Die Funktion $u\colon\Omega\to\mathbb R$ ist genau dann eine Lösung der
Differentialgleichung
\[
\Delta u =\lambda u
\]
wenn sie das Funktional
\[
I(u)
=
\int_{\Omega} \nabla u(x,y)^2 + \lambda u(x,y)^2 \,dx\,dy
\]
minimiert.
\index{Funktional}%
\end{satz}

\begin{proof}[Beweis]
Die Ableitung von $I(\varepsilon)$ an der Stelle $\varepsilon=0$
wurde früher schon berechnet, das Integral~\eqref{pde:fem:part2d}
muss jetzt mit Hilfe der Formel für die partielle Integration
von Lemma~\ref{pde:lemma:partint2} umgeformt werden.
Dazu setzen wir $\vec{v}=\nabla u$ und erhalten für
\[
\frac{\partial v_x}{\partial x}
+
\frac{\partial v_y}{\partial y}
=
\frac{\partial^2 u}{\partial x^2}
+
\frac{\partial^2 u}{\partial y^2}
=
\Delta u.
\]
Es folgt
\begin{align*}
0
&=
\int_{\partial\Omega} \nabla u(x,y) h(x,y) \cdot d\vec{n}
-
\int_{\Omega} \Delta u(x,y) h(x,y)\,dx\,dy
+
\int_{\Omega} \lambda u(x,y) h(x,y)\,dx\,dy
\\
&=
-
\int_{\Omega} \bigl(\Delta u(x,y)-\lambda u(x,y)\bigr) h(x,y)\,dx\,dy.
\end{align*}
Dies gilt genau dann für jede Funktion $h$, wenn
\[
\Delta u = \lambda u,
\]
wenn also $u$ eine Lösung des Eigenwertproblems ist.
\index{Eigenwertproblem}%
\end{proof}

%
% loesung.tex -- Beispiel-File für die Beschreibung der Loesung
%
% (c) 2020 Prof Dr Andreas Müller, Hochschule Rapperswil
%
\section{Approximation
\label{kettenbruch:section:Approximation}}
\rhead{Approximation}

In der Einleitung wurde erwähnt, dass die Bestimmung von guten
Näherungsbrüchen eine wichtige Anwendung von Kettenbrüchen ist. Es
gilt nämlich, dass jeder Näherungsbruch der Kettenbruchentwicklung
einer reellen Zahl eine besonders gute rationale Näherung dieser
Zahl ist.

\subsection{Definition}

Eine rationale Zahl $\frac{a}{b}$ mit $b>0$ heisst beste Näherung
erster Art an eine reelle Zahl $x$, wenn es keine von $\frac{a}{b}$
verschiedene rationale Zahl mit gleichem oder kleinerem Nenner gibt,
die bezüglich des Absolutbetrages näher bei $x$ liegt.
Das heisst, dann gilt für alle rationalen Zahlen $\frac{c}{d} \ne
\frac{a}{b}$ mit $0<d\le b$:
\begin{equation}
\biggl|x-\frac{a}{b}\biggr| < \biggl| x-\frac{c}{d}\biggr|.
\end{equation}

\subsection{Näherungsgesetz}
Ziel dieses Abschnitt ist es, eine genügend gute Approximation der
Näherungsbrüche nachzuweisen. Gibt man sich eine beliebige Zahl $x$
vor, so kann man sich die Frage stellen, welche "unkürzbaren" Brüche
$\frac{p}{q}$ mit vorgegebenem Höchstnenner sich gut approximieren
lässt.

\begin{beispiel}
Näherung von $\pi$ mit dem unendliche Dezimalbruch:
$\pi = [3;7,15,1,292,1,1,1,2,1,3,1,14,2,\cdots]$
Die Näherung $3.14 = \frac{314}{100}$ ist eine Näherung. Aber
$\frac{22}{7} = 3.14285714\dots$ hat einen viel kleineren Nenner und
ist eine deutlich bessere Näherung von $\pi$.
Eine noch bessere Näherung ist der Kettenbruch
\begin{equation}
\frac{355}{113} = 3 + \cfrac{1}{7+\cfrac{1}{15+\cfrac{1}{1}}} = 3.1415\overline{92}
\end{equation}
Folgende Näherungswerte von $\pi$ können schnell und einfach gerechnet werden:
\begin{equation}
3,\frac{22}{7} \approx 3.143 ; \frac{333}{106} \approx 3.14151 ; \frac{355}{113} 
\approx 3.1415929 ; \frac{103993}{33102} \approx 3.1415926530 ; \cdots.
\end{equation}
\end{beispiel}
Die Bestapproximation ist einfach formuliert durch die Bestimmung
derjenigen rationalen Brüchen, welche von einer gegebenen rationalen
oder irrationalen Zahl einen festgelegten minimalen Abstand haben
und dabei einen möglichst kleinen positiven Nenner besitzen.

\subsubsection{Beispiel}
\begin{beispiel}
Die Funktion $\tan^{-1}(x)$ spielt bei der Berechnung von $\pi$ an vielen Stellen eine Rolle. 
Es gilt die Leibnizsche Reihe
\begin{equation}
\frac{\pi}{4} = \tan^{-1}(1)
=
1 - \frac{1}{3} + \frac{1}{5} - \frac{1}{7} + \frac{1}{9} - \frac{1}{11} +\dots
\end{equation}
Dank der Euler-Transformation gilt folgender Kettenbruch
\begin{equation}
\tan^{-1}(x) = x - \frac{x^3}{3} + \frac{x^5}{5} - \frac{x^7}{7} + \frac{x^9}{9} - 
\frac{x^{11}}{11}\dots,		\qquad x \leq 1
\end{equation}
Somit kann der Kettenbruch von $\tan^{-1}(x)$ folgendermassen dargestellt werden.
\begin{equation}
\tan^{-1}(x)
=
\cfrac{x}{1+\cfrac{x^2}{3+\cfrac{4x^2}{5+\cfrac{9x^2}{7+\cfrac{16x^2}{9+\cdots}}}}}
\qquad	(|x|< 1)
\end{equation}
Das Gleichungssystem kann umgeschrieben werden als Funktion $f_n$
\begin{equation}
f_n(x) = \frac{x}{1+}\frac{x^2}{3+}\frac{4x^2}{5+}\cdots\frac{(n-1)^2 x^2}{2n-1}
\qquad	(|n|\ge 2)
\end{equation}
\end{beispiel}
Hiermit kann nach $n$-te Bildung der Kettenbruchreaktion einen Grenzwert
erreichen:
\begin{equation}
\tan^{-1}(x) = \lim_{n\to\infty} f_n(x), \qquad (|x| < 1)
\end{equation}
Die Konvergenz der Funktion kann an einem Beispiel beurteilt werden. 
\begin{equation}
\tan^{-1}(1) = \pi/4 \approx 0.785398
\end{equation}

\begin{table}
\centering
\begin{tabular}{>{$}c<{$}>{$}l<{$}}
n	& f_n(1) 	\\
\hline
2	& 0.750000 	\\
3	& 0.791667 	\\
4	& 0.784314 	\\
5	& 0.785586 	\\
6	& 0.785366 	\\
7	& 0.785404	\\
8	& 0.785397	\\
9	& 0.785398	\\
\hline
\end{tabular}
\caption{Näherungsstufen mit Kettenbruchentwicklung von der Funktion $\tan^{-1}(1)$
\label{kettenbruch:tabelle}}
\end{table}

In wenigen und einfachen Schritten haben wir mit Hilfe einer
Kettenbruchentwicklung ein System gebildet das die Konvergenz der
Funktion $\tan^{-1}(x)$ vorantreibt und präzise Resultate liefert.


%
% quadratisch.tex
%
% (c) 2020 Prof Dr Andreas Müller, Hochschule Rapperswil
%
\subsection{Quadratische Minimalprobleme
\label{buch:qm:subsection:quadratischeminimalprobleme}}
In den vorangegangenen Beispielen wurde gezeigt, wie sich viele
partielle Differentialgleichungen auf äquivalente Minimalprobleme
zurückführen lassen.
\index{quadratisches Minimalproblem}
Indem man die Funktionen dann als Linearkombinationen approximiert,
kann man daraus quadratische Minimalprobleme für die Koeffizienten
dieser Linearkombinationen gewinnen.
In diesem Abschnitt soll gezeigt werden, wie solche quadratischen
Minimalprobleme algebraisch gelöst werden können.

%
% Least-Squares-Problem als Beispiel
%
\subsubsection{Least-Squares-Problem}
\index{Least-Squares-Problem}%
Das folgende Problem ist ein wohlbekanntes Beispiel für eine einfaches
quadratisches Minimalproblem.
Es soll hier als Einstieg dienen, Ziel ist eine allgemeinere Form von
Minimalproblem zu formulieren und die Analysis hinzuzuziehen, um das
verallgemeinerte Problem zu lösen.
Natürlich springt dabei auch ein neuer Beweis für die bekannte 
Lösung des Least-Squares-Problems heraus.

\begin{problem}
\label{buch:qm:problem:ls}
Sei $A$ eine $n\times m$-Matrix mit $m<n$ und $\operatorname{Rang} A=m$
und $b$ ein $n$-dimensionaler Spaltenvektor.
Finde einen $m$-dimensionalen Vektor $x$ derart, dass $|Ax-b|$ minimal ist.
\end{problem}

Die geometrische Intuition zur Lösung des Problems ist, dass 
$Ax$ die Punkte eines Unterraums sind, der von den Spalten von $A$
aufgespannt wird.
\index{Unterraum}%
Gesucht wird jetzt derjenige Punkt, der möglichst nahe am Punkt $b$
liegt, der möglicherweise ausserhalb des Unterraums ist.
Für diesen Punkt $Ax$ muss die Differenz $Ax-b$ auf dem Unterraum
senkrecht stehen.
Da dieser von den Spalten von $A$ aufgespannt wird, muss $Ax-b$ auf allen
Spalten von $A$ senkrecht stehen.
Das Skalarprodukt aller Spalten von $A$ mit $Ax-b$ muss daher verschwinden,
also
\index{Skalarprodukt}%
\[
A^t(Ax-b) = 0
\qquad\Rightarrow\qquad
A^tAx = A^tb 
\qquad\Rightarrow\qquad
x = (A^tA)^{-1}A^tb.
\]
Diese Argumentation verwendet die geometrischen Eigenschaften des
Skalarproduktes.
Und funktioniert daher nur für den euklidischen Abstand, der von
dem sehr speziellen quadratischen Ausdruck $x_1^2+\dots + x_n^2$
berechnet wird.
Im Folgenden soll das Problem für beliebige quadratische Ausdrücke
verallgemeinert werden.

%
% Problemstellung
%
\subsubsection{Problemstellung}
Das Least Squares Problem\ref{buch:qm:problem:ls} ist ein Spezialfall eines
quadratischen Minimalproblems.

\begin{problem}
\label{buch:qm:problem:ls2}
Sei $A$ eine Matrix und $b$ ein Vektor wie in Problem~\ref{buch:qm:problem:ls}.
Finde einen Vektor $x$ derart, dass die quadratische Funktion
\begin{equation}
Q(x)
=
|Ax-b|^2
=
(Ax-b)^t(Ax-b)
=
x^tA^tAx -x^tA^tb - b^tAx +b^t b
\label{buch:qm:eqn:ls2}
\end{equation}
minimiert wird.
\end{problem}

Während es in der ursprünglichen Formulierung des Least-Squares-Problems
um die Minimierung des Abstands ging, ist jetzt ein Problem entstanden,
in der ein allgemeinerer quadratischer Ausdruck der Form $x^tBx$ 
minimiert werden muss, wobei $B=A^tA$ eine symmetrische Matrix ist.
Allerdings ist immer noch die Beschreibung des Unterraums der zulässigen
Punkte $Ax$ vermischt mit dem quadratischen Ausdruck, der minimiert werden
soll, in dem ebenfalls die Matrix $A$ vorkommt.
Um diese beiden Komponenten klarer zu trennen, formulieren wird das
Problem wie folgt:

\begin{problem}
\label{buch:qm:problem:allg}
Sei $B$ eine positiv definite, symmetrische $n\times n$-Matrix 
und $A$ eine $m\times n$-Matrix mit $m<n$ und $\operatorname{Rang}A=m$.
Sei weiter $b$ ein $m$-dimensionaler Vektor.
Finde einen $n$-dimensionalen Vektor $x$ der $Ax=b$ erfüllt und
$Q(x)=x^tbx$ minimiert.
\end{problem}

%
% Nichtlineare Minimalprobleme
%
\subsubsection{Nichtlineare Minimalprobleme}
\index{Nichtlineares Minimalproblem}%
\index{Minimalproblem!nichtlinear}%
In der Theorie der Funktionen mehrere Variablen lernt man die folgende
Methode zur Lösung nichtlinearer Minimalprobleme.
Sie wird oft auch die Methode der Lagrange-Multiplikatoren genannt,
die Zahlen $\lambda_i$ heissen Lagrange-Multiplikatoren.
\index{Lagrange-Multiplikator}%

\begin{lemma}
\label{buch:qm:lemma:lagrangemultiplikatoren}
Gegeben ist eine Funktion $f\colon \mathbb R^n \to \mathbb R$
und $m$-Funktionen $g_i\colon\mathbb R^n \to \mathbb R$ mit $m<n$.
Ist $x$ ein Punkt, in dem $f(x)$ minimiert wird und gleichzeitig
$g_i(x)=0$ gilt, dann gibt es Zahlen $\lambda_i$ derart dass
\[
\nabla f(x) = \lambda_1 \nabla g_1(x)  + \dots + \lambda_m g_m(x)
\]
gilt.
\end{lemma}

Das Lemma~\ref{buch:qm:lemma:lagrangemultiplikatoren} besagt, dass
$x$ und die Zahlen $\lambda_i$ als Lösung der Gleichungen
\begin{equation}
\begin{aligned}
g_i(x)&=0\qquad i=1,\dots,m\\
\nabla f(x) -\sum_{i=1}^m \nabla g_i(x)&=0
\end{aligned}
\label{label:qm:eqn:lagrangemultiplikatoren2}
\end{equation}
gefunden werden kann.
Die letzte Gleichung ist eine Vektorgleichung mit $n$-Komponenten,
das System~\eqref{label:qm:eqn:lagrangemultiplikatoren2} ist also
ein Gleichungssystem mit $m+n$ Gleichungen für die $n+m$
Unbekannten $x$ und $\lambda=(\lambda_1,\dots,\lambda_m)$, ein Zeilenvektor.
\index{Zeilenvektor}%

Die Funktionen $g_i(x)=0$ können in einen $m$-dimensionalen Vektor
$g(x)$ zusammengefasst werden, für die wir auch den Gradienten
\index{Gradient}%
\[
\nabla g(x)
=
\renewcommand\arraystretch{1.25}
\begin{pmatrix}
\displaystyle\frac{\partial g_1}{\partial x_1} & \dots
	& \displaystyle\frac{\partial g_m}{\partial x_1}\\
\displaystyle\frac{\partial g_1}{\partial x_2} & \dots
	& \displaystyle\frac{\partial g_m}{\partial x_2}\\
\vdots & \ddots & \vdots \\
\displaystyle\frac{\partial g_1}{\partial x_n} & \dots
	& \displaystyle\frac{\partial g_m}{\partial x_n}
\end{pmatrix}
\]
definieren können.
Mit diesen Notation wird das Gleichungssystem
\eqref{label:qm:eqn:lagrangemultiplikatoren2}
schreiben als
\begin{equation}
\begin{aligned}
g(x)&=0\\
\nabla f(x) - \lambda  \nabla g(x) &= 0.
\end{aligned}
\label{label:qm:eqn:lagrangemultiplikatoren3}
\end{equation}
In dieser Form versuchen wir das Problem zu lösen.

%
% Ableitungen
%
\subsubsection{Gradienten von linearen und quadratischen Formen}
Für die Lösung eines quadratischen Minimalproblems mit Hilfe der Gleichungen
\eqref{label:qm:eqn:lagrangemultiplikatoren3} muss der Gradient
von quadratischen und linearen Funktionen berechnet werden können.
In diesem Abschnitt tragen wir die benötigten Formeln zusammen.

\begin{lemma}
\label{buch:qm:lemma:gradlin}
Sei $v$ ein $n$-dimensionaler Spaltenvektor und $w$ ein $n$-dimensionaler
Zeilenvektor.
Der Gradient der Funktionen $f(x)=x^tv$ und $g(x)=wx$ ist
\begin{equation}
\begin{aligned}
\nabla f &= v
&&\text{und}&
\nabla g &= w^t.
\end{aligned}
\end{equation}
\end{lemma}

\begin{proof}[Beweis]
Die Funktionen $f$ und $g$ sind etwas ausführlicher geschrieben
\[
\begin{aligned}
f(x) &= x^tv = \sum_{i=1}^n x_iv_i
&&\text{und}&
g(x) &= wx = \sum_{i=1}^n w_ix_i.
\end{aligned}
\]
Die partiellen Ableitungen von $f$ und $g$ sind
\begin{align*}
\frac{\partial f}{\partial x_k}
&=
\sum_{i=1}^n \frac{\partial x_i}{\partial x_k} v_i
=
\sum_{i=1}^n \frac{\partial x_i}{\partial x_k} v_i
=
\sum_{i+1}^n \delta_{ik} v_i
=
v_k
\\
\frac{\partial f}{\partial x_k}
&=
\sum_{i=1}^n w_i\frac{\partial x_i}{\partial x_k}
=
\sum_{i=1}^n w_i\frac{\partial x_i}{\partial x_k}
=
\sum_{i+1}^n w_i\delta_{ik} =w_k
\end{align*}
also
$\nabla f = v$ und $\nabla g = w$.
\end{proof}

\begin{lemma}
\label{buch:qm:lemma:gradsquare}
Ist $B$ eine $n\times n$-Matrix, dann ist der Gradient der
quadratischen Form $q(x) = x^tBx$
\begin{equation}
\nabla q(x) = (B+B^t)x.
\end{equation}
Falls $B$ symmetrisch ist, ist $\nabla q(x) = 2Bx$.
\end{lemma}

\begin{proof}[Beweis]
Die Funktion $q(x)$ ist ausführlicher geschrieben
\[
q(x) = x^tBx = \sum_{i,j=1}^n x_ib_{ij}x_j.
\]
Die partiellen Ableitungen sind
\begin{align*}
\frac{\partial q}{\partial x_k}
&=
\sum_{i,j=1}^n \frac{\partial}{\partial x_k} x_ib_{ij}x_j
=
\sum_{i,j=1}^n \frac{\partial x_i}{\partial x_k} b_{ij}x_j
+
\sum_{i,j=1}^n x_ib_{ij}\frac{\partial x_j}{\partial x_k}
=
\sum_{i,j=1}^n \delta_{ik} b_{ij}x_j
+
\sum_{i,j=1}^n x_ib_{ij}\delta_{jk}
\\
&=
\sum_{j=1}^n b_{kj}x_j
+
\sum_{i=1}^n x_ib_{ik}.
\end{align*}
Die beiden Terme sind die $k$-Komponente von $Bx$ und die $k$-Komponente von
$B^tx$, es folgt $\nabla q(x) = (B+B^t)x$.
\end{proof}


%
% Ein paar Matrix-Regeln
%
\subsubsection{Invertierung von Blockmatrizen}
Eine $2\times 2$-Matrix ist sehr leicht zu invertieren, es ist
\begin{equation}
A=\begin{pmatrix}
a&b\\
c&d
\end{pmatrix}
\qquad\Rightarrow\qquad
A^{-1}
=
\frac{1}{ad-bc}
\begin{pmatrix}
d&-b\\
-c&a
\end{pmatrix}.
\label{buch:qm:eqn:inverse22}
\end{equation}
Dies lässt sich zum Beispiel mit dem Gauss-Algorithmus sofort
beweisen.
Auf die gleiche Weise kann man aber auch eine Formel für die Inverse
einer Blockmatrix herleiten.
\index{Blockmatrix}%
\index{Inverse!einer Blockmatrix}%

\begin{lemma}
\label{buch:qm:lemma:blockinverse}
Gegeben ist die reguläre Matrix $(n+m)\times(n+m)$-Matrix
\[
M = \begin{pmatrix}
A&B\\
C&D
\end{pmatrix},
\qquad\text{und}\quad\left\{
\quad
\begin{aligned}
&\text{$A$ eine $n\times n$-Matrix}\\
&\text{$B$ eine $n\times m$-Matrix}\\
&\text{$C$ eine $m\times n$-Matrix}\\
&\text{$D$ eine $m\times m$-Matrix}.
\end{aligned}
\right.
\]
Falls $A$ regulär ist, ist auch $D-CA^{-1}B$ regulär und
die Inverse von $M$ ist
\begin{equation}
M
=
\begin{pmatrix}
A^{-1} - A^{-1}B(D-CA^{-1}B)^{-1}CA^{-1}  & -A^{-1}B(D-CA^{-1}B)^{-1} \\
(D-CA^{-1}B)^{-1}CA^{-1}                 & (D-CA^{-1}B)^{-1}
\end{pmatrix}.
\label{buch:qm:eqn:blockinverse}
\end{equation}
\end{lemma}

\begin{proof}[Beweis]
Wir schreiben $E_n$ für die $n\times n$-Einheitsmatrizen und führen den
Gauss-Algorithmus auf einem Tableau von Blockmatrizen durch.
\index{Gauss-Algorithmus}%
\index{Einheitsmatrix}%
Der Leser ist aufgefordert, sich zu überlegen, dass die Operationen
des Gauss-Algorithmus auch funktionieren, wenn man sie in einer Algebra
von Matrizen durchführt, man muss nur sorgfältig darauf achten, die
Reihenfolge der Faktoren nicht zu verändern.

Der erste Schritt im Gauss-Algorithmus ist die Pivot-Division durch
$A$, was in der Matrizenalgebra die Multiplikation von links mit
$A^{-1}$ wird.
Nach Voraussetzung existiert $A^{-1}$, so dass die Zeilenoperation
mit Pivot $A$ durchgeführt werden kann:
\begin{align*}
\renewcommand\arraystretch{1.25}
\begin{tabular}{|>{$}c<{$}>{$}c<{$}|>{$}c<{$}>{$}c<{$}|}
\hline
A&B&E_n&0   \\
C&D&0  &E_m \\
\hline
\end{tabular}
&\rightarrow
\renewcommand\arraystretch{1.25}
\begin{tabular}{|>{$}c<{$}>{$}c<{$}|>{$}c<{$}>{$}c<{$}|}
\hline
E_n&A^{-1}B&A^{-1}&0   \\
C  &D      &0     &E_m \\
\hline
\end{tabular}
\\
&\rightarrow
\renewcommand\arraystretch{1.25}
\begin{tabular}{|>{$}c<{$}>{$}c<{$}|>{$}c<{$}>{$}c<{$}|}
\hline
E_n&A^{-1}B    &A^{-1}   &0   \\
0  &D-CA^{-1}B &-CA^{-1} &E_m \\
\hline
\end{tabular}.
\intertext{Jetzt muss die Pivotdivision durch $D-CA^{-1}B$ durchgeführt
werden, was aber nur möglich ist, wenn $D-CA^{-1}B$ regulär ist.
Wäre $D-CA^{-1}B$ nicht regulär, dann könnte auch $M$ nicht regulär
sein.
Somit kann auch die Pivotdivision durch $D-CA^{-1}B$ und das
\index{Pivotdivision}%
\index{Rückwärtseinsetzen}%
Rückwärtseinsetzen durchgeführt werden, was auf}
&\rightarrow
\renewcommand\arraystretch{1.25}
\begin{tabular}{|>{$}c<{$}>{$}c<{$}|>{$}c<{$}>{$}c<{$}|}
\hline
E_n&A^{-1}B &A^{-1}                    &0   \\
0  &E_m     &-(D-CA^{-1}B)^{-1}CA^{-1} &(D-CA^{-1}B)^{-1} \\
\hline
\end{tabular}
\\
&\rightarrow
\renewcommand\arraystretch{1.25}
\begin{tabular}{|>{$}c<{$}>{$}c<{$}|>{$}c<{$}>{$}c<{$}|}
\hline
E_n&0  &A^{-1}+A^{-1}B(D-CA^{-1}B)^{-1}CA^{-1} &-A^{-1}B(D-CA^{-1}B)^{-1} \\
0  &E_m&-(D-CA^{-1}B)^{-1}CA^{-1}              &(D-CA^{-1}B)^{-1} \\
\hline
\end{tabular}
\end{align*}
führt.
Daraus kann man die inverse Matrix von $M$ ablesen und findet
\eqref{buch:qm:eqn:blockinverse}
\end{proof}

\begin{beispiel}
Im Fall $n=m=1$ sind die Blöcke von $M$ gewöhnliche reelle Zahlen und es
kommt nicht mehr auf die Reihenfolge der Faktoren an, damit bekommt man
für die Inverse der Matrix
\begin{align*}
A&=\begin{pmatrix}a&b\\c&d\end{pmatrix}
&
A^{-1}
&=
\begin{pmatrix}
\displaystyle \frac1a + \frac{b}a\biggl(d-\frac{cb}a\biggr)^{-1}\frac{c}a
	&\displaystyle -\frac{b}a \biggl(d-\frac{cb}{a}\biggr)^{-1}
\\[10pt]
\displaystyle -\frac{c}a\biggl(d-\frac{cb}{a}\biggr)^{-1}
	&\displaystyle \biggl(d-\frac{bc}{a}\biggr)^{-1}
\end{pmatrix}
\\
&&&=
\begin{pmatrix}
\displaystyle \frac1a + \frac{bc}{a}\frac{1}{ad-bc}
	&\displaystyle \frac{-b}{ad-bc}
\\[10pt]
\displaystyle \frac{-c}{ad-bc}
	&\displaystyle \frac{a}{ad-bc}
\end{pmatrix}
\\
&&&=
\frac{1}{ad-bc}
\begin{pmatrix}
\displaystyle \frac{(ad-bc)-ad}a&-b\\
-c&a
\end{pmatrix}
=
\frac{1}{ad-bc}
\begin{pmatrix}
 d&-b\\
-c& a
\end{pmatrix},
\end{align*}
die Formel \eqref{buch:qm:eqn:inverse22}
für die Inverse einer $2\times 2$-Matrix.
\end{beispiel}

\begin{korollar}
\label{buch:qm:korollar:quadr}
Ist $B$ eine symmetrische, positiv definite $n\times n$-Matrix und 
und $A$ eine $m\times n$-Matrix mit $m\le n$ und $\operatorname{Rang}A=m$.
Dann ist 
\begin{equation}
\begin{pmatrix}
2B&-A^t \\
-A & 0
\end{pmatrix}^{-1}
=
\begin{pmatrix}
\frac12B^{-1} - \frac12B^{-1}A^t(AB^{-1}A^t)^{-1}AB^{-1}
      & -B^{-1}A(AB^{-1}A^t)^{-1} \\
-(AB^{-1}A^t)^{-1}AB^{-1} & \frac12 (AB^{-1}A^t)^{-1}
\end{pmatrix}
\label{buch:qm:eqn:quadr}
\end{equation}
\end{korollar}

\begin{proof}[Beweis]
Dies ist der Fall $A=2B$, $B=-A^t$, $C=-A$ und $D=0$ des
Lemmas~\ref{buch:qm:lemma:blockinverse}.
\end{proof}

%
% Lösung des quadratischen Minimalproblems
%
\subsubsection{Die Lösung eines quadratischen Minimalproblems}
Das quadratische Minimalproblem~\ref{buch:qm:problem:allg} sucht
einen Vektor $x$ derart, dass $f(x)=x^tBx$ minimiert wird unter der
Nebenbedingung $Ax=b$.
Die Funktion $g(x)$ für dieses nichtlineare Extremalproblem ist
$g(x)=Ax-b$.
Nach dem Verfahren der Lagrange-Multiplikatoren
\ref{buch:qm:lemma:lagrangemultiplikatoren}
sind Vektoren $x$ und $\lambda$ zu finden derart, dass
\begin{align*}
g(x) &= 0
\\
\nabla f(x) -\lambda \nabla g(x) &=0
\end{align*}
gilt.
Wir verwenden die Lemmata~\ref{buch:qm:lemma:gradlin}
und \ref{buch:qm:lemma:gradsquare}
zur Berechnung der Gradienten von $f(x)$ und $g(x)$:
\begin{align*}
\nabla f(x) &= (B+B^t)x
\\
\nabla g(x) &= A^t.
\end{align*}
So entsteht das lineare Gleichungssystem
\[
\left.
\begin{aligned}
2Bx - A^t\lambda &=0
\\
Ax-b&=0 &&\Rightarrow& Ax&=b
\end{aligned}
\right\}
\quad\text{oder}\quad
\begin{pmatrix}
2B & -A^t \\
 A & 0
\end{pmatrix}
\begin{pmatrix}x\\\lambda\end{pmatrix}
=
\begin{pmatrix}0\\ b\end{pmatrix}
\]
in Matrixform.

Die Matrix hat die in Korollar
\ref{buch:qm:korollar:quadr} untersuchte Form.
Mit der Formel \eqref{buch:qm:eqn:quadr} für die Inverse können wir jetzt 
die Lösung angeben:
\begin{align*}
x       &= B^{-1}A(AB^{-1}A^t)^{-1} b \\
\lambda &= \frac12(AB^{-1}A^t)^{-1}b.
\end{align*}
Insbesondere ist damit die Lösung des quadratischen Minimalproblems
vollständig auf Operationen mit Matrizenoperationen zurückgeführt.

%
% Neue Lösung für das Least-Squares-Problem
%
\subsubsection{Least Squares mit Hilfe des Gradienten}
Zur weiteren Illustration der Rechentechnik mit Gradienten von
Matrixfunktionen lösen wir hier auch noch das Least-Squares-Problem
\ref{buch:qm:problem:ls2} mit dieser Methode.
In diesem Fall haben wir keine Nebenbedingungen.
Wir müssen nur das Minimum des Ausdrucks~\eqref{buch:qm:eqn:ls2}
bestimmen.
Wir tun dies, indem wir Nullstellen der Ableitung suchen.
Der Gradient von $Q(x)$ ist
\begin{align*}
\nabla
Q(x)
&=
2A^tA x -2 A^tb = 0
&&\Rightarrow&A^tAx&=A^tb.
\end{align*}
Daraus leitet man die bekannte Lösung
\[
x = (A^tA)^{-1}A^tb
\]
ab.






