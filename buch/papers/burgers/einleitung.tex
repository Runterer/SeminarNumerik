% !TEX root = ../../buch.tex
% einleitung.tex -- Beispiel-File für die Einleitung
%
% (c) 2020 Prof Dr Andreas Müller, Hochschule Rapperswil
%
\section{Einleitung \label{burgers:section:einleitung}}
\rhead{Einleitung}

	Die Gleichung von Burgers,
\index{Gleichung von Burgers}%
\index{Burgers, Gleichung von}%
	\begin{equation}
		  \frac {\partial u}{\partial t}+u{\frac {\partial u}{\partial x}}=\nu {\frac {\partial ^{2}u}{\partial x^{2}}},
		  \label{burgers:eq_burgers}
	\end{equation}
	ist eine nichtlineare partielle Differentialgleichung.
\index{nichtlinear}%
\index{partielle Differentialgleichung}%
	Das Definitionsgebiet
	\begin{equation}
		\Omega = \left \{ (x,t) \in  \mathbb{R} \, \times \,  \mathbb{R}^t \right \}
	\end{equation}
	besteht aus der Dimension $x$ und der Zeit $t$.
	Die Lösung der Gleichung von Burgers $u(t,x)$ ist in $\Omega$ definiert, wobei die Anfangswerte $u(0,x)$ bekannt sein müssen.
	In diesem Paper wird die reibungsfreie Burgersgleichung
\index{reibungsfrei}%
	\begin{equation}
		\frac {\partial u}{\partial t}+u{\frac {\partial u}{\partial x}}=0
		\label{burgers:eq_invisid_burgers}
	\end{equation}
	betrachtet, wobei der Diffusionsterm $\nu$ von \eqref{burgers:eq_burgers} auf 0 gesetzt ist.

	Das Aussergewöhnliche an der Gleichung ist die Nichtlinearität. Diese Eigenschaft erschwert die numerische L\"osung erheblich.
\index{Nichtlinearität}%

	Als Beispiel kann f\"ur die Randbedingung bei $u(0,x)$ eine Normalverteilung verwendet werden.
	Die gel\"oste Gleichung ist in \autoref{burgers:fig:b1} abgebildet.

	    \begin{figure}
		\centering
		\includegraphics[width=.49\textwidth]{papers/burgers/BurgersEquation/images/Implicit_front.pdf}
		\includegraphics[width=.49\textwidth]{papers/burgers/BurgersEquation/images/Implicit_top.pdf}
		\caption{Gel\"oste Burgersgleichung}
		\label{burgers:fig:b1}
		\end{figure}


	\subsection{Beziehung zu den Navier-Stokes-Gleichungen}
\index{Navier-Stokes-Gleichungen}%
		Die Gleichung wird häufig für die Modelierung von Fluiden verwendet.
		Sie wird gebraucht um die Navier-Stokes-Gleichungen zu vereinfachen.
		Die Navier-Stokes-Gleichungen beschreiben die Str\"omung von Fluiden.
		Die Gleichung von Burgers kann aus der Navier-Stokes-Gleichung

		\begin{equation}
			\rho \left(\frac{\partial u}{\partial t} + u \, \nabla u \right) = -\nabla p + \mu \nabla^2 u + F
			\label{burgers:eq_navier}
		\end{equation}
		 f\"ur newtonsche inkompressible Fluide hergeleitet werden \cite{burgers:navier}.
\index{newtonisches Fluid}%
		Wenn der Druck $p$ und die externe Kraft $F$ vernachl\"asigt wird, ergibt sich
		\begin{equation}
			\rho \left(\frac{\partial u}{\partial t} + u \, \nabla u \right) = \mu \nabla^2 u
			 \label{burgers:eq_navier2}
		\end{equation}
		Da die Dichte $\rho$ für inkompressible Fluide konstant ist, kann sie mit der Viskosität $\mu$ zur Konstante der kinematischen Viskosität $\nu = \frac{\mu}{\rho}$
\index{Dichte}%
\index{Viskosität}%
		\begin{equation}
			 \frac{\partial u}{\partial t} + u \,\nabla u = \nu \nabla^2 u
			 \label{burgers:eq_navier3}
		\end{equation}
		umgeschrieben werden.
		Wie bereits erw\"ahnt, wird die reibungsfreie Gleichung betrachtet ($\nu = 0$).
		Weiter werden die kommenden Berechnungen im eindimensionalen Raum durchgef\"uhrt.
		Somit kann der Ableitungsoperator $\nabla$ als die Ableitung nach $x$ umgeschrieben werden.
		Mit diesen Vereinfachungen der Navier-Stokes-Gleichung f\"ur newtonsche inkompressible Fluide ist man bei der urspr\"unglichen Gleichung \eqref{burgers:eq_invisid_burgers} angelangt.

	%\subsection{Numerische Wetter Vorhersage}
	%	Lewis Fry Richardson (1881 - 1953) war ein Britischer Mathematiker/Physiker.
