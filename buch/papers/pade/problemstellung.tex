%
% problemstellung.tex -- Beispiel-File für die Beschreibung des Problems
%
% (c) 2020 Prof Dr Andreas Müller, Hochschule Rapperswil
%

\section{Problemstellung
\label{pade:section:Problemstellung}}
Eine Padé-Approximation ist ein Bruch, aus zwei Polynomen, welche aus den Koeffizienten der Taylorreihe einer Funktion gewonnen werden. 
Ziel dieses Kapitels ist, dem Leser den Nutzen der Padé-Approximation näher zu bringen und zu zeigen, wie man aus einer analytischen Funktion eine solche Approximation bildet.
Des Weiteren wird auf mehrere praktische Beispiele eingegangen und mögliche Fehlerquellen aufgezeigt. 



\subsection{Potenzreihen
\label{pade:subsection:Potenzreihen}}
\rhead{Potenzreihen}
Die Koeffizienten einer Potenzreihe können verwendet werden, um die Koeffizienten einer Padé-Approximation zu gewinnen. 
In der Analysis kann eine Funktion mit einer Taylorreihe um eine Stelle $x_{0}$ durch eine Potenzreihe dargestellt werden. 
Diese Potenzreihen werden um eine vorgegebene Stelle $x_{0}$ als eine unendliche Summe 
\begin{equation}
f(x)=\sum_{n=0}^{\infty} a_{n} (x-x_{0})^{n} 
\label{pade:expofunk}
\end{equation}
gebildet.
Für viele Funktionen sind die dazugehörigen Potenzreihen schon bekannt. 
Funktionen welche durch eine Potenzreihe dargestellt werden können, werden auch analytische Funktionen genannt.
Es gibt verschiedene Methoden, um eine Potenzreihe einer Funktion zu erhalten. 
Bei analytischen Funktionen sind die Potenzreihen konvergent.
Diese sind oft bekannt und können durch Differentialgleichungen oder eine andere Reihenentwicklungsmethoden hergeleitet werden.

\subsubsection{Beispiel Potenzreihen
\label{pade:section:Bsp_Potenzreihen}}
In diesem Beispiel möchten wir die Potenzreihe der Exponentialfunktion erhalten, welche an der Stelle $x_0 = 0$ entwickelt wird. 
Wir wissen welche Eigenschaften die Exponentialfunktion hat und nutzen diese, um Gleichungen aufzustellen. 
Gesucht ist eine Potenzreihe, welche das Verhalten 
\begin{equation*}
	f^{\prime}(x) = f(x) , \text{ für alle } x \in \mathbb{R} 
\end{equation*}
aufweist.

Die Form der Potenzreihe ist gegeben durch die schon gezeigte Summe \ref{pade:expofunk}.
Weiter wissen wir, dass die Exponentialfunktion 
\begin{equation*}
f(0) = 1
\end{equation*}
erfüllen muss.
Daraus schliessen wir, dass
\begin{equation*}
f(x)=\sum_{n=0}^{\infty} a_{n} \cdot x^{n}
\qquad\Rightarrow\qquad
\sum_{n=0}^{\infty} a_{n} \cdot 0^{n} 
=
a_{0} \cdot 0^{0} + a_{1} \cdot 0^{1} + a_{2} \cdot 0^{2} \dots = 1
\qquad\Rightarrow\qquad
a_{0} = 1
\end{equation*}
und somit ist der erste Koeffizient $a_0$ gefunden.
Anschliessend wird das Polynom abgeleitet

\begin{equation*}
f^{\prime}(x)
=
\sum_{n=0}^{\infty}(n+1) \cdot a_{n+1} \cdot x^{n},
\end{equation*}
um die weiteren Koeffizienten $a_n$ zu erhalten.
Da immer noch die Anforderung $f(x) = f^{\prime}(x)$ gilt, wissen wir, dass die Koeffizienten vor $x^n$  von der Form
\begin{equation*}
(n+1) \cdot a_{n+1} 
= 
a_{n} , \text{ für alle } n \in \mathbb{R}
\end{equation*}
sein müssen. 
Mit dem bereits bekannten $a_0 = 1$ folgt rekursiv
\begin{equation*}
a_{n+1} 
= 
\frac{1}{(n+1)!}
\qquad\Rightarrow\qquad
a_{n} 
= 
\frac{1}{n!}
\end{equation*}
und damit sind die Koeffizienten des Polynoms der Exponentialfunktion ermittelt.
Ausgeschrieben 


\begin{equation}
e^{x}
=
\sum_{n=0}^{\infty} \frac{x^{n}}{n !}
=
\frac{x^{0}}{0 !}+\frac{x^{1}}{1 !}+\frac{x^{2}}{2 !}+\frac{x^{3}}{3 !}+\cdots 
\label{pade:potenzexp}
\end{equation}
Aus der bekannten Potenzreihe der Exponentialfunktion \ref{pade:potenzexp} können nun auch andere Potenzreihen gewonnen werden.
Aus der Exponentialfunktion kann die Potenzreihe für die Sinus- und Kosinus- Funktion ermittelt werden.
Um dies zu erreichen wird, zuerst die Reihe komplex erweitert
\begin{equation*}
e^{x}
=
\sum_{n=0}^{\infty} \frac{x^{n}}{n !}
\qquad\Rightarrow\qquad
e^{ix}
=
\sum_{n=0}^{\infty} \frac{ix^{n}}{n !}.
\end{equation*}
Die komplexe Reihe bringen wir durch den Gebrauch von Euler in eine sehr bekannte Form  
\begin{equation}
e^{ix}
=
\sum_{n=0}^{\infty} \frac{(ix)^{n}}{n !}
=
\frac{(ix)^{0}}{0 !}+\frac{(ix)^{1}}{1 !}+\frac{(ix)^{2}}{2 !}+\frac{(ix)^{3}}{3 !}+\cdots
\end{equation}

Man sieht gleich, dass die komplexen Zahlen bei den geraden Exponenten verschwinden und bei denn ungeraden Exponenten bestehen bleiben. 
Sortiert man nun das Polynom nach komplexen und reellen Werten erkennt man die $cis$-Form 
\begin{align*}
e^{ix}
&=
\left(1-\frac{x^{2}}{2 !}+\frac{x^{4}}{4 !}-\ldots\right)+\mathrm{i} \cdot\left(x-\frac{x^{3}}{3 !}+\frac{x^{5}}{5 !}-\ldots\right)
\\
&=
\cos(x)+i\cdot \sin(x).
\end{align*}
und kann diese beiden Teile als Sinus und Kosinus aufschreiben,
wobei die Sinusfunktion 
\begin{equation*}
\sin (x)
=
\frac{x}{1 !}-\frac{x^{3}}{3 !}+\frac{x^{5}}{5 !} \mp \cdots
=
\sum_{n=0}^{\infty}(-1)^{n} \frac{x^{2 n+1}}{(2 n+1) !}
\end{equation*}
und die Kosinusfunktion 
\begin{equation*}
\cos (x)
=
\frac{x^{0}}{0 !}-\frac{x^{2}}{2 !}+\frac{x^{4}}{4 !} \mp \cdots
=
\sum_{n=0}^{\infty}(-1)^{n} \frac{x^{2 n}}{(2 n) !}
\end{equation*}
beide als einzelne Summen ausgeschrieben werden können.



\subsection{Padé-Approximation erstellen
	\label{pade:subsection:Pade_erstellen}}

Um eine Padé-Approximation zu erstellen startet man immer mit der Potenzreihe der zu approximierenden Funktion.
Diese Potenzreihe der Form 
\begin{equation}
f(x)=\sum_{n=0}^{\infty} c_{n} (z)^{n} 
\end{equation}
(\textcolor{red}{Achtung $c_n$ wird statt $a_n$ ab hier verwendet}) 
möchte man dann durch einen rationalen Bruch 
\begin{equation}
[L/M]
=
\frac{a_0 + a_1 z + \dots + a_L z^L}{b_0 + b_1 z + \dots + b_M z^M}
+O(z^{L+M+1})
\end{equation}
beschreiben.
Der erste Koeffizient des Nennerpolynoms wird als $b_0 = 1$ bestimmt.
Dank dieser Definition haben wir $L+1$ bestimmbare Zähler- und $M$ bestimmbare Nenner Koeffizienten. 
Dies resultiert in $[L+M+1]$ unbekannten Koeffizienten, welche mit der Nummerierung der Potenzreihe $z^0, z^1, z^2,\dots , z^{L+M}$ übereinstimmt.

Um die einzelnen $a_L$ und $b_M$ Koeffizienten zu berechnen wir die Approximation mit der Potenzreihe gleichgesetzt
\begin{equation}
\sum_{n=0}^{\infty} c_{n} (z)^{n} 
=
\frac{a_0 + a_1 z + \dots + a_L z^L}{b_0 + b_1 z + \dots + b_M z^M}
+O(z^{L+M+1})
\end{equation}
wobei das ganze umgestellt
\begin{equation}
\sum_{n=0}^{\infty} c_{n} (z)^{n} 
\cdot
(b_0 + b_1 z + \dots + b_M z^M)
=
(a_0 + a_1 z + \dots + a_L z^L) 
+
O(z^{L+M+1}).
\end{equation}
und als lineares Gleichungssystem betrachtet werden kann.
Diese lineare Gleichungen können in der Matrixform aufgeschrieben
\[
\renewcommand\arraystretch{1.25}
\begin{pmatrix}
c_{L-M+1} & c_{L-M+2} & c_{L-M+3} &\dots & c_{L}\\
c_{L-M+2} & c_{L-M+3} & c_{L-M+4} &\dots & c_{L+1}\\
c_{L-M+3} & c_{L-M+4} & c_{L-M+5} &\dots & c_{L+2}\\
\vdots & \vdots  & \vdots  &  & \vdots \\
c_{L} & c_{L+1} & c_{L+2} &\dots & c_{L+M-1}\\
\end{pmatrix}
\cdot
\begin{pmatrix}
b_{M}\\
b_{M-1}\\
b_{M-2}\\
\vdots \\
b_{1}\\
\end{pmatrix}
=
-
\begin{pmatrix}
c_{L+1}\\
c_{L+2}\\
c_{L+3}\\
\vdots \\
c_{L+M}\\
\end{pmatrix}
\label{pade:bKoeff}
\]
und nach den $b_M$ Koeffizienten aufgelöst werden.
Die dazugehörigen $a_L$ Koeffizienten können dann aus den $b_M$ und den $c_n$ Koeffizienten 
\begin{equation}
\begin{array}{l}
a_{0}=c_{0} \\
a_{1}=c_{1}+b_{1} c_{0} \\
a_{2}=c_{2}+b_{1} c_{1}+b_{2} c_{0} \\
\vdots \\
a_{L}=c_{L}+\displaystyle\sum_{n=1}^{\min (L, M)} b_{n} c_{L-n}
\end{array}
\label{pade:aKoeff}
\end{equation}
berechnet werden.
Wenn die Berechnungen der $a_L$ und $b_M$ Koeffizienten funktioniert hat, haben wir einen $[L/M]$ Padé-Approximant konstruiert.
Dieser Approximant stimmt mit 
\begin{equation}
\sum_{n=0}^{\infty} c_{n} z^{n}
\end{equation}
bis zu der Ordnung $z^{[L+M]}$ überein.

Nehmen wir das Beispiel der Einleitung
\begin{equation}
f(x)
=
\left(\frac{1+2x}{1+x}\right)^{\frac{1}{2}}
\approx
1+\frac{1}{2}x - \frac{5}{8}x^2+\frac{13}{16}x^3 -\frac{141}{128}x^4 +\frac{399}{256}x^5 - \frac{2353}{1024}x^6 + \frac{7205}{2048}x^7 \mp \cdots
\label{pade:bspPotenz}
\end{equation}
von welchem wir nun den Padé-Approximanten der dritten Ordnung $R_{[3/3]}$ berechnen möchten.
Zuerst Kreieren wir die Matrix um die $b_M$ Koeffizienten zu berechnen
\[
\renewcommand\arraystretch{1.25}
\begin{pmatrix}
c_{1} & c_{2} & c_{3}\\
c_{2} & c_{3} & c_{4}\\
c_{3} & c_{4} & c_{5} \\
\end{pmatrix}
\cdot
\begin{pmatrix}
b_{3}\\
b_{2}\\
b_{1}\\
\end{pmatrix}
=
-
\begin{pmatrix}
c_{4}\\
c_{5}\\
c_{6}\\
\end{pmatrix}
\]
in welche die Koeffizienten der Potenzreihe \ref{pade:bspPotenz} eingesetzt werden.
\[
\renewcommand\arraystretch{1.25}
\begin{pmatrix}
\frac{1}{2} & -\frac{5}{8} & \frac{13}{16}\\
-\frac{5}{8} & \frac{13}{16}& -\frac{141}{128}\\
\frac{13}{16} & -\frac{141}{128} & \frac{399}{256} \\
\end{pmatrix}
\cdot
\begin{pmatrix}
b_{3}\\
b_{2}\\
b_{1}\\
\end{pmatrix}
=
-
\begin{pmatrix}
-\frac{141}{128}\\
\frac{399}{256}\\
-\frac{2353}{1024}\\
\end{pmatrix}
\]
Da es sich in diesem Fall um eine Matrix gleicher Dimensionen handelt kann die Inverse der Matrix $C$
\begin{equation}
C\cdot b = -c
\qquad\Rightarrow\qquad
b = C^{-1} \cdot -c
\end{equation}
genommen werden um die Gleichung 

\[
\renewcommand\arraystretch{1.25}
\begin{pmatrix}
b_{3}\\
b_{2}\\
b_{1}\\
\end{pmatrix}
=
\begin{pmatrix}
\frac{1}{2} & -\frac{5}{8} & \frac{13}{16}\\
-\frac{5}{8} & \frac{13}{16}& -\frac{141}{128}\\
\frac{13}{16} & -\frac{141}{128} & \frac{399}{256} \\
\end{pmatrix}^{-1}
\cdot
-
\begin{pmatrix}
-\frac{141}{128}\\
\frac{399}{256}\\
-\frac{2353}{1024}\\
\end{pmatrix}
=
\begin{pmatrix}
\frac{169}{64}\\
\frac{47}{8}\\
\frac{17}{4}\\
\end{pmatrix}
\]
zu lösen.
Die $a_L$ Koeffizienten 
\begin{equation}
\begin{array}{l}
a_{0}=c_{0} \\
a_{1}=c_{1}+b_{1} c_{0} \\
a_{2}=c_{2}+b_{1} c_{1}+b_{2} c_{0} \\
a_{3}=c_{3}+b_{1} c_{2}+b_{2} c_{1} + b_{3} c_{0} \\
\end{array}
\qquad\Rightarrow\qquad
\begin{array}{l}
a_{0}=1 \\
a_{1}=\frac{1}{2}+\frac{17}{4} \cdot 1 = \frac{19}{4}\\
a_{2}=-\frac{5}{8} +\frac{17}{4}\cdot \frac{1}{2} +\frac{47}{8} \cdot 1 = \frac{59}{8} \\
a_{3}=\frac{13}{16}+\frac{17}{4}\cdot -\frac{5}{8} + \frac{47}{8}\cdot \frac{1}{2} + \frac{169}{64} \cdot 1 = \frac{239}{64}\\
\end{array}
\end{equation}
können nun aus den $b_M$ und den $c_n$ Koeffizienten berechnet werden. 
Daraus resultiert der gesamte Padé-Approximant 
\begin{equation}
R_{[3, 3]}(x)
=
\frac{1+\frac{19}{4}x+\frac{59}{8}x^2+\frac{239}{64}x^3}
{1+\frac{17}{4}x+\frac{47}{8}x^2+\frac{169}{64}x^3}
\end{equation}
der dritten Ordnung welcher aus der Taylorreihe der sechsten Ordnung gewonnen werden konnte.
Es fällt gleich auf, dass dieser Approximant der dritten Ordnung aus komplett anderen Polynomen zusammengesetzt ist, als der Approximant zweiter Ordnung des obigen Beispiels \ref{pade:bspordnung2}.



Dieser Aufwand um einen Approximant zu berechnen muss jedoch nicht immer betrieben werden.
Es gibt Funktionen welche glücklicherweise die Padé-Approximanten in einem Muster angeordnet haben, sodass für die Konstruktion der Nenner und Zähler- Polynome eigene Formeln gefunden wurden.
Wie man auf diese Formeln kommt, ist in der weiterführenden Literatur \cite{pade:Baker2009} beschrieben.
Die Exponentialfunktion, welche für viele Anwendungen gebraucht wird, besitzt eine Formel \cite{pade:moler} für die Konstruktion der Zähler- und Nenner Polynome.
Wir verwenden nun für den Approximanten $R_{[L/M]}$ die Notation
\begin{equation}
R_{[L/M]}(x)
=
\frac{P_{[L/ M]}(x)}{Q_{[L, M]}(x)} \approx e^{-x}
\end{equation}
wobei $P_{[L/ M]}(x)$ das Zählerpolynom
\begin{equation}
P_{[L/ M]}(x)
=
\sum_{n=0}^{L} \frac{(L+M-n) ! L !}{(L+M) ! n !(L-n) !}(-x)^{n}
\label{pade:expP}
\end{equation}
und $Q_{[L/ M]}(x)$ das Nennerpolynom
\begin{equation}
Q_{[L/ M]}(x)
=
\sum_{n=0}^{M} \frac{(L+M-n) ! M !}{(L+M) ! n !(M-n) !} x^{n}.
\label{pade:expQ}
\end{equation}
 der Padé-Approximation ist.

Es ist gängige Praxis die Padé-Approximanten in einer Tabelle 
\begin{center}
	\begin{tabular}{c| c c c c}
		
		$[L/M]$ 	& 0 		& 1 		& 2 		& $\cdots$ \\
		\hline
		0 		& $[0/0]$ 	& $[1/0]$ 	& $[2/0]$ 	& $\dots$ \\
		1 		& $[0/1]$ 	& $[1/1]$ 	& $[2/1]$ 	& $\dots$ \\
		2 		& $[0/2]$ 	& $[1/2]$ 	& $[2/2]$ 	& $\dots$ \\
		$\vdots$ 	& $\vdots$ 	& $\vdots$  	& $\vdots$  	&  \\
	\end{tabular}
\end{center}
darzustellen.
Diese Darstellung dient zur Übersicht der einzelnen Approximanten und macht deutlich das alle Approximanten der Padé-Approximation individuell sind. 
Die Polynome sehen bei jeder Ordnung komplett verschieden aus und werden nicht wie bei der Taylorreihe einfach immer länger. 


In diese Tabelle können nun die einzelnen Approximanten einer Funktion eingefüllt werden.
Als Beispiel wurde die Exponentialfunktion, mit den dazugehörigen $[L/M]$ Approximanten bis Approximant $[2/2]$, mit Hilfe der Formeln \ref{pade:expP} und \ref{pade:expQ} berechnet  
\begin{center}
	\begin{tabular}{c| c c c }
		$[L/M]$ 	& 0 		& 1 		& 2 	 \\
		\hline
		0 		& $\frac{1}{1}$ 	& $\frac{1+z}{1}$ 	& $\frac{2+2 z+z^{2}}{2}$ \\
		1 		& $\frac{1}{1-z}$ 	& $\frac{2+z}{2-z}$ 	& $\frac{6+4 z+z^{2}}{6-2 z}$ 	 \\
		2 		& $\frac{2}{2-2 z+z^{2}} $ 	& $\frac{6+2 z}{6-4 z+z^{2}}$ 	& $\frac{12+6 z+z^{2}}{12-6 z+z^{2}}$ 	 \\
	\end{tabular}
\end{center}
und in die Tabelle eingefügt.













