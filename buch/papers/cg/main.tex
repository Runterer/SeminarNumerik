%
% main.tex -- Paper zum Thema <cg>
%
% (c) 2020 Hochschule Rapperswil
%
\chapter{Die Methode der konjugierten Gradienten\label{chapter:cg}}
\lhead{Konjugierte Gradienten}
\begin{refsection}
\chapterauthor{Raphael Unterer}
\index{Unterer, Raphael}%
\index{konjugierte Gradienten}%

{\parindent0pt
Die} Methode der konjugierten Gradienten (kurz CG) bezeichnet einen schnellen Algorithmus zum Lösen grosser linearer Gleichungssysteme.
Grosse lineare Gleichungssysteme zu lösen ist ein klassisches numerisches Problem.
Diese treten in diversen Anwendungen auf, unter anderem beim Lösen von partiellen Differenzialgleichungen.

Im Kapitel \ref{chapter:linsys} wurden bereits einige numerische Algorithmen vorgestellt um lineare Gleichungssysteme approximativ zu lösen.
Die Methode der konjugierten Gradienten löst im Gegensatz dazu ein lineares Gleichungssystem nicht approximativ sondern exakt.
Dazu benötigt CG genau $N$ Schritte, wobei $N$ die Anzahl Gleichungen und Unbekannten bezeichnet.

Im Rahmen dieses Kapitels wird dieser CG-Algorithmus hergeleitet und einige Untersuchungen werden vorgenommen.

\input{papers/cg/voraussetzungen.tex}
\input{papers/cg/steepest_descent.tex}
\input{papers/cg/herleitung_algorithmus.tex}
\section{Algorithmus\label{cg:section:algorithmus}}
\rhead{Algorithmus}

In diesem Abschnitt wird der ganze Algorithmus noch einmal formell aufgeschrieben.
Diese Formulierung des Algorithmus bildet die Grundlage für eine erfolgreiche Implementation.
Die Resultate einer solchen (einfachen) Implementation folgen im nächsten Abschnitt.

\begin{enumerate}
	\item Wähle initiales $x_1$ zufällig
	\item Berechne die erste Abstiegsrichtung als $d_1 = r_1 =  b-Ax_1$
	\item Berechne die optimale Schrittlänge  $ \alpha	= 	\displaystyle  \frac{\langle d_k , r_k \rangle}{\langle d_k , d_k \rangle_A} 
																			= \frac{d_k^T  r_k}{d_k^T A d_k }$
	\item Führe den Schritt aus $x_{k+1} = x_k + \alpha d_k$
	\item Berechne das neue Residuum $r_{k+1} = b-Ax_{k+1}$
	\item Falls $r_{k+1} = 0$: Beende den Algorithmus
	\item Berechne neue Abstiegsrichtung $d_{k+1} = d_{k+1}	= 	r_{k+1} - \displaystyle \frac{\langle d_k , r_{k+1} \rangle_A}{\langle d_k , d_k \rangle_A} d_k 
															= r_{k+1} - \displaystyle \frac{d_k^T A r_{k+1}}{d_k^T A d_k} d_k $
	\item Wiederhole ab Punkt 3.
\end{enumerate}

\input{papers/cg/ergebnisse.tex}

\printbibliography[heading=subbibliography]
\end{refsection}
