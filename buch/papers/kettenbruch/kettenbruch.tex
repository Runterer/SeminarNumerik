%
% kettenbruch.tex -- Paper zum Thema <kettenbruch>
%
% (c) 2020 Hochschule Rapperswil
%
\documentclass{book}
\usepackage{etex}
\usepackage{geometry}
\geometry{papersize={170mm,240mm},total={140mm,200mm},top=21mm,bindingoffset=10mm}
\usepackage[english,ngerman]{babel}
\usepackage[utf8]{inputenc}
\usepackage[T1]{fontenc}
\usepackage{cancel}
\usepackage{times}
\usepackage{amsmath,amscd}
\usepackage{amssymb}
\usepackage{amsfonts}
\usepackage{amsthm}
\usepackage{graphicx}
\usepackage{fancyhdr}
\usepackage{textcomp}
\usepackage{txfonts}
\newcommand\hmmax{0}
\newcommand\bmmax{0}
\usepackage{bm}
\usepackage{epic}
\usepackage{verbatim}
%\usepackage{suffix}
\usepackage{paralist}
\usepackage{makeidx}
\usepackage{array}
\usepackage{hyperref}
\usepackage{subfigure}
\usepackage{tikz}
\usepackage{pgfplots}
\usepackage{pgfplotstable}
\usepackage{pdftexcmds}
\usepackage{pgfmath}
\usepackage[autostyle=false,english=american]{csquotes}
\usepackage{wasysym}
\usepackage{environ}
\usepackage{appendix}
\usepackage[all]{xy}
\usetikzlibrary{calc,intersections,through,backgrounds,graphs,positioning,shapes,arrows,fit,math}
\usetikzlibrary{patterns,decorations.pathreplacing}
\usetikzlibrary{decorations.pathreplacing}
\usetikzlibrary{external}
\usetikzlibrary{datavisualization}
\usepackage[europeanvoltages,
            europeancurrents,
            europeanresistors,   % rectangular shape
            americaninductors,   % "4-bumbs" shape
            europeanports,       % rectangular logic ports
            siunitx,             % #1<#2>
            emptydiodes,
            noarrowmos,
            smartlabels]         % lables are rotated in a smart way
           {circuitikz}          %
\usepackage{siunitx}
\usepackage{tabularx}
\usetikzlibrary{arrows}
\usepackage{algpseudocode}
\usepackage{algorithm}
\usepackage{gensymb}
\usepackage{mathtools}

% import the listing styles

\usepackage{caption}
\usepackage[mode=buildnew]{standalone}
\usepackage[backend=bibtex]{biblatex}
\begin{document}
\def\chapterauthor#1{{\large #1}\bigskip\bigskip}

\newenvironment{beispiel}{%
\begin{proof}[Beispiel]%
\renewcommand{\qedsymbol}{$\bigcirc$}
}{\end{proof}}
\setcounter{page}{352}
\setcounter{chapter}{16}
\allowdisplaybreaks
\renewcommand{\floatpagefraction}{0.7}
\pagestyle{fancy}
\lhead{}
\rhead{}

\chapter{Kettenbrüche\label{chapter:kettenbruch}}
\lhead{Kettenbrüche}
\begin{refsection}
\chapterauthor{Benjamin Bouhafs-Keller}

%
% einleitung.tex -- Beispiel-File für die Einleitung
%
% (c) 2020 Prof Dr Andreas Müller, Hochschule Rapperswil
%
\section{Einleitung\label{steps:section:einleitung}}
\rhead{Einleitung}

Numerische Verfahren zur Berechnung von Differentialgleichungen wie das Euler- oder Runge-
Kutta Verfahren bekanntlich auf iterativem Vorgehen in Schritten, das bedeutet, dass der Verlauf
der Kurve mit vielen Einzelschritten bis zum gesuchten Endpunkt angenähert wird und nicht exakt
bestimmt werden kann. Um den Fehler möglichst klein zu halten,
kann bei gegebenem Lösungsverfahren beispielsweise die Schrittlänge zwischen den einzelnen Punkten verkleinert werden, wodurch
jedoch der Rechenaufwand steigt. Die für die geforderte Genauigkeit nötige Schrittweite richtet sich
nach dem Punkt, an dem die "Krümmung" der Kurve am grössten ist. Die Aufgabe der Schrittlän-
gensteuerung ist es nun, die Schrittlänge laufend der Dynamik der Kurve anzupassen, sodass der
Zielpunkt mit möglichst wenig Schritten und der erforderlichen Genauigket berechnet werden
kann.



%
% rationalezahlen.tex -- 
%
% (c) 2020 Benjamin Bouhars-Keller
%
\section{Rationale Zahlen
\label{kettenbruch:section:Zahlen}}
\rhead{Rationale Zahlen}
\subsection{Definition}
Die Kettenbruch entwicklung von $x \in \mathbb{R}$ bricht genau dann nach endlich
vielen Schritten ab, wenn $x$ rational ist.
Die Menge aller rationalen Zahlen wird mit $\mathbb{Q}$ bezeichnet.
Ein endlicher Kettenbruch ist ein Bruch der Form
\begin{equation}
a_0 + \cfrac{1}{a_1+\cfrac{1}{a_2+\cfrac{\cdots}{\cdots+\cfrac{1}{a_{n-1} + \cfrac{1}{a_n}}}}}
\end{equation}
in welchem $a_0, a_1,\dots,a_n$ und $b_1,b_2,\dots,b_n$ ganze Zahlen
darstellen, die mit Ausnahme möglicherweise von $a_0$ alle positiv sind.
Die Kettenbruchentwicklung von $x \in \mathbb{R}$ bricht genau dann
nach endlich vielen Schritten ab, wenn $x$ rational ist. So bilden
rationale Zahlen endliche Kettenbrüche.

\subsection{Euklidischer Algorithmus}
Die Umwandlung einer rationalen Zahl in einen Kettenbruch erfolgt
mit Hilfe des euklidischen Algorithmus.
Als Beispiel rechnen wir für $\frac{17}{10} = [1;1,2,3]$.
\begin{beispiel}
\begin{equation}
\frac{17}{10}
=
1 + \frac{7}{10}
=
1 + \cfrac{1}{\frac{10}{7}}
=
1 + \cfrac{1}{1+\frac{3}{7}}
=
1 + \cfrac{1}{1+\cfrac{1}{1+\cfrac{1}{\frac{7}{3}}}}
=
1 + \cfrac{1}{1+\cfrac{1}{2+\frac{1}{3}}}
\end{equation}
\begin{align*}
17 &= 1\cdot + 7 \\
10 &= 1\cdot 7 + 3 \\
7 &= 2\cdot 3 + 1 \\
3 &= 3\cdot 1 + 0
\end{align*}
Diese Methode kann auch umgekehrt angewendet werden.
Berechnen wir nun den Kettenbruch in einem Bruch zurück.
\begin{equation}
[1;1,2,3] \rightarrow	2 + \frac{1}{3} = \frac{7}{3}
						1 + \frac{3}{7} = \frac{10}{7}
						1 + \frac{7}{10} = \frac{17}{0}
\end{equation}
\end{beispiel}
Wie man im Beispiel sieht, ist die intuitive Berechnung der
Näherungsbrüche eines Kettenbruchs, indem man ihn von unten her
auflöst, sehr umständlich. Durch ein rekursives Bildungsgesetz für
Zähler und Nenner kann man die Berechnung erheblich vereinfachen.
Ausserdem kann man mit Hilfe dieser Rekursionsformel die Grenzwerte
unendlicher Kettenbrüche untersuchen.

Jede rationale Zahl $x$ lässt sich auf eindeutige Weise in einen
endlichen Kettenbruch entwicklen, dessen letzer Teilnenner grösser
oder gleich 2 ist.

\subsection{Rekursionsformel}
Die Berechnung der Konvergenten (siehe Abschnitt 10.3.3) eines Kettenbruches 
kann erheblich vereinfacht werden, indem eine Rekursionsformeln für den Zähler 
und den Nenner einführt wird.
\begin{beispiel}
\begin{align*}
A_0 &= b_0                     &   B_0 &= 1                                  \\
A_1 &= a_1a_0 + 1              &   B_1 &= a_1                                \\
A_k &= a_kA_{k-1} + A_{k-2}    &   B_k &= a_kB_{k-1} + B_{k-2} &(i &\ge 1),   \\
A_2 &= b_2b_1b_0 + b_2 + b_0   &   B_2 &= b_2b_1 + 1		             \\
A_3 &= b_3b_2b_1b_0 + b_3b_2 + b_3b_0 + b_1b_0 + 1 & B_3 &= b_3b_2b_1 + b_3 + b_1
\end{align*}
Dies lässt sich auch durch die folgende Matrizenschreibweise ausdrücken:
\begin{equation}
P_0 = 	\begin{pmatrix}
			A_{_1}&	A_{_2}\\
			B_{_1}&	A_{_2}
		\end{pmatrix}
		=\begin{pmatrix}
			1&	0\\
			0&	1
		\end{pmatrix}
\end{equation}
\begin{equation}
P_i = 	\begin{pmatrix}
			A_i&	A_{i-1}\\
			B_i&	B_{i-1}
		\end{pmatrix}
		=\begin{pmatrix}
			A_{i-1}&	A_{i-2}\\
			B_{i-1}&	B_{i-2}
		\end{pmatrix} 
		\begin{pmatrix}
			b_i	&	1\\
			1	&	0
		\end{pmatrix} 
\end{equation}
Rekursives Einsetzen ergibt:
\begin{equation}
		\begin{pmatrix}
			A_i&	A_{i-1}\\
			B_i&	B_{i-1}
		\end{pmatrix}
		=\begin{pmatrix}
			b_0	&	1\\
			1	&	0
		\end{pmatrix}
		\begin{pmatrix}
			b_1	&	1\\
			1	&	0
		\end{pmatrix}
		\cdots
		\begin{pmatrix}
			b_i	&	1\\
			1	&	0
		\end{pmatrix} 
		=\displaystyle\prod_{j=0}^{i}\begin{pmatrix}
			b_j	&	1\\
			1	&	0
		\end{pmatrix}
\end{equation}
Aus diesen Überlegungen resultiert eine Aussage, von deren letztem 
Teil wir noch sehr oft Gebrauch machen werden:
Somit gilt 
\begin{equation}
\det(P_i) = (-1)^i
\end{equation}
und für alle $i$:
\begin{equation}
A_iB_{i-1} - B_iA_{i-1} = (-1)^i
\end{equation}
\subsubsection{Beispiel}
Für $x = [b_0;b,\cdots,b_{i-1},\beta_1]$ gilt
\begin{equation}
x = \frac{A_{i-1}\beta_i + A_{i-2}}{B_{i-1}\beta_i + B_{i-2}}
\qquad \text{für alle $i \ge 0$.}
\end{equation}
Aus der Rekursionsformeln ergeben sich folgende Gleichung
\begin{equation}
[b_0;b_1,\cdots,b_n]
=
\frac{A_n}{B_n} = \frac{b_nA_{n-1} + A_{n-2}}{b_nB_{n-1} + B_{n-2}}
\end{equation}
und analog
\begin{equation}
[b_0;b,\cdots,b_{i-1},\beta_1] = [b_0;b_1,\cdots,b_{i-1},\beta_i]
=
\frac{A_{i-1}\beta_i + A_{i-2}}{B_{i-1}\beta_i + B_{i-2}}
\end{equation}
\end{beispiel}
Hiermit haben wir die wichtigsten Zusammenhänge in Bezug auf die Näherungszahlen herausgearbeitet.


%
% irrationalezahlen.tex 
%
% (c) 2020 Benjamin Bouhafs-Keller
%
\section{Irrationale Zahlen
\label{kettenbruch:section:Irrationale Zahlen}}
\rhead{Irrationale Zahlen}
\subsection{Definition}
Die Menge aller reellen Zahlen bezeichnet man mit $\mathbb{R}$.
Irrationale Zahlen bilden unendliche Kettenbrüche, d.~h.~sind durch
eine periodische oder nicht periodische Kettenbruchentwicklung
ausgezeichnet.
Ein unendlicher regelmässiger Kettenbruch wird in folgender Form dargestellt
\begin{equation}
a_0 + \cfrac{1}{a_1+\cfrac{1}{a_1+\cfrac{1}{a_3+\cfrac{1}{\dots}}}}
\end{equation}
wobei $a_0,a_1,a_2,\dots$ eine unendliche Folge von positiven
ganzen Zahlen bilden. Sie sind auch wie beim endlichen Kettenbruch
alle bis auf möglicherweise $a_0$ positiv.


Zunächst sollen einige Beispiele für die Kettenbruchenentwicklung
irrationaler Zahlen betrachtet werden.

\subsection{Periodische Kettenbrüche}
In diesem Abschnitt wollen wir nun auf eine spezielle Form eingehen
und zwar auf unendliche regelmässige Kettebrüche, die ein bemerkenswertes
Bildungsgesetz befolgen. Das besondere an diesen Kettenbrüchen ist,
dass gleiche Teilnenner wiederholt auftreten.
Für den Kettenbruch $[3;1,2,1,6,1,2,1,6,\dots]$ heisst das: den
Kettenbruch die Periode 1,2,1,6 mit der Periodenlänge $n=4$ beträgt
und wird in der Form $[3;\overline{1,2,1,6}]$ geschrieben.

\begin{beispiel}
Betrachten wir den periodischen einfachen Kettenbruch $[3;\bar{6}] = (3,6,6,6,\dots)$.
\begin{equation}
[3;\bar{6}]
=
3 + \cfrac{1}{6+\cfrac{1}{6+\cfrac{1}{6+\cfrac{1}{6+\dots}}}}
=
x
\end{equation}
Die Euklidische Methode mit der rekursive Bildungsgesetz für Zähler
und Nenner würde hier unendlich sein und deshalb schwierig zu
berechnen.
Um diesen Kettenbruch vollständig darzustellen müssen wir ein Muster
erzeugen. Daher werden nur die ersten Brüche (Zahlen) betrachtet.

\begin{equation}
[3;6,6,6]
=
3 + \cfrac{1}{6+\cfrac{1}{6+\cfrac{1}{6}}}
=
3 + \cfrac{1}{6+\cfrac{1}{\frac{37}{6}}}
=
3 + \cfrac{1}{6+\frac{6}{37}}
=
3 + \cfrac{1}{\frac{228}{37}}
=
3 + \frac{37}{228}
=
\frac{684+37}{228}
=
\frac{721}{228}
\approx
3.162280702
\end{equation}
Wenn wir den periodischen Kettenbruch so verändern das die Kettenbruchentwicklung 
immer mit 6 vortläuft dann können wir unser Kettenbruch als $x$ bezeichnen 
und darauf 3 addieren. Anders formuliert ist wie in einem periodischen Kettenbruch 
ein {\color{red}Teil oder inneren kettenbruch} der similär ist wie der ganze Kettenbruch. Dies ergibt 
folgendes Muster das wir infolge quadratische
Gleichungen lösen können:
\begin{align*}
y = x+3 &= 6 + \cfrac{1}{\color{red}6+\cfrac{1}{6+\dots}} = [6;\bar{6}]
\\
\Rightarrow y &= 6 + \frac{1}{y}	&&\vert\;\cdot y
\\
\Rightarrow y^2 &= 6y + 1
\\
\Rightarrow y &= 3\pm \sqrt{9+1} = 3 \pm \sqrt{10}\qquad\Rightarrow\qquad x = \sqrt{10}
\approx
3.16227766
\end{align*}
\end{beispiel}
Mit dieser Gleichung haben wir die Menge der Zahlen mit periodischer 
Kettenbruchentwicklung bestimmt. 

\begin{beispiel}
Betrachten wir einen neuen Kettenbruch, $[\overline{2,3}] =  [2;3,2,3,\dots]$.
Sein Wert ist als unendlicher Kettenbruch irrational und lässt sich
wie folgt berechnen. Setzen wir $x:=(\overline{2,3})$, dann gilt
\begin{equation}
x
=
2 + \cfrac{1}{3+\cfrac{1}{\color{red}2+\cfrac{1}{3+\dots}}}
=
2 + \cfrac{1}{3+\frac{1}{\color{red}x}}.
\end{equation}
Dies führt auf die quadratische Gleichung 
\begin{align*}
x &= 2 + \cfrac{1}{3+\frac{1}{x}}
\\
x - 2 &= \cfrac{1}{3+\frac{1}{x}}
\\
(x - 2)\biggl(3 + \frac{1}{x}\biggr) &= 1
\\
3x - 6 - \frac{2}{x} &= 0 &&\vert \times\frac{x}{3}
\\
x^2 - 2x - \frac{2}{3} &= 0
\end{align*}
was die positive Lösung $x = \frac{3+\sqrt{15}}{3}$ liefert.
Der oben aufgeführte Kettenbruch $x$ ist ein Beispiel für periodische
einfache Kettenbrüche, die Nullstelle eines quadratischen Polynoms
mit rationalen Koeffizienten ist. Anders gesagt ist die reelle
Irrationalzahl $x$ Wurzel einer quadratischen Gleichung:
\begin{equation}
ax^2 + bx + c = 0
\end{equation}
dann ist die Kettenbruchentwicklung von $x$ periodisch, das bedeutet
die Existenz einer Schranke $n_0$ und einer Periode $k \in \mathbb{N}$
mit $x_n+k = x_n$ für alle $n\ge n_0$.
\end{beispiel}
\subsubsection{Satz von Euler-Lagrange}
Jeder periodische regulärer Kettenbruch ist eine quadratische
Irrationalzahl und umgekehrt. $x \in \mathbb{R}$ hat genau dann eine 
Darstellung als unendlicher, periodischer Kettenbruch, wenn $x$ eine
reell-quadratische Irrationalzahl ist (d.h. $x \notin \mathbb{Q}$ ist Lösung
einer quadratischen Gleichung $aX^2 + bX + c = 0, a \neq 0$ mit rationalen 
Koeffizienten $a,b,c$).

\subsection{Nicht periodische Kettenbrüche}
Es stellen sich dieselben Fragen wie im vorangegangenen Abschnitt.
Neu hinzu kommt das Problem, ob bzw. wann die Kettenbruchentwicklung
überhaupt konvergiert.
Für eine unendliche Folge $x_0,x_1,\dots$ ist der Kettenbruch
$[x_0;x_1,\dots]$ nur dann definiert wenn die Folge der Näherungsbrüche
$(\frac{p_n}{q_n})$ konvergiert. In diesem Fall hat der unendliche
Kettenbruch $[x_0;x_1,\dots]$ den Wert
\begin{equation}
\lim_{n\to\infty} [x_0;x_1,\dots,x_n]
\end{equation}
oder anders dargestellt
\begin{equation}
\omega
=
x_0 + \cfrac{1}{x_1+\cfrac{1}{x_2+\frac{1}{x_n+\dots}}}
\end{equation}
Folgt $\omega > 0$ durch einen unendliche Kettenbruch darstellbar
ist, wenn die endlichen Kettenbrüche $n$-ter Ordnung
$[x_0;x_1,x_2,\dots,x_n]$ gegen $\omega$ konvergieren.

\subsubsection{Beispiel}
\begin{beispiel}
Betrachten wir folgenden Kettenbruch
\begin{align*}
\frac{19}{51} &= [0;2,1,2,6]
\\
	K_0 &= [0] = 0
\\
	K_1 &= [0;2] = 0 + \frac{1}{2} = \frac{1}{2}
\\
	K_2 &= [0;2,1] = 0 + \cfrac{1}{2+\frac{1}{1}} = \frac{1}{3}
\\
	K_3 &= [0;2,1,2] = 0 + \cfrac{1}{2+\cfrac{1}{1+\frac{1}{2}}} = \frac{3}{8}
\\
	K_4 &= [0;2,1,2,6] = \frac{19}{51}
\end{align*}
Mit diesem Beispiel werden die Teilkettenbrüche und Näherung zum Endresultat ersichtlich.
Folge der Näherungsbrüche mit geradem (bzw. ungeradem) Index
ist streng monoton steigend (bzw. fallend) und jeder Näherungsbruch
mit geradem Index ist kleiner als jeder Näherungsbruch mit ungeradem Index.
\begin{enumerate}
\item
$K_0 < K_2 < K_4 < \cdots$
\item
$K_1 > K_3 > K_5 > \cdots$
\item
$K_{2s} < K_{2r+1}, r,s \in \mathbb{N}$
\end{enumerate}

Es gilt offensichtlich
$K_0 < K_2 < K_4 < \cdots < K_{2n} < \cdots < K_{2n+1} < \cdots < K_5
< K_3 < K_1$
und $\frac{19}{51}$ wird von jeweils zwei aufeinanderfolgenden
Näherungsbrüchen eingeschlossen.
\end{beispiel}

Die Folge der Näherungsbrüche des unendlichen Kettenbruchs
$[a_0;a_1,a_2,\dots]$ mit $a_0 \in \mathbb{Z}$ konvergiert 
von unten gegen einen Grenzwert, den wir als $\alpha$
bezeichnen. Anderseits sind die Näherungsbrüche mit ungeradem
Index nach unten begrenzt. Somit sind beide Folgen monoton und beschränkt 
und konvergieren in $\alpha$. Den erwähnten Ansatz gilt nur für reguläre Kettenbrüche, es 
wäre zum Beispiel nicht der Fall, wenn als Teilnenner beliebige positive reelle Zahlen 
zugelassen wären. Der Kettenbruch $[b_0;b_1,\dots]$ konvergiert genau dann, wenn die Summe
$\sum\limits_{i=0}^\infty b_i $ divergiert.

Es gibt auch Zahlen, deren Kettenbruchdarstellung gewisse
Regelmässigkeiten aufweisen, ohne periodisch zu sein. Zum Beispiel
die Identität $e = [2;1,2,1,1,4,1,1,6,1,\dots]$. Dieser
Kettenbruch ist nicht periodisch, die Teilnenner können aber durch
eine rekursive Folge bestimmt werden.

\subsubsection{Zusammenfassung}
\begin{itemize}
\item
Jede positive rationale Zahl lässt sich durch einen endlichen
Kettenbruch darstellen, und jeder endliche Kettenbruch stellt eine
positve rationale Zahl dar.
\item
Jeder unendliche Kettenbruch stellt eine positive irrationale Zahl
dar, und jede irrationale Zahl lässt sich durch einen unendliche
Kettenbruch darstellen.
\item
Jeder periodische Kettenbruch stellt eine quadratische Irrationalität
dar und jede quadratische Irrationalität lässt sich durch einen
periodischen Kettenbruch darstellen.
\end{itemize}

%
% loesung.tex -- Beispiel-File für die Beschreibung der Loesung
%
% (c) 2020 Prof Dr Andreas Müller, Hochschule Rapperswil
%
\section{Approximation
\label{kettenbruch:section:Approximation}}
\rhead{Approximation}

In der Einleitung wurde erwähnt, dass die Bestimmung von guten
Näherungsbrüchen eine wichtige Anwendung von Kettenbrüchen ist. Es
gilt nämlich, dass jeder Näherungsbruch der Kettenbruchentwicklung
einer reellen Zahl eine besonders gute rationale Näherung dieser
Zahl ist.

\subsection{Definition}

Eine rationale Zahl $\frac{a}{b}$ mit $b>0$ heisst beste Näherung
erster Art an eine reelle Zahl $x$, wenn es keine von $\frac{a}{b}$
verschiedene rationale Zahl mit gleichem oder kleinerem Nenner gibt,
die bezüglich des Absolutbetrages näher bei $x$ liegt.
Das heisst, dann gilt für alle rationalen Zahlen $\frac{c}{d} \ne
\frac{a}{b}$ mit $0<d\le b$:
\begin{equation}
\biggl|x-\frac{a}{b}\biggr| < \biggl| x-\frac{c}{d}\biggr|.
\end{equation}

\subsection{Näherungsgesetz}
Ziel dieses Abschnitt ist es, eine genügend gute Approximation der
Näherungsbrüche nachzuweisen. Gibt man sich eine beliebige Zahl $x$
vor, so kann man sich die Frage stellen, welche "unkürzbaren" Brüche
$\frac{p}{q}$ mit vorgegebenem Höchstnenner sich gut approximieren
lässt.

\begin{beispiel}
Näherung von $\pi$ mit dem unendliche Dezimalbruch:
$\pi = [3;7,15,1,292,1,1,1,2,1,3,1,14,2,\cdots]$
Die Näherung $3.14 = \frac{314}{100}$ ist eine Näherung. Aber
$\frac{22}{7} = 3.14285714\dots$ hat einen viel kleineren Nenner und
ist eine deutlich bessere Näherung von $\pi$.
Eine noch bessere Näherung ist der Kettenbruch
\begin{equation}
\frac{355}{113} = 3 + \cfrac{1}{7+\cfrac{1}{15+\cfrac{1}{1}}} = 3.1415\overline{92}
\end{equation}
Folgende Näherungswerte von $\pi$ können schnell und einfach gerechnet werden:
\begin{equation}
3,\frac{22}{7} \approx 3.143 ; \frac{333}{106} \approx 3.14151 ; \frac{355}{113} 
\approx 3.1415929 ; \frac{103993}{33102} \approx 3.1415926530 ; \cdots.
\end{equation}
\end{beispiel}
Die Bestapproximation ist einfach formuliert durch die Bestimmung
derjenigen rationalen Brüchen, welche von einer gegebenen rationalen
oder irrationalen Zahl einen festgelegten minimalen Abstand haben
und dabei einen möglichst kleinen positiven Nenner besitzen.

\subsubsection{Beispiel}
\begin{beispiel}
Die Funktion $\tan^{-1}(x)$ spielt bei der Berechnung von $\pi$ an vielen Stellen eine Rolle. 
Es gilt die Leibnizsche Reihe
\begin{equation}
\frac{\pi}{4} = \tan^{-1}(1)
=
1 - \frac{1}{3} + \frac{1}{5} - \frac{1}{7} + \frac{1}{9} - \frac{1}{11} +\dots
\end{equation}
Dank der Euler-Transformation gilt folgender Kettenbruch
\begin{equation}
\tan^{-1}(x) = x - \frac{x^3}{3} + \frac{x^5}{5} - \frac{x^7}{7} + \frac{x^9}{9} - 
\frac{x^{11}}{11}\dots,		\qquad x \leq 1
\end{equation}
Somit kann der Kettenbruch von $\tan^{-1}(x)$ folgendermassen dargestellt werden.
\begin{equation}
\tan^{-1}(x)
=
\cfrac{x}{1+\cfrac{x^2}{3+\cfrac{4x^2}{5+\cfrac{9x^2}{7+\cfrac{16x^2}{9+\cdots}}}}}
\qquad	(|x|< 1)
\end{equation}
Das Gleichungssystem kann umgeschrieben werden als Funktion $f_n$
\begin{equation}
f_n(x) = \frac{x}{1+}\frac{x^2}{3+}\frac{4x^2}{5+}\cdots\frac{(n-1)^2 x^2}{2n-1}
\qquad	(|n|\ge 2)
\end{equation}
\end{beispiel}
Hiermit kann nach $n$-te Bildung der Kettenbruchreaktion einen Grenzwert
erreichen:
\begin{equation}
\tan^{-1}(x) = \lim_{n\to\infty} f_n(x), \qquad (|x| < 1)
\end{equation}
Die Konvergenz der Funktion kann an einem Beispiel beurteilt werden. 
\begin{equation}
\tan^{-1}(1) = \pi/4 \approx 0.785398
\end{equation}

\begin{table}
\centering
\begin{tabular}{>{$}c<{$}>{$}l<{$}}
n	& f_n(1) 	\\
\hline
2	& 0.750000 	\\
3	& 0.791667 	\\
4	& 0.784314 	\\
5	& 0.785586 	\\
6	& 0.785366 	\\
7	& 0.785404	\\
8	& 0.785397	\\
9	& 0.785398	\\
\hline
\end{tabular}
\caption{Näherungsstufen mit Kettenbruchentwicklung von der Funktion $\tan^{-1}(1)$
\label{kettenbruch:tabelle}}
\end{table}

In wenigen und einfachen Schritten haben wir mit Hilfe einer
Kettenbruchentwicklung ein System gebildet das die Konvergenz der
Funktion $\tan^{-1}(x)$ vorantreibt und präzise Resultate liefert.


%
% problemstellung.tex -- Beispiel-File für die Beschreibung des Problems
%
% (c) 2020 Prof Dr Andreas Müller, Hochschule Rapperswil
%
\section{Folgerungen
\label{qr:section:folgerungen}}
\rhead{Folgerungen}
Um einen Vergleich anzustellen, wurden die beiden betrachteten Varianten wie beschrieben in einem Python-Skript implementiert. 
\index{Python}%
Der Datentyp der Matrixelemente ist dabei eine 32-Bit Floatingpoint-Zahl.
Beiden Implementationen wurde dann die Matrix (welche schon vorher benutzt wurde)
\begin{equation*}
	A(\epsilon)=
	\begin{pmatrix}
	1+\epsilon&1&1\\
	1&1+\epsilon&1\\
	1&1&1+\epsilon
	\end{pmatrix}
\end{equation*}
übergeben, allerdings mit $\epsilon$ als Laufvariable zwischen 0 und 0.004 mit Schrittweite $10^{-5}$.

In Abbildung \ref{qr:comp} ist der Winkel von $q_2$ und $q_3$ resultierend aus beiden Implementationen geplotted.
\begin{figure}[ht]
	\centering
	\includegraphics[width=0.9\textwidth]{papers/qr/pics/comp.pdf}
	\caption{Winkel zwischen $q_2$ und $q_3$.\label{qr:comp}}
\end{figure}
Klar ersichtlich ist, wie stabil die Implementation mit den Givens-Rotationen immer einen Winkel von $90^\circ$ liefert, die Implementation mit dem Gram-Schmidt-Orthonormalisierungsverfahren dagegen sehr instabil wirkt.

Die $QR$-Zerlegung über das Gram-Schmidt-Orthonormalisierungsverfahren liefert zwar eine sehr intuitive Methode, ist aber numerisch nicht stabil und sollte deshalb nicht so verwendet werden.
Bessere numerische Eigenschaften bringt eine Implementation mit Givens-Rotationen mit sich.


\printbibliography[heading=subbibliography]
\end{refsection}
\end{document}
