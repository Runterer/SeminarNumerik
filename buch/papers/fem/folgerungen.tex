%
% problemstellung.tex -- Beispiel-File für die Beschreibung des Problems
%
% (c) 2020 Prof Dr Andreas Müller, Hochschule Rapperswil
%
\section{Folgerungen
\label{fem:section:folgerungen}}
\rhead{Folgerungen}
In diesem Kapitel wurde die Bildung von Ansatzfunktionen in der Ebene eingeführt sowie die Eigenschaften einiger Typen aufgezeigt. Es wurde dabei gezeigt,
\begin{itemize}
	\item dass die Wahl der Ansatzfunktionen bestimmte Voraustzungen erfüllen müssen
	\item dass es unterschiedliche Arten von Ansatzfunktionen gibt
	\item was der Sinn dieser Ansatzfunktionen ist
	\item wie die Matrizen mit den Koeffizienten aufgestellt werden
\end{itemize}
Auch wurde gezeigt wie die Berechnungen vereinfacht werden durch die Transformation in ein Einheitsform wie dem Einheitsdreieck. 



